%------------------------------------------------

\begin{fullwidth}

Most academic programs that prepare students for a career in the type of work discussed in this book
spend a disproportionately small amount of time teaching their students coding skills, in relation to the share of
their professional time they will spend writing code their first years after graduating. Recent
Masters' program graduates that have joined our team tended to have very good knowledge in the theory of our
trade, but tended to require a lot of training in its practical skills. To us, it is like hiring architects
that can sketch, describe, and discuss the concepts and requirements of a new building very well, but do
not have the technical skill set to actually contribute to a blueprint using professional standards that can be used
and understood by other professionals during construction. The reasons for this are probably a topic
for another book, but in today's data-driven world, people working in quantitative economics research
must be proficient programmers, and that includes more than being able to compute the correct numbers.

This appendix first has a short section with instructions on how to access and use the code shared in
this book. The second section contains a the current DIME Analytics style guide for Stata code.
Widely accepted and used style guides are common in most programming languages,
and we think that using such a style guide greatly improves the quality
of research projects coded in Stata. We hope that this guide can  help to increase the emphasis in the Stata community on using,
improving, sharing and standardizing code style. Style guides are the most important tool in how
you, like an architect, draw a blueprint that can be understood and used by everyone in your trade.

\end{fullwidth}

%------------------------------------------------

\section{Using the code examples in this book}

You can access the raw code used in examples in this book in many ways.
We use GitHub to version control everything in this book, the code included.
To see the code on GitHub, go to: \url{https://github.com/worldbank/d4di/tree/master/code}.
If you are familiar with GitHub you can fork the repository and clone your fork.
We only use Stata's built-in datasets in our code examples,
so you do not need to download any data from anywhere.
If you have Stata installed on your computer, then you will have the data files used in the code.

A less technical way to access the code is to click the individual file in the URL above, then click
the button that says \textbf{Raw}. You will then get to a page that looks like the one at:
\url{https://raw.githubusercontent.com/worldbank/d4di/master/code/code.do}.
There, you can copy the code from your browser window to your do-file editor with the formatting intact.
This method is only practical for a single file at the time.
If you want to download all code used in this book, you can do that at:
\url{https://github.com/worldbank/d4di/archive/master.zip}. That link offers a \texttt{.zip} file download
with all the content used in writing this book, including the \LaTeX{} code used for the book itself. After
extracting the .zip-file you will find all the code in a folder called \texttt{/code/}.

\subsection{Understanding Stata code}

Regardless if you are new to Stata or have used it for decades, you will always run into commands that
you have not seen before or do not remember what they do. Every time that happens, you should always look
that command up in the helpfile. For some reason, we often encounter the conception that the helpfiles
are only for beginners. We could not disagree with that conception more, as the only way to get better at Stata
is to constantly read helpfiles. So if there is a command that you do not understand in any of our code
examples, for example \texttt{isid}, then write \texttt{help isid}, and the helpfile for the command \texttt{isid} will open.

We cannot emphasize enough how important we think it is that you get into the habit of reading helpfiles.

Sometimes, you will encounter code employing user-written commands,
and you will not be able to read those helpfiles until you have installed the commands.
Two examples of these in our code are \texttt{reandtreat} or \texttt{ieboilstart}.
The most common place to distribute user-written commands for Stata is the Boston College Statistical Software Components
(SSC) archive. In our code examples, we only use either Stata's built-in commands or commands available from the
SSC archive. So, if your installation of Stata does not recognize a command in our code, for example
\texttt{randtreat}, then type \texttt{ssc install randtreat} in Stata.

Some commands on SSC are distributed in packages, for example \texttt{ieboilstart}, meaning that you will
not be able to install it using \texttt{ssc install ieboilstart}. If you do, Stata will suggest that you
instead use \texttt{findit ieboilstart} which will search SSC (among other places) and see if there is a
package that has a command called \texttt{ieboilstart}. Stata will find \texttt{ieboilstart} in the package
\texttt{ietoolkit}, so then you will type \texttt{ssc install ietoolkit} instead in Stata.

We understand that this can be confusing the first time you work with this, but this is the best way to set
up your Stata installation to benefit from other people's work that they have made publicly available, and
once used to installing commands like this it will not be confusing at all.
All code with user-written commands, furthermore, is best written when it installs such commands
at the beginning of the master do-file, so that the user does not have to search for packages manually.

\subsection{Why we use a Stata style guide}

Programming languages used in computer science always have style guides associated with them.
Sometimes they are official guides that are universally agreed upon, such as PEP8 for
Python.\sidenote{\url{https://www.python.org/dev/peps/pep-0008/}} More commonly, there are well-recognized but
non-official style guides like the JavaScript Standard Style\sidenote{\url{https://standardjs.com/\#the-rules}} for
JavaScript or Hadley Wickham's\sidenote{\url{http://adv-r.had.co.nz/Style.html}} style guide for R.

Aesthetics is an important part of style guides, but not the main point. The existence of style guides
improves the quality of the code in that language produced by all programmers in the community.
It is through a style guide that unexperienced programmers can learn from more experienced programmers
how certain coding practices are more or less error-prone. Broadly-accepted style guides make it easier to
borrow solutions from each other and from examples online without causing bugs that might only be found too
late. Similarly, globally standardized style guides make it easier to solve each others'
problems and to collaborate or move from project to project, and from team to team.

There is room for personal preference in style guides, but style guides are first and foremost
about quality and standardization -- especially when collaborating on code. We believe that a commonly used Stata style guide
would improve the quality of all code written in Stata, which is why we have begun the one included here. You do not necessarily need to follow our
style guide precisely. We encourage you to write your own style guide if you disagree with us. The best style guide
woud be the one adopted the most widely. What is most important is that you adopt a style guide and follow it consistently across your projects.

\newpage

\section{The DIME Analytics Stata style guide}

While this section is called a \textit{Stata} Style Guide, many of these practices are agnostic to which
programming language you are using: best practices often relate to concepts that are common across many
languages. If you are coding in a different language, then you might still use many of the guidelines
listed in this section, but you should use your judgment when doing so.
All style rules introduced in this section are the way we suggest to code,
but the most important thing is that the way you style your code is \textit{consistent}.
This guide allows our team to have a consistent code style.

\subsection{Commenting code}

Comments do not change the output of code, but without them, your code will not be accessible to your colleagues.
It will also take you a much longer time to edit code you wrote in the past if you did not comment it well.
So, comment a lot: do not only write \textit{what} your code is doing but also \textit{why} you wrote it like that.

There are three types of comments in Stata and they have different purposes:
\begin{enumerate}
  \item \texttt{/* */} indicates narrative, multi-line comments at the beginning of files or sections.
  \item \texttt{*} indicates a change in task or a code sub-section and should be multi-line only if necessary.
  \item \texttt{//} is used for inline clarification a single line of code.
\end{enumerate}

\codeexample{stata-comments.do}{./code/stata-comments.do}

\subsection{Abbreviating commands}

Stata commands can often be abbreviated in the code. In the helpfiles you can tell if a command can be
abbreviated, indicated by the part of the name that is underlined in the syntax section at the top.
Only built-in commands can be abbreviated; user-written commands can not.
Although Stata allows some commands to be abbreviated to one or two characters,
this can be confusing -- two-letter abbreviations can rarely be ``pronounced''
in an obvious way that connects them to the functionality of the full command.
Therefore, command abbreviations in code should not be shorter than three characters,
with the exception of \texttt{tw} for \texttt{twoway} and \texttt{di} for \texttt{display},
and abbreviations should only be used when widely accepted abbreviation exists.
We do not abbreviate \texttt{local}, \texttt{global}, \texttt{save}, \texttt{merge}, \texttt{append}, or \texttt{sort}.
Here is our non-exhaustive list of widely accepted abbreviations of common Stata commands.

\begin{center}
	\begin{tabular}{ c | l }
    Abbreviation & Command \\
		\hline
		\texttt{tw} & \texttt{twoway} \\
		\texttt{di} & \texttt{display} \\
		\texttt{gen} & \texttt{generate} \\
		\texttt{mat} & \texttt{matrix} \\
		\texttt{reg} & \texttt{regress} \\
		\texttt{lab} & \texttt{label} \\
		\texttt{sum} & \texttt{summarize} \\
		\texttt{tab} & \texttt{tabulate} \\
		\texttt{bys} & \texttt{bysort} \\
		\texttt{qui} & \texttt{quietly} \\
		\texttt{cap} & \texttt{capture} \\
		\texttt{forv} & \texttt{forvalues} \\
		\texttt{prog} & \texttt{program} \\
		\hline
	\end{tabular}
\end{center}

\subsection{Writing loops}

In Stata examples and other code languages, it is common that the name of the local generated by \texttt{foreach} or \texttt{forvalues}
is named something as simple as \texttt{i} or \texttt{j}. In Stata, however,
loops generally index a real object, and looping commands should name that index descriptively.
One-letter indices are acceptable only for general examples;
for looping through \textbf{iterations} with \texttt{i};
and for looping across matrices with \texttt{i}, \texttt{j}.
Other typical index names are \texttt{obs} or \texttt{var} when looping over observations or variables, respectively.
But since Stata does not have arrays such abstract syntax should not be used in Stata code otherwise.
Instead, index names should describe what the code is looping over, for example household members, crops, or
medicines. This makes code much more readable, particularly in nested loops.

\codeexample{stata-loops.do}{./code/stata-loops.do}

\subsection{Using whitespace}

In Stata, one space or many spaces does not make a difference, and this can be used to make the code
much more readable. In the example below the exact same code is written twice, but in the good example whitespace
is used to signal to the reader that the central object of this segment of code is the variable
\texttt{employed}. Organizing the code like this makes the code much quicker to read, and small typos
stand out much more, making them easier to spot.

We are all very well trained in using whitespace in software like PowerPoint and Excel: we would never
present a PowerPoint presentation where the text does not align or submit an Excel table with unstructured
rows and columns, and the same principles apply to coding.

\codeexample{stata-whitespace-columns.do}{./code/stata-whitespace-columns.do}

Indentation is another type of whitespace that makes code more readable. Any segment of code that is
repeated in a loop or conditional on an if-statement should have indentation of 4 spaces relative
to both the loop or conditional statement as well as the closing curly brace.
Similarly, continuing lines of code should be indented under the initial command.
If a segment is in a loop inside a loop, then it should be indented another 4 spaces, making it 8 spaces more
indented than the main code. In some code editors this indentation can be achieved by using the tab button on
your keyboard. However, the type of tab used in the Stata do-file editor does not always display the same across
platforms, such as when publishing the code on GitHub. Therefore we recommend that indentation be 4 manual spaces
instead of a tab.

\codeexample{stata-whitespace-indentation.do}{./code/stata-whitespace-indentation.do}

\subsection{Writing conditional expressions}

All conditional (true/false) expressions should be within at least one set of parentheses.
The negation of logical expressions should use bang (\texttt{!}) and not tilde (\texttt{\~}).

\codeexample{stata-conditional-expressions1.do}{./code/stata-conditional-expressions1.do}

You should also always use \texttt{if-else} statements when applicable even if you can express the same
thing with two separate \texttt{if} statements. When using \texttt{if-else} statements you are
communicating to anyone reading your code that the two cases are mutually exclusive in an \texttt{if-else} statement
which makes your code more readable. It is also less error-prone and easier to update.

\codeexample{stata-conditional-expressions2.do}{./code/stata-conditional-expressions2.do}

\subsection{Using macros}

Stata has several types of \textbf{macros} where numbers or text can be stored temporarily, but the two most common
macros are \textbf{local} and \textbf{global}. Locals should always be the default type and globals should only
be used when the information stored is used in a different do-file. Globals are error-prone since they are
active as long as Stata is open, which creates a risk that a global from one project is incorrectly used in
another, so only use globals where they are necessary. Our recommendation is that globals should only be defined in the \textbf{master do-file}.
All globals should be referenced using both the the dollar sign and the curly brackets around their name;
otherwise, they can cause readability issues when the endpoint of the macro name is unclear.

There are several naming conventions you can use for macros with long or multi-word names.
Which one you use is not as important as whether you and your team are consistent in how you name then.
You can use all lower case (\texttt{mymacro}), underscores (\texttt{my\_macro}), or ``camel case'' (\texttt{myMacro}), as long as you are consistent.

\codeexample{stata-macros.do}{./code/stata-macros.do}

\subsection{Writing file paths}

All file paths should be absolute and dynamic, should always be enclosed in double quotes, and should
always use forward slashes for folder hierarchies (\texttt{/}), since Mac and Linux computers cannot read file paths with backslashes.
File paths should also always include the file extension (\texttt{.dta}, \texttt{.do}, \texttt{.csv}, etc.),
since to omit the extension causes ambiguity if another file with the same name is created (even if there is a default).

\textbf{Absolute file paths} means that all file paths must start at the root folder of the computer, for
example \texttt{C:/} on a PC or \texttt{/Users/} on a Mac. This makes sure that
you always get the correct file in the correct folder. \textbf{We never use \texttt{cd}.}
We have seen many cases when using \texttt{cd} where
a file has been overwritten in another project folder where \texttt{cd} was currently pointing to.
Relative file paths are common in many other programming languages, but there they are relative to the
location of the file running the code, and then there is no risk that a file is saved in a completely different folder.
Stata does not provide this functionality.

\textbf{Dynamic file paths} use globals that are set in a central master do-file to dynamically build your file
paths. This has the same function in practice as setting \texttt{cd}, as all new users should only have to change these
file path globals in one location. But dynamic absolute file paths are a better practice since if the
global names are set uniquely there is no risk that files are saved in the incorrect project folder, and
you can create multiple folder globals instead of just one location as with \texttt{cd}.

\codeexample{stata-filepaths.do}{./code/stata-filepaths.do}

\subsection{Placing line breaks}

Long lines of code are difficult to read if you have to scroll left and right to see the full line of code.
When your line of code is wider than text on a regular paper you should introduce a line break.
A common line breaking length is around 80 characters, and Stata and other code editors
provide a visible ``guide line'' to tell you when you should start a new line using \texttt{///}.
This breaks the line in the code editor while telling Stata that the same line of code continues on the next row in the
code editor. Using the \texttt{\#delimit} command is only intended for advanced programming
and is discouraged for analytical code in an article in Stata's official journal.\cite{cox2005styleguide}
These do not need to be horizontally aligned in code unless they have comments,
since indentations should reflect that the command continues to a new line.
Line breaks and indentations can similarly be used to highlight the placement
of the \textbf{option comma} in Stata commands.

\codeexample{stata-linebreak.do}{./code/stata-linebreak.do}

\subsection{Using boilerplate code}

Boilerplate code is a type of code that always comes at the top of the code file, and its purpose is to harmonize
the settings across users running the same code to the greatest degree possible. There is no way in Stata to
guarantee that any two installations of Stata will always run code in exactly the same way. In the vast
majority of cases it does, but not always, and boilerplate code can mitigate that risk (although not eliminate
it). We have developed a command that runs many commonly used boilerplate settings that are optimized given
your installation of Stata. It requires two lines of code to execute the \texttt{version}
setting that avoids difference in results due to different versions of Stata.

\codeexample{stata-boilerplate.do}{./code/stata-boilerplate.do}

\subsection{Saving data}

Similarly to boilerplate code, there are good practices that should be followed before saving the data set.
These are sorting and ordering the data set, dropping intermediate variables that are not needed, and
compressing the data set to save disk space and network bandwidth.

If there is an ID variable or a set of ID variables, then the code should also test that they are uniqueally and
fully identifying the data set.\sidenote{\url{https://dimewiki.worldbank.org/wiki/ID\_Variable\_Properties}} ID
variables are also perfect variables to sort on, and to order leftmost in the data set.

The command \texttt{compress} makes the data set smaller in terms of memory usage without ever losing any
information. It optimizes the storage types for all variables and therefore makes it smaller on your computer
and faster to send over a network or the internet.

\codeexample{stata-before-saving.do}{./code/stata-before-saving.do}


\mainmatter
