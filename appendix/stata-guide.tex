%------------------------------------------------

\begin{fullwidth}

Most academic programs that prepare students for a career
in the type of work discussed in this book
spend a disproportionately small amount of time teaching their students coding skills
in relation to the share of their professional time they will spend writing code
their first years after graduating.
Recent masters-level graduates that have joined our team
tended to have very good theoretical knowledge,
but have required a lot of training in practical skills.
To us, this is like an architecture graduate having learned
how to sketch, describe, and discuss
the concepts and requirements of a new building very well --
but without having the technical skills
to contribute to a blueprint following professional standards
that can be used and understood by other professionals.
The reasons for this are a topic for another book,
but in today's data-driven world,
people working in quantitative development research must be proficient collaborative programmers,
and that includes more than being able to compute the correct numbers.

This appendix begins with a short section containing instructions
on how to access and use the code examples shared in this book.
The second section contains the DIME Analytics Stata Style Guide.
We believe these resources can help anyone write more understandable code,
no matter how proficient they are in writing Stata code.
Widely accepted and used style guides are common in most programming languages,
and we think that using such a style guide greatly improves the quality
of research projects coded in Stata.
We hope that this guide can help increase the emphasis
given to using, improving, sharing, and standardizing code style among the Stata community.
Style guides are the most important tool in how you, like an architect,
can draw a blueprint that can be understood and used by everyone in your trade.

\end{fullwidth}

%------------------------------------------------

\section{Using the code examples in this book}

You can access the raw code used in examples in this book in several ways.
We use GitHub to version control everything in this book, the code included.
To see the code on GitHub, go to: \url{https://github.com/worldbank/d4di/tree/master/code}.
If you are familiar with GitHub you can fork the repository and clone your fork.
We only use Stata's built-in datasets in our code examples,
so you do not need to download any data.
If you have Stata installed on your computer, then you will already have the data files used in the code.

A less technical way to access the code is to click the individual file in the URL above, then click
the button that says \textbf{Raw}. You will then get to a page that looks like the one at:
\url{https://raw.githubusercontent.com/worldbank/d4di/master/code/code.do}.
There, you can copy the code from your browser window to your do-file editor with the formatting intact.
This method is only practical for a single file at the time.
If you want to download all code used in this book, you can do that at:
\url{https://github.com/worldbank/d4di/archive/master.zip}. That link offers a \texttt{.zip} file download
with all the content used in writing this book, including the \LaTeX{} code used for the book itself. After
extracting the .zip-file you will find all the code in a folder called \texttt{/code/}.

\subsection{Understanding Stata code}

Whether you are new to Stata or have used it for decades,
you will always run into commands that
you have not seen before or whose function you do not remember.
(Whether you are new or not, you should frequently revisit the most common commands --
often you will learn they can do something you never realized.\sidenote{
  \url{https://www.stata.com/manuals13/u27.pdf}})
Every time that happens,
you should always look up the help file for that command.
We often encounter the conception that help files are only for beginners.
We could not disagree with that conception more,
as the only way to get better at Stata is to constantly read help files.
So if there is a command that you do not understand in any of our code examples,
for example \texttt{isid}, then write \texttt{help isid},
and the help file for the command \texttt{isid} will open.
We cannot emphasize enough how important it is
that you get into the habit of reading help files.
Most of us have a help file window open at all times.

Sometimes, you will encounter code that employs user-written commands,
and you will not be able to read their help files until you have installed the commands.
Two examples of these in our code are \texttt{randtreat} or \texttt{ieboilstart}.
The most common place to distribute user-written commands for Stata
is the Boston College Statistical Software Components (SSC) archive.\sidenote{
\url{https://ideas.repec.org/s/boc/bocode.html}}
In our code examples, we only use either Stata's built-in commands or commands available from the
SSC archive.
So, if your installation of Stata does not recognize a command in our code, for example
\texttt{randtreat}, then type \texttt{ssc install randtreat} in Stata.

Some commands on SSC are distributed in packages.
This is the case, for example, of \texttt{ieboilstart}.
That means that you will not be able to install it using \texttt{ssc install ieboilstart}.
If you do, Stata will suggest that you instead use \texttt{findit ieboilstart},
which will search SSC (among other places) and see if there is a
package that contains a command called \texttt{ieboilstart}.
Stata will find \texttt{ieboilstart} in the package \texttt{ietoolkit},
so to use this command you will type \texttt{ssc install ietoolkit} in Stata instead.

We understand that it can be confusing to work with packages for first time,
but this is the best way to set up your Stata installation to benefit from other
people's work that has been made publicly available.
Once you get used to installing commands like this it will not be confusing at all.
All code with user-written commands, furthermore, is best written when it installs such commands
at the beginning of the master do-file,
so that the user does not have to search for packages manually.

\subsection{Why we use a Stata style guide}

Programming languages used in computer science always have style guides associated with them.
Sometimes they are official guides that are universally agreed upon, such as PEP8 for
Python.\sidenote{\url{https://www.python.org/dev/peps/pep-0008}} More commonly, there are well-recognized but
non-official style guides like the JavaScript Standard Style\sidenote{\url{https://standardjs.com/\#the-rules}} for
JavaScript or Hadley Wickham's style guide for R.\sidenote{\url{https://style.tidyverse.org/syntax.html}}
Google, for example, maintains style guides for all languages
that are used in its projects.\sidenote{
  \url{https://github.com/google/styleguide}}

Aesthetics is an important part of style guides, but not the main point.
Neither is telling you which commands to use:
there are plenty of guides to Stata's extensive functionality.\sidenote{
  \url{https://scholar.harvard.edu/files/mcgovern/files/practical_introduction_to_stata.pdf}}
The important function is to allow programmers who are likely to work together
to share conventions and understandings of what the code is doing.
Style guides therefore help improve the quality of the code
in that language that is produced by all programmers in a community.
It is through a shared style that newer programmers can learn from more experienced programmers
how certain coding practices are more or less error-prone.
Broadly-accepted style conventions make it easier to borrow solutions
from each other and from examples online
without causing bugs that might only be found too late.
Similarly, globally standardized style guides make it easier to solve each others'
problems and to collaborate or move from project to project, and from team to team.

There is room for personal preference in style guides,
but style guides are first and foremost about quality and standardization --
especially when collaborating on code.
We believe that a commonly used Stata style guide would improve the quality of all code written in Stata,
which is why we have begun the one included here.
You do not necessarily need to follow our style guide precisely.
We encourage you to write your own style guide if you disagree with us.
The best style guide would be the one adopted the most widely.
What is important is that you adopt a style guide and follow it consistently across your projects.

\newpage

\section{The DIME Analytics Stata Style Guide}

While this section is called a \textit{Stata} style guide,
many of these practices are agnostic to which programming language you are using:
best practices often relate to concepts that are common across many languages.
If you are coding in a different language,
then you might still use many of the guidelines listed in this section,
but you should use your judgment when doing so.
All style rules introduced in this section are the way we suggest to code,
but the most important thing is that the way you style your code is \textit{consistent}.
This guide allows our team to have a consistent code style.

\subsection{Commenting code}

Comments do not change the output of code, but without them,
your code will not be accessible to your colleagues.
It will also take you a much longer time to edit code you wrote in the past if you did not comment it well.
So, comment a lot: do not only write \textit{what} your code is doing
but also \textit{why} you wrote it like the way you did.
In general, try to write simpler code that needs less explanation,
even if you could use an elegant and complex method in less space,
unless the advanced method is a widely accepted one.

There are three types of comments in Stata and they have different purposes:

\codeexample{stata-comments.do}{./code/stata-comments.do}

\subsection{Abbreviating commands}

Stata commands can often be abbreviated in the code.
You can tell if a command can be abbreviated if the help file indicates an abbreviation by underlining part of the name in the syntax section at the top.
Only built-in commands can be abbreviated; user-written commands cannot.
(Many commands additionally allow abbreviations of options:
these are always acceptable at the shortest allowed abbreviation.)
Although Stata allows some commands to be abbreviated to one or two characters,
this can be confusing -- two-letter abbreviations can rarely be ``pronounced''
in an obvious way that connects them to the functionality of the full command.
Therefore, command abbreviations in code should not be shorter than three characters,
with the exception of \texttt{tw} for \texttt{twoway} and \texttt{di} for \texttt{display},
and abbreviations should only be used when widely a accepted abbreviation exists.
We do not abbreviate \texttt{local}, \texttt{global}, \texttt{save}, \texttt{merge}, \texttt{append}, or \texttt{sort}.
The following is a list of accepted abbreviations of common Stata commands:

\begin{center}
	\begin{tabular}{ c | l }
    Abbreviation & Command \\
		\hline
		\texttt{tw} & \texttt{twoway} \\
		\texttt{di} & \texttt{display} \\
		\texttt{gen} & \texttt{generate} \\
		\texttt{mat} & \texttt{matrix} \\
		\texttt{reg} & \texttt{regress} \\
		\texttt{lab} & \texttt{label} \\
		\texttt{sum} & \texttt{summarize} \\
		\texttt{tab} & \texttt{tabulate} \\
		\texttt{bys} & \texttt{bysort} \\
		\texttt{qui} & \texttt{quietly} \\
		\texttt{noi} & \texttt{noisily} \\
		\texttt{cap} & \texttt{capture} \\
		\texttt{forv} & \texttt{forvalues} \\
		\texttt{prog} & \texttt{program} \\
		\texttt{hist} & \texttt{histogram} \\
		\hline
	\end{tabular}
\end{center}

\subsection{Abbreviating variables}

Never abbreviate variable names. Instead, write them out completely.
Your code may change if a variable is later introduced
that has a name exactly as in the abbreviation.
\texttt{ieboilstart} executes the command \texttt{set varabbrev off} by default,
and will therefore break any code using variable abbreviations.

Using wildcards and lists in Stata for variable lists
(\texttt{*}, \texttt{?}, and \texttt{-}) is also discouraged,
because the functionality of the code may change
if the dataset is changed or even simply reordered.
If you intend explicitly to capture all variables of a certain type,
prefer \texttt{unab} or \texttt{lookfor} to build that list in a local macro,
which can then be checked to have the right variables in the right order.

\subsection{Writing loops}

In Stata examples and other code languages, it is common for the name of the local generated by \texttt{foreach} or \texttt{forvalues}
to be something as simple as \texttt{i} or \texttt{j}. In Stata, however,
loops generally index a real object, and looping commands should name that index descriptively.
One-letter indices are acceptable only for general examples;
for looping through \textbf{iterations} with \texttt{i};
and for looping across matrices with \texttt{i}, \texttt{j}.
Other typical index names are \texttt{obs} or \texttt{var} when looping over observations or variables, respectively.
But since Stata does not have arrays,
such abstract syntax should not be used in Stata code otherwise.
Instead, index names should describe what the code is looping over --
for example household members, crops, or medicines.
Even counters should be explicitly named.
This makes code much more readable, particularly in nested loops.

\codeexample{stata-loops.do}{./code/stata-loops.do}

\subsection{Using whitespace}

In Stata, adding one or many spaces does not make a difference to code execution,
and this can be used to make the code much more readable.
We are all very well trained in using whitespace in software like PowerPoint and Excel:
we would never present a PowerPoint presentation where the text does not align
or submit an Excel table with unstructured rows and columns.
The same principles apply to coding.
In the example below the exact same code is written twice,
but in the better example whitespace is used to signal to the reader
that the central object of this segment of code is the variable \texttt{employed}.
Organizing the code like this makes it much quicker to read,
and small typos stand out more, making them easier to spot.

\codeexample{stata-whitespace-columns.do}{./code/stata-whitespace-columns.do}

\noindent Indentation is another type of whitespace that makes code more readable.
Any segment of code that is repeated in a loop or conditional on an
\texttt{if}-statement should have indentation of 4 spaces relative
to both the loop or conditional statement as well as the closing curly brace.
Similarly, continuing lines of code should be indented under the initial command.
If a segment is in a loop inside a loop, then it should be indented another 4 spaces,
making it 8 spaces more indented than the main code.
In some code editors this indentation can be achieved by using the tab button.
However, the type of tab used in the Stata do-file editor does not always display the same across platforms,
such as when publishing the code on GitHub.
Therefore we recommend that indentation be 4 manual spaces instead of a tab.

\codeexample{stata-whitespace-indentation.do}{./code/stata-whitespace-indentation.do}

\subsection{Writing conditional expressions}

All conditional (true/false) expressions should be within at least one set of parentheses.
The negation of logical expressions should use bang (\texttt{!}) and not tilde (\texttt{\~}).
Always use explicit truth checks (\texttt{if \`{}value\textquotesingle==1})
rather than implicits (\texttt{if \`{}value\textquotesingle}).
Always use the \texttt{missing(\`{}var\textquotesingle)} function
instead of arguments like (\texttt{if \`{}var\textquotesingle<=.}).
Always consider whether missing values will affect the evaluation of conditional expressions.

\codeexample{stata-conditional-expressions1.do}{./code/stata-conditional-expressions1.do}

\noindent Use \texttt{if-else} statements when applicable
even if you can express the same thing with two separate \texttt{if} statements.
When using \texttt{if-else} statements you are communicating to anyone reading your code
that the two cases are mutually exclusive, which makes your code more readable.
It is also less error-prone and easier to update if you want to change the condition.

\codeexample{stata-conditional-expressions2.do}{./code/stata-conditional-expressions2.do}

\subsection{Using macros}

Stata has several types of \textbf{macros} where numbers or text can be stored temporarily,
but the two most common macros are \textbf{local} and \textbf{global}.
All macros should be defined using the \texttt{=} operator.
Never abbreviate the commands for \textbf{local} and \textbf{global}.
Locals should always be the default type and globals should only
be used when the information stored is used in a different do-file.
Globals are error-prone since they are active as long as Stata is open,
which creates a risk that a global from one project is incorrectly used in another,
so only use globals where they are necessary.
Our recommendation is that globals should only be defined in the \textbf{master do-file}.
All globals should be referenced using both the the dollar sign and curly brackets around their name (\texttt{\$\{\}});
otherwise, they can cause readability issues when the endpoint of the macro name is unclear.

You should use descriptive names for all macros (up to 32 characters; preferably fewer).
There are several naming conventions you can use for macros with long or multi-word names.
Which one you use is not as important as whether you and your team are consistent in how you name then.
You can use all lower case (\texttt{mymacro}), underscores (\texttt{my\_macro}),
or ``camel case'' (\texttt{myMacro}), as long as you are consistent.
Simple prefixes are useful and encouraged such as \texttt{this\_estimate} or \texttt{current\_var},
or, using \texttt{camelCase}, \texttt{lastValue}, \texttt{allValues}, or \texttt{nValues}.
Nested locals (\texttt{\`{}\`{}value\textquotesingle\textquotesingle})
are also possible for a variety of reasons when looping, and should be indicated in comments.
If you need a macro to hold a literal macro name,
it can be done using the backslash escape character;
this causes the stored macro to be evaluated
at the usage of the macro rather than at its creation.
This functionality should be used sparingly and commented extensively.

\codeexample{stata-macros.do}{./code/stata-macros.do}

\subsection{Writing file paths}

All file paths should always be enclosed in double quotes,
and should always use forward slashes for folder hierarchies (\texttt{/}).
File names should be written in lower case with dashes (\texttt{my-file.dta}).
Mac and Linux computers cannot read file paths with backslashes,
and backslashes cannot easily be removed with find-and-replace
because the character has other functional uses in code.
File paths should always include the file extension
(\texttt{.dta}, \texttt{.do}, \texttt{.csv}, etc.).
Omitting the extension causes ambiguity
if another file with the same name is created
(even if there is a default file type).

File paths should also be absolute and dynamic.
\textbf{Absolute} means that all
file paths start at the root folder of the computer,
often \texttt{C:/} on a PC or \texttt{/Users/} on a Mac.
This ensures that you always get the correct file in the correct folder.
Do not use \texttt{cd} unless there is a function that requires it.
When using \texttt{cd}, it is easy to overwrite a file in another project folder.
Many Stata functions use \texttt{cd} and therefore the current directory may change without warning.
Relative file paths are common in many other programming languages,
but there they are always relative to the location of the file running the code.
Stata does not provide this functionality.

\textbf{Dynamic} file paths use global macros for the location of the root folder.
These globals should be set in a central master do-file.
This makes it possible to write file paths that work very similarly to relative paths.
It also achieves the functionality that setting \texttt{cd} is often intended to:
executing the code on a new system only requires updating file path globals in one location.
If global names are unique, there is no risk that files are saved in the incorrect project folder.
You can create multiple folder globals as needed and this is encouraged.

\codeexample{stata-filepaths.do}{./code/stata-filepaths.do}

\subsection{Line breaks}

Long lines of code are difficult to read if you have to scroll left and right to see the full line of code.
When your line of code is wider than text on a regular paper, you should introduce a line break.
A common line breaking length is around 80 characters.
Stata's do-file editor and other code editors provide a visible guide line.
Around that length, start a new line using \texttt{///}.
Using \texttt{///} breaks the line in the code editor,
while telling Stata that the same line of code continues on the next line.
The \texttt{///} breaks do not need to be horizontally aligned in code,
although you may prefer to if they have comments that read better aligned,
since indentations should reflect that the command continues to a new line.
Break lines where it makes functional sense.
You can write comments after \texttt{///} just as with \texttt{//}, and that is usually a good thing.
The \texttt{\#delimit} command should only be used for advanced function programming
and is officially discouraged in analytical code.\cite{cox2005styleguide}
Never, for any reason, use \texttt{/* */} to wrap a line:
it is distracting and difficult to follow compared to the use
of those characters to write regular comments.
Line breaks and indentations may be used to highlight the placement
of the \textbf{option comma} or other functional syntax in Stata commands.

\codeexample{stata-linebreak.do}{./code/stata-linebreak.do}

\subsection{Using boilerplate code}

\textbf{Boilerplate} code is a few lines of code that always come at the top of the code file,
and its purpose is to harmonize settings across users
running the same code to the greatest degree possible.
There is no way in Stata to guarantee that any two installations
will always run code in exactly the same way.
In the vast majority of cases it does, but not always,
and boilerplate code can mitigate that risk.
We have developed the \texttt{ieboilstart} command
to implement many commonly-used boilerplate settings
that are optimized given your installation of Stata.
It requires two lines of code to execute the \texttt{version}
setting, which avoids differences in results due to different versions of Stata.
Among other things, it turns the \texttt{more} flag off
so code never hangs while waiting to display more output;
it turns \texttt{varabbrev} off so abbrevated variable names are rejected;
and it maximizes the allowed memory usage and matrix size
so that code is not rejected on other machines for violating system limits.
(For example, Stata/SE and Stata/IC, allow for different maximum numbers of variables,
and the same happens with Stata 14 and Stata 15,
so it may not be able to run code written in one of these version using another.)
Finally, it clears all stored information in Stata memory,
such as non-installed programs and globals,
getting as close as possible to opening Stata fresh.

\codeexample{stata-boilerplate.do}{./code/stata-boilerplate.do}

\subsection{Saving data}

There are good practices that should be followed before saving any data set.
These are to \texttt{sort} and \texttt{order} the data set,
dropping intermediate variables that are not needed,
and compressing the data set to save disk space and network bandwidth.

If there is a unique ID variable or a set of ID variables,
the code should test that they are uniqueally and
fully identifying the data set.\sidenote{
  \url{https://dimewiki.worldbank.org/ID_Variable_Properties}}
ID variables are also perfect variables to sort on,
and to \texttt{order} first in the data set.

The command \texttt{compress} makes the data set smaller in terms of memory usage
without ever losing any information.
It optimizes the storage types for all variables
and therefore makes it smaller on your computer
and faster to send over a network or the internet.

\codeexample{stata-before-saving.do}{./code/stata-before-saving.do}

\subsection{Miscellaneous notes}

Write multiple graphs as \texttt{tw (xx)(xx)(xx)}, not \texttt{tw xx||xx||xx}.

\bigskip\noindent In simple expressions, put spaces around each binary operator except \texttt{\^}.
Therefore write \texttt{gen z = x + y} and \texttt{x\^}\texttt{2}.

\bigskip\noindent When order of operations applies, you may adjust spacing and parentheses: write
\texttt{hours + (minutes/60) + (seconds/3600)}, not \texttt{hours + minutes / 60 + seconds / 3600}.
For long expressions, \texttt{+} and \texttt{-} operators should start the new line,
but \texttt{*} and \texttt{/} should be used inline. For example:

\texttt{gen newvar =   x ///}

\texttt{             - (y/2) ///}

\texttt{             + a * (b - c)}

\bigskip\noindent  Make sure your code doesn't print very much to the results window as this is slow.
This can be accomplished by using \texttt{run file.do} rather than \texttt{do file.do}.
Interactive commands like \texttt{sum} or \texttt{tab} should be used sparingly in dofiles,
unless they are for the purpose of getting \texttt{r()}-statistics.
In that case, consider using the \texttt{qui} prefix to prevent printing output.
It is also faster to get outputs from commands like \texttt{reg} using the \texttt{qui} prefix.

\mainmatter
