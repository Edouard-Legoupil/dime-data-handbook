%------------------------------------------------

\begin{fullwidth}
	
Most academic programs that prepare student for a career in the type of work discussed in this book 
spend a disproportionate time teaching their students coding skills in relation to the share of 
their professional time they will spend writing code their first years after graduating. Recent 
Masters' program graduates that join our team tend to have very good knowledge in the theory of our
trade, but tend to require a lot of training in its practical skills. To us it is like an architect 
that can draw, describe and discuss a concept or requirement of a new building very well, but do 
not have the skill set to contribute to a blue print using professional standards and can be used 
and understood by other professionals during construction. The reasons for this is probably a topic
for another book, but in today's data driven world, people working in quantitative economics research 
must be proficient programmers, and that includes more than being able to compute the correct number.

This appendix first has a short section with instructions on how to access and use the code shared in 
this book. The second section contains a style guide to Stata. Widely accepted and used style guides 
are common in most programming languages, and we think that it the end it introduces a quality risk 
for research projects coded in Stata that there is so little emphasize in the Stata community on using, 
improving, sharing and standardizing style guides. Style guides are the most important tool in how 
you like an architect draw a blue print that can be understood and used by everyone in your trade, 
and not just by you.


\end{fullwidth}

%------------------------------------------------

\section{How to use the Stata code examples in this book}

You can access the code in the code examples used in this book in many ways. We only use Stata's built
in data sets in our code examples, so you do not need to download any data from anywhere. If you have 
Stata installed on your computer, then you will have the data files used in the code. 

We use GitHub to version control everything in this book, the code included. To see the code on GitHub go to
\url{https://github.com/worldbank/d4di/tree/master/code}. If you are familiar with GitHub you can
fork the repository and clone your fork. 

A less technical way to access the code is to click the individual file in the URL below, then click
the button that says \textbf{RAW}. You will then get to a page that looks like this
\url{https://raw.githubusercontent.com/worldbank/d4di/master/code/code.do} where you can copy the code
from your browser window to your do-file editor with the formating intact and without other things from
the website. If you want another code file you can modify the URL. This method is only practical for a 
single file at the time. If you want to download all code used in this book then you can do that at
\url{https://github.com/worldbank/d4di/archive/master.zip}. That link offers you to download a .zip-file
with all the content used in writing this book, including the \LaTeX{} code used for the book itself. After 
extracting the .zip-file you will find all the code in a folder called \textit{code}. 

\subsection{Understanding Stata code}

Regardless if you are new to Stata or have used it for decades, you will always run in to commands that 
you have not seen before or do not remember what they do. Every time that happens you should always look 
that command up in the help file. For some reason we often encounter the conception that the help files 
are only for beginner. We could not disagree with that conception as the only way to get better at Stata 
is to constantly read help files. So if there is a command that you do not understand in any of our code 
examples, for example \textit{isid}, then write \verb+help isid+ and the help file for the 
command \textit{isid} is opened.

We cannot emphasize too much how important we think it is that you get into the habit of reading help files.

Sometimes you will see user written commands and you will not be able to read the help file until you have
installed them. One example of that in our code if \textit{reandtreat} or \textit{ieboilstart}. The by far
most common place to distribute user written commands are Boston College Statistical Software Components
(SSC) archive. In our code examples we only use either Stata's built in commands or commands from the 
SSC archive. So if your installation of Stata does not recognize a command in our code, for example 
\textit{randtreat}, then type \verb+ssc install randtreat+ in Stata. 

Some commands on SSC are distributed in packages, for example \textit{ieboilstart}, meaning that you will 
not be able to install it using \verb+ssc install ieboilstart+. If you do Stata will suggest that you 
instead use \verb+findit ieboilstart+ which will search SSC (among other places) and see if there is a 
package that has a command called \textit{ieboilstart}. Stata will find \textit{ieboilstart} in the package 
\textit{ietoolkit}, so then you will type \verb+ssc install ietoolkit+ instead in Stata.

We understand that this seems confusing the first time you work with this, but this is the best way to set 
up your Stata installation to benefit on other people' work that they have made publicly available, and 
once used to installing commands like this it will not be confusing at all.


\section{Stata Style Guide}

Programming languages used in computer science always have style guides associated with them. Sometimes 
they are official guides that are universally agreed upon as PEP8 for 
Python\sidenote{https://www.python.org/dev/peps/pep-0008/}. But more common are well-recognized but 
unofficial style guides like JavaScript Standard Style\sidenote{https://standardjs.com/#the-rules} for 
JavaScript or Hadley Wickham's\sidenote{http://adv-r.had.co.nz/Style.html} style guide for R.

Aesthetics is an important part of style guides but \textbf{if you are thinking that the only purpose of a 
style guide is aesthetics, then you are completely missing the main point}. The existence of style guides 
lifts the quality of the code produced by all programmers in the community of the language of that style 
guide. It is through a style guide that unexperienced programmers learn from more experienced programmers 
how certain coding practices are more or less error prone. Broadly accepted style guides makes it easier to 
borrow solutions from each other and from examples online without causing bugs that might only be found too 
late. Furthermore, if we in the Stata community would code better it would be easier to solve each others 
problem and move from project to project, or employer to employer.

There is room for personal preference in style guides, but \textbf{style guides are first and foremost 
about quality} - especially when collaborating on code - and if style guides were commonly used in the 
Stata community, then all our coding products would be better. You do not necessarily need to follow our 
style guide. We encourage you to write your own style guide if you disagree with us. The best style guide 
will be used the most and eventually the Stata community will have the best style guide possible which will 
benefit us all. What is important though, is that you adopt a style guide and follow it consistently in 
your project.



\mainmatter
