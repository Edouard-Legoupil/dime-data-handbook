%------------------------------------------------

\begin{fullwidth}
	
Most academic programs that prepare student for a career in the type of work discussed in this book 
spend a disproportionate time teaching their students coding skills in relation to the share of 
their professional time they will spend writing code their first years after graduating. Recent 
Masters' program graduates that join our team tend to have very good knowledge in the theory of our
trade, but tend to require a lot of training in its practical skills. To us it is like an architect 
that can draw, describe and discuss a concept or requirement of a new building very well, but do 
not have the skill set to contribute to a blue print using professional standards and can be used 
and understood by other professionals during construction. The reasons for this is probably a topic
for another book, but in today's data driven world, people working in quantitative economics research 
must be proficient programmers, and that includes more than being able to compute the correct number.

This appendix first has a short section with instructions on how to access and use the code shared in 
this book. The second section contains a style guide to Stata. Widely accepted and used style guides 
are common in most programming languages, and we think that it the end it introduces a quality risk 
for research projects coded in Stata that there is so little emphasize in the Stata community on using, 
improving, sharing and standardizing style guides. Style guides are the most important tool in how 
you like an architect draw a blue print that can be understood and used by everyone in your trade, 
and not just by you.


\end{fullwidth}

%------------------------------------------------

\section{How to use the Stata code examples in this book}


\section{Stata Style Guide}

Style guide

\mainmatter
