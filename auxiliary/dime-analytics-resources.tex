
%Make each resoruce a paragraph
\newcommand{\resourcepar}{\vspace{.75\baselineskip}\noindent}

%------------------------------------------------

\begin{fullwidth}

The resources listed in this appendix
are mentioned elsewhere in the chapters of this book,
and this appendix includes them all
in one place for easy reference.
All these resources are made public
under generous open-source licenses.
This means that you are free to use, reuse, and adapt these resources
for any purpose as you see fit,
so long as you include an appropriate citation.

\end{fullwidth}

%------------------------------------------------

\section{Public Resources and Tools}

\textbf{DIME Wiki.\sidenote{
		\url{https://dimewiki.worldbank.org}}}
One-stop shop for impact evaluation research solutions.
The DIME Wiki is a resource focused on
practical implementation guidelines rather than theory,
open to the public, easily searchable,
suitable for users of varying levels of expertise,
up-to-date with the latest technological advances
in electronic data collection,
and curated by a vibrant network of editors
with expertise in the field.

\resourcepar\textbf{Stata Visual Library.\sidenote{
		\url{https://worldbank.github.io/Stata-IE-Visual-Library/}}}
A curated, easy-to-browse selection of graphs created in Stata.
Clicking on each graph reveals the source code,
to allow for easy replication.

\resourcepar\textbf{R Econ Visual Library.\sidenote{
		\url{https://worldbank.github.io/r-econ-visual-library/}}}
A curated, easy-to-browse selection of graphs created in R.
Clicking on each graph reveals the source code,
to allow for easy replication.

\resourcepar\textbf{DIME Analytics Research Standards.\sidenote{
		\url{https://github.com/worldbank/dime-standards}}}
A repository outlining DIME's public commitments to
research ethics, transparency, reproducibility,
data security and data publication,
along with supporting tools and resources.

%%%%%%%%%%%%%%%%%%%%%%%%%%%%%%%%%%%%%

\section{Flagship Courses}

\textbf{Manage Successful Impact Evaluations (MSIE).\sidenote{
		\url{https://osf.io/h4d8y/}}}
DIME Analytics' flagship training is a week-long annual course,
held in person in Washington, D.C.
MSIE is intended to improve the skills and knowledge of
impact evaluation (IE) practitioners,
familiarizing them with critical issues in
IE implementation, recurring challenges,
and cutting-edge technologies.
The course consists of lectures and hands-on sessions.
Through small group discussions and interactive computer lab sessions,
participants work together to apply what they've learned
and have a first-hand opportunity to develop skills.
Hands-on sessions are offered in parallel tracks,
with different options based on software preferences and skill level.

\resourcepar\textbf{Manage Successful Impact Evaluation Surveys (MSIES).\sidenote{
		\url{https://osf.io/resya/}}}
A fully virtual course,
in which participants learn the workflow for primary data collection.
The course covers best practices at all stages of the survey workflow,
from planning to piloting instruments
and monitoring data quality once fieldwork begins.
There is a strong focus throughout on research ethics and reproducible workflows.
The course uses a combination of virtual lectures,
case studies, readings, and hands-on exercises.

\resourcepar\textbf{Research Assistant Onboarding Course.\sidenote{
		\url{https://osf.io/fqhdt/}}}
This course is designed to familiarize Research Assistants and Research Analysts
with DIME's standards for data work.
By the end of the course's six sessions,
participants will have the tools and knowledge to
implement best practices for transparent and reproducible research.
The course will focus on how to set up a collaborative workflow for
code, datasets, and research outputs.
Most content is platform and software agnostic,
but participants are expected to be familiar with statistical software.

\resourcepar\textbf{Introduction to R for advanced Stata users\sidenote{
		\url{https://osf.io/wzjtk}}}
This course is an introduction to the R programming language,
building upon knowledge of Stata.
The course focuses on common tasks in
development research, related to descriptive analysis,
data visualization, data processing, and geospatial data work.

\resourcepar\textbf{DIME Analytics Trainings.\sidenote{
		\url{https://osf.io/wzjtk}}}
DIME Analytics' home on the Open Science Framework,
with links to materials for all past courses and technical trainings.

%%%%%%%%%%%%%%%%%%%%%%%%%%%%%%%%%%%%%

\section{Software tools and trainings}

\textbf{\texttt{ietoolkit}.\sidenote{
		\url{https://github.com/worldbank/ietoolkit}}}
Suite of Stata commands to routinize common tasks for
data management and impact evaluation analysis.

\resourcepar\textbf{\texttt{iefieldkit}.\sidenote{
		\url{https://github.com/worldbank/iefieldkit}}}
Suite of Stata commands to routinize 
and document common tasks in primary data collection.

\resourcepar\textbf{DIME Analytics GitHub Trainings and Resources.\sidenote{
		\url{https://github.com/worldbank/dime-github-trainings}}}
A GitHub repository containing all
the GitHub training materials and resources developed by DIME Analytics.
The trainings follow DIME's model for organizing research teams on GitHub,
and are designed for face-to-face delivery,
but materials are shared so that they may be used and adapted by others.

\resourcepar\textbf{DIME Analytics \LaTeX-training.\sidenote{
		\url{https://github.com/worldbank/DIME-LaTeX-Templates}}}
A user-friendly guide to getting started with LaTeX.
Exercises provide opportunities to
practice creating appendices,
exporting tables from R or Stata to LaTeX,
and formatting tables in LaTeX.

%%%%%%%%%%%%%%%%%%%%%%%%%%%%%%%%%%%%%

\mainmatter
