%----------------------------------------------------------------------------------------
%	PACKAGES AND OTHER DOCUMENT CONFIGURATIONS
%----------------------------------------------------------------------------------------

\documentclass{tufte-book} % Use the tufte-book class which in turn uses the tufte-common class
%Tufte docs: http://ftp.math.purdue.edu/mirrors/ctan.org/macros/latex/contrib/tufte-latex/sample-book.pdf
\usepackage{geometry}

  \geometry{
    % showframe,
   % paperwidth=8.5in,
    %paperheight=11in,
   % left=0.55in,
   % right=0.45in,
    %top=.5in,
    %bottom=.5in,
    %marginparsep=0.25in,
    %marginparwidth=1in,
   % includemp,
    %includehead,
    % The text width and height are calculated automatically.
  }

\hypersetup{colorlinks,linkcolor=violet} % Comment this line if you don't wish to have colored links
\expandafter\def\expandafter\UrlBreaks\expandafter{\UrlBreaks  \do\-} %  Allow URLs to wrap on dash

\setlength\marginparpush{12pt} %https://tex.stackexchange.com/questions/89098/vertical-spacing-between-sidenotes-and-between-sidenotes-and-captions-in-tuft


\usepackage{microtype} % Improves character and word spacing

\usepackage{lipsum} % Inserts dummy text

\usepackage{booktabs} % Better horizontal rules in tables

\usepackage{graphicx} % Needed to insert images into the document
\graphicspath{{graphics/}} % Sets the default location of pictures
\setkeys{Gin}{width=\linewidth,totalheight=\textheight,keepaspectratio} % Improves figure scaling

\usepackage{upquote}
\usepackage{fancyvrb} % Allows customization of verbatim environments
%Fancyvrb docs: http://mirrors.ibiblio.org/CTAN/macros/latex/contrib/fancyvrb/doc/fancyvrb-doc.pdf
\fvset{fontsize=\small} % The font size of all verbatim text can be changed here

\renewcommand{\FancyVerbFormatLine}{\color{violet}}

\newcommand{\hangp}[1]{\makebox[0pt][r]{(}#1\makebox[0pt][l]{)}} % New command to create parentheses around text in tables which take up no horizontal space - this improves column spacing
\newcommand{\hangstar}{\makebox[0pt][l]{*}} % New command to create asterisks in tables which take up no horizontal space - this improves column spacing

\usepackage{xspace} % Used for printing a trailing space better than using a tilde (~) using the \xspace command

\newcommand{\monthyear}{\ifcase\month\or January\or February\or March\or April\or May\or June\or July\or August\or September\or October\or November\or December\fi\space\number\year} % A command to print the current month and year

\newcommand{\openepigraph}[2]{ % This block sets up a command for printing an epigraph with 2 arguments - the quote and the author
\begin{fullwidth}
\sffamily\large
\begin{doublespace}
\noindent\allcaps{#1}\\ % The quote
\noindent\allcaps{#2} % The author
\end{doublespace}
\end{fullwidth}
}

\newcommand{\blankpage}{\newpage\hbox{}\thispagestyle{empty}\newpage} % Command to insert a blank page

\usepackage{imakeidx} % Used to generate the index
\makeindex % Generate the index which is printed at the end of the document

%So we can use option FloatBarrier, which is similar to [H] but is an
%alternative solition when the algorithm can't solce [H] as too many
%settings are going on. [H] seems to get stuck in infinite loop
%https://tex.stackexchange.com/questions/2275/keeping-tables-figures-close-to-where-they-are-mentioned
\usepackage{placeins}
\newcommand{\codeexample}[2]{
	\begin{figure*}[h]
		\VerbatimInput[
			framesep=3mm,
			frame=lines, % line above and below code section
			numbers=left, %Line number
			label= #1, %name of code section
			baselinestretch=0.75, %Use line space more similat to line space in code editors
		]{#2} %Write the relative file path and the name of the file to be included
	\end{figure*}
	\FloatBarrier
}

%%%%%%%%%%%%%%%%%%%%%%%%%%%%%
% Customizing section/subsection titles
% https://tex.stackexchange.com/questions/96090/formatting-subsections-and-chapters-in-tufte-book

% section format
\titleformat{\section}%
{\normalfont\Large\bfseries}% format applied to label+text
{}% label
{}% horizontal separation between label and title body
{}% before the title body
[]% after the title body

% subsection format
\titleformat{\subsection}%
{\normalfont\itshape\large}% format applied to label+text
{}% label
{}% horizontal separation between label and title body
{}% before the title body
[]% after the title body

%----------------------------------------------------------------------------------------
%	BOOK META-INFORMATION
%----------------------------------------------------------------------------------------

\title{Development \\ \noindent Research \\ \noindent in Practice: \\ \bigskip
\noindent The DIME Analytics \\ \noindent Data Handbook} % Title of the book

\author{Kristoffer Bj{\"a}rkefur \\ \noindent Lu{\'i}za Cardoso de Andrade \\ \noindent Benjamin Daniels \\ \noindent Maria Jones \\} % Author

\publisher{DIME Analytics} % Publisher

%----------------------------------------------------------------------------------------

%--------------------------------------
%	Add URL to commit on copyright page
%--------------------------------------

\usepackage{xstring}
\usepackage{catchfile}

%Set this user input
\newcommand{\gitfolder}{.git} %relative path to .git folder from .tex doc
\newcommand{\reponame}{worldbank/dime-data-handbook} % Name of account and repo be set in URL

%Based on this https://tex.stackexchange.com/questions/455396/how-to-include-the-current-git-commit-id-and-branch-in-my-document
\CatchFileDef{\headfull}{\gitfolder/HEAD}{} 				%Get path to head file for checked out branch
\StrGobbleRight{\headfull}{1}[\head]						%Remove end of line character
\StrBehind[2]{\head}{/}[\branch]							%Parse out the path only
\CatchFileDef{\commit}{\gitfolder/refs/heads/\branch}{}	%Get the content of the branch head
\StrGobbleRight{\commit}{1}[\commithash]					%Remove end of line characted

%Build the URL to this commit based on the information we now have
\newcommand{\commiturl}{\url{https://github.com/\reponame/commit/\commithash}}

%----------------------------------------------------------------------------------------

% Reset the sidenote number each chapter
\let\oldchapter\chapter
\def\chapter{%
  \setcounter{footnote}{0}%
  \oldchapter
}

%----------------------------------------------------------------------------------------


\begin{document}

\frontmatter

%----------------------------------------------------------------------------------------
%	EPIGRAPH
%----------------------------------------------------------------------------------------

%----------------------------------------------------------------------------------------

\maketitle % Print the title page

%----------------------------------------------------------------------------------------
%	COPYRIGHT PAGE
%----------------------------------------------------------------------------------------

\newpage
\begin{fullwidth}
~\vfill
\thispagestyle{empty}
\setlength{\parindent}{0pt}
\setlength{\parskip}{\baselineskip}
Copyright \copyright\ \the\year\ \\ \thanklessauthor

\bigskip\par\smallcaps{Published by \thanklesspublisher}

\par\smallcaps{\url{https://worldbank.github.com/dime-data-handbook}}

\par Compiled from commit: \newline
\vspace{-0.5cm}
\commiturl

\par Released under a Creative Commons Attribution 4.0 International (CC BY 4.0) license.\newline
\vspace{-0.5cm}
\url{https://creativecommons.org/licenses/by/4.0}

\par\textit{First printing, \monthyear}

\end{fullwidth}

%----------------------------------------------------------------------------------------
%	Edition notes
%----------------------------------------------------------------------------------------

\cleardoublepage
\chapter*{Notes on this edition} % The asterisk leaves out this chapter from the table of contents

This is a draft edition of
\textit{Development Research in Practice:
The DIME Analytics Data Handbook}.
We want to thank all the people who helped us get here, especially
Arianna Legovini, for her leadership at DIME, unending support of our work, 
and detailed comments on this book; and 
Florence Kondylis, for her leadership in founding and growing DIME Analytics
and supporting this project from the very first.
We also thank the following members 
of DIME Analytics for their contributions 
to the ideas in this book and their help organizing them: 
Roshni Khincha, Avnish Singh, Patricia Paskov, Radhika Kaul,
Mizuhiro Suzuki, Yifan Powers, and Maria Arnal Canudo.
This work has been financially supported by the United Kingdom Foreign, 
Commonwealth \& Development Office (FCDO) through the
 DIME i2i Umbrella Facility for Impact Evaluation at the World Bank.

Our graditude to the many people who read and offered feedback as the book took shape:
%Alphabetical by last
Stephanie Annijas,
Maria Camila Ayala Guerrero,
Thomas Escande,
Aram Gassama,
Steven Glover,
Nausheen Khan,
Michael Orevba,
Caio Piza,
Francesco Raffaelli,
Daniel Rogger,
Ankriti Singh,
Ravi Somani,
and Leonardo Viotti.
Although they number far too many to name individually, 
we also thank all the members of DIME and its teams across all the years
for the innovative work they have done, the lessons learned,
and the team spirit that makes our work so fruitful and rewarding.

This version of the book has been revised since its release in June 2019
with feedback from readers and experts.
It contains most of the content we plan for the finished version.
This book is a living product that is written and maintained publicly.
The code and edit history are at:
\url{https://github.com/worldbank/dime-data-handbook}.
You can get a PDF copy at:
\url{https://worldbank.github.com/dime-data-handbook}.
The website includes updated instructions
for providing feedback, as well as notes on updates to the content.

Whether you work with DIME, the World Bank,
or another organization or university,
we ask that you read the contents of this book critically.
We welcome feedback and corrections to improve the book. 
Please visit
\url{https://worldbank.github.com/dime-data-handbook/feedback} 
to provide feedback.
You can also email us at \url{dimeanalytics@worldbank.org}, 
and we will be very thankful.
We hope you enjoy \textit{Development Research in Practice}!


%----------------------------------------------------------------------------------------
%	Abbreviations
%----------------------------------------------------------------------------------------

\cleardoublepage
\chapter*{Abbreviations} % The asterisk leaves out this chapter from the table of contents

\noindent\textbf{2SLS} -- Two-Stage Least Squares

\noindent\textbf{AEA} -- American Economic Association

\noindent\textbf{CAPI} -- Computer-Assisted Personal Interviewing

\noindent\textbf{DD or DiD} -- Differences-in-Differences

\noindent\textbf{DIME} -- Development Impact Evaluations

\noindent\textbf{HFC} -- High-Frequency Checks

\noindent\textbf{IRB} -- Instituional Review Board

\noindent\textbf{IV} -- Instrumental Variables

\noindent\textbf{MDE} -- Minimum Detectable Effect

\noindent\textbf{NGO} -- Non-Governmental Organization

\noindent\textbf{ODK} -- Open Data Kit

\noindent\textbf{OLS} -- Ordinary Least Squares

\noindent\textbf{OSF} -- Open Science Foundation

\noindent\textbf{PI} -- Principal Investigator

\noindent\textbf{PII} -- Personally-Identifying Information

\noindent\textbf{RA} -- Research Assistant

\noindent\textbf{RD} -- Regression Discontinuity

\noindent\textbf{RCT} -- Randomized Control Trial

\noindent\textbf{SSC} -- Statistical Software Components


%----------------------------------------------------------------------------------------

\tableofcontents % Print the table of contents

%----------------------------------------------------------------------------------------

% \listoffigures % Print a list of figures

%----------------------------------------------------------------------------------------

% \listoftables % Print a list of tables

%----------------------------------------------------------------------------------------
%	DEDICATION PAGE
%----------------------------------------------------------------------------------------

\cleardoublepage
\thispagestyle{empty}
~\vfill
\begin{doublespace}
\noindent\fontsize{18}{22}\selectfont\itshape
\nohyphenation
Dedicated to all the research assistants who have
wrangled data without being taught how,
hustled to get projects done on time,
wondered if they really should get their PhD after all,
and in doing so made this knowledge necessary and possible.
\end{doublespace}
\vfill
\vfill
