%------------------------------------------------

\begin{fullwidth}
Increasingly, research assistants are relied on to manage some or all
of the publication process. This can include
managing the choice of software,
coordinating referencing and bibliography,
tracking changes across various authors and versions,
and preparing final reports or papers for release or submission.
Modern software tools can make a lot of these processes easier.
Unfortunately there is some learning curve,
particularly for lead authors who have been publishing for a long time.
This chapter suggests some tools and processes
that can make writing and publishing in a team significantly easier.
It will provide resources
to judge how best to adapt your team to the tools you agree upon,
since all teams vary in composition and technical experience.

Ideally, your team will spend as little time as possible
fussing with the technical requirements of publication.
It is in nobody's interest for a skilled and busy research assistant
to spend days re-numbering references (and it can take days)
if a small amount of up-front effort could automate the task.
However, experienced academics will likely have a workflow
with which they are already comfortable,
and since they have worked with many others in the past,
that workflow is likely to be the least-common-denominator:
Microsoft Word with tracked changes.
This chapter will show you how you can avoid at least some
of the pain of Microsoft Word,
while still providing materials in the format
that co-authors prefer and journals request.
\end{fullwidth}

%------------------------------------------------

\section{Collaborating on academic writing}

The gold standard for academic writing is \LaTeX.\sidenote{\url{https://www.maths.tcd.ie/~dwilkins/LaTeXPrimer/GSWLaTeX.pdf}}
\index{\LaTeX}
\LaTeX\ allows automatically-organized sections like titles and bibliographies,
imports tables and figures in a dynamic fashion,
and can be version controlled using Git.
Unfortunately, \LaTeX\ can be a challenge to set up and use at first,
particularly for people who are unfamiliar with plaintext, code, or file management.
\LaTeX\ requires that all formatting be done in its special code language,
and it is not particularly informative when you do something wrong.
This can be off-putting very quickly for people
who simply want to get to writing, like lead authors.
Therefore, if we want to take advantage of the features of \LaTeX,
without getting stuck in the weeds of it,
we will need to adopt a few tools and tricks to make it effective.

The first is choice of software. The easiest way
for someone new to \LaTeX\ to be able to ``just write''
is often the web-based Overleaf suite.\sidenote{\url{https://www.overleaf.com}}
Overleaf offers a \textbf{rich text editor}
that behaves pretty similarly to familiar tools like Word.
TeXstudio\sidenote{\url{https://www.texstudio.org}} and atom-latex\sidenote{\url{https://atom.io/packages/atom-latex}}
are two popular desktop-based tools for writing \LaTeX;
they allow more advanced integration with Git,
among other advantages, but the entire team needs to be comfortable
with \LaTeX\ before adopting one of these tools.
With minimal workflow adjustments, you can
to show coauthors how to write and edit in Overleaf,
so long as you make sure you are always available to troubleshoot
\LaTeX\ crashes and errors.
The most common issue will be special characters, namely
\texttt{\&}, \texttt{\%}, and \texttt{\_},
which need to be \textbf{escaped} (instructed to interpret literally)
by writing a backslash (\texttt{\textbackslash}) before them,
such as  \texttt{40\textbackslash\%} for the percent sign to function.
Overleaf offers a convenient selection of templates
so it is easy to start up a project
and replicate a lot of the underlying setup code.
The main issue with using Overleaf is that you need to upload input files
(such as figures and tables) manually.
This can create conflicts when these inputs are still being updated --
namely, the main document not having the latest results.
One solution is to move to Overleaf only once there will not be substantive changes to results.

One of the important tools available in \LaTeX\ is the BibTeX bibliography manager.\sidenote{\url{http://citeseerx.ist.psu.edu/viewdoc/download?doi=10.1.1.365.3194&rep=rep1&type=pdf}}
This tool stores unformatted references
in an accompanying \texttt{.bib} file,
and \LaTeX\ then inserts them in text
using the \texttt{\textbackslash cite\{\}} command.
With this structure, \LaTeX\ automatically pulls
all the citations into text. \LaTeX\ allows you to specify
how they should be displayed in text
(ie, as superscripts, inline references, etc.)
and how the bibliography should be styled and in what order.
A reference in the \texttt{.bib} file
can be copied directly from Google Scholar
by clicking "BibTeX" at the bottom of the Cite window.
When pasted into the \texttt{.bib} file they look like the following:

\codeexample{sample.bib}{./code/sample.bib}

\noindent BibTeX citations are then used as follows:

\codeexample{citation.tex}{./code/citation.tex}

With these tools, you can ensure that co-authors are writing
in a format you can manage and control.\cite{flom2005latex}
The purpose of this setup, just like with other synced folders,
is to avoid there ever being more than one master copy of the document.
This means that people can edit simultaneously without fear of conflicts,
and it is never necessary to manually resolve differences in the document.
Finally, \LaTeX\ has one more useful trick:
if you download a journal styler from the \textbf{Citation Styles Library}\sidenote{
\url{https://github.com/citation-style-language/styles}}
and use \textbf{\texttt{pandoc}},\sidenote{\url{http://pandoc.org/}}
you can translate the raw document into Word by running the following code from the command line:

\codeexample{pandoc.sh}{./code/pandoc.sh}

Therefore, even in the case where you are requested to provide
\texttt{.docx} versions of materials to others, or tracked-changes versions,
you can create them effortlessly,
and use Word's compare feature to generate a single integrated tracked version.

%------------------------------------------------

\section{Publishing data and code for replication}

Data and code should always be released with any publication.\sidenote{\url{https://dimewiki.worldbank.org/wiki/Publishing_Data}}
\index{data publication}
Many journals and funders have strict open data policies,
and providing these materials to other researchers
allows them to evaluate the credibility of your work
as well as to re-use your materials for further research.\sidenote{\url{https://dimewiki.worldbank.org/wiki/Exporting_Analysis}}
If you have followed the steps in this book carefully,
you will find that this is a very easy requirement to fulfill.\sidenote{
\url{https://www.bitss.org/2016/05/23/out-of-the-file-drawer-tips-on-prepping-data-for-publication/}}
You will already have a publishable (either de-identified or constructed)
version of your research dataset.
You will already have your analysis code well-ordered,
and you won't have any junk lying around for cleanup.
You will have written your code so that others can read it,
and you will have documentation for all your work,
as well as a well-structured directory containing the code and data.

If you are at this stage,
all you need to do is find a place to publish your work!
GitHub provides one of the easiest solutions here,
since it is completely free for static, public projects
and it is straightforward to simply upload a fixed directory
and obtain a permanent URL for it.
The Open Science Framework also provides a good resource,
as does ResearchGate (which can also assign a permanent
digital object identifier link for your work).
Any of these locations is acceptable --
the main requirement is that the system can handle
the structured directory that you are submitting,
and that it can provide a stable, structured URL for your project.

You should release a structured directory that allows a user
to immediately run your code after changing the project directory.
The folders should include:
all necessary de-identified data for the analysis
(including only the data needed for analysis);
the data documentation for the analysis code
(describing the source and construction of variables);
the ready-to-run code necessary for the analysis; and
the raw outputs you have used for the paper.
Using \texttt{iefolder} from our \texttt{ietoolkit} can help standardize this in Stata.
In either the \texttt{/dofiles/} folder or in the root directory,
include a master script (\texttt{.do} or \texttt{.r} for example).
The master script should allow the reviewer to change
one line of code setting the directory path.
Then, running the master script should run the entire project
and re-create all the raw outputs exactly as supplied.
Check that all your code will run completely on a new computer.
This means installing any required user-written commands in the master script
(for example, in Stata using \texttt{ssc install} or \texttt{net install}
and in R include code for installing packages,
including installing the appropriate version of the package if necessary).
Make sure settings like \texttt{version}, \texttt{matsize}, and \texttt{varabbrev} are set.\sidenote{In Stata, \texttt{ietoolkit}'s \texttt{ieboilstart} command will do this for you.}
All outputs should clearly correspond by name to an exhibit in the paper, and vice versa.

Finally, you may also want to release an author's copy or preprint.
Check with your publisher before doing so;
not all journals will accept material that has been released.
Therefore you may need to wait until acceptance is confirmed.
This can be done on a number of pre-print websites,
many of which are topically-specific.
You can also use GitHub and link the file directly
on your personal website or whatever medium you are
sharing the preprint through.
Do not use Dropbox or Google Drive for this purpose:
many organizations do not allow access to these tools,
and that includes blocking staff from accessing your material.
