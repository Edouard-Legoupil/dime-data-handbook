%-----------------------------------------------------------------------------------------------

\begin{fullwidth}
Research design is the process of defining the methods and data
that will be used to answer a specific research question.


Thinking through research design before starting data work is important for several reasons. You will save a lot of time by understanding the way
your data needs to be organized
in order to be able to produce meaningful analytics throughout your projects.




\end{fullwidth}

%-----------------------------------------------------------------------------------------------
%-----------------------------------------------------------------------------------------------

\section{Different research designs different data requirement }

Your project's data needs will differ depending on what research design your project use. 
There are already many great resources on research design,
so this chapter will only cover how they impact your data needs and related tools. 
We assume that the reader have some level of familiarity with the resarch designs mentioned here. 
If not go and read the appendix XYZ where you have more details and links to even more details.  

All research designs discussed here compare a group that received some treatment\sidenote{
	\textbf{Treatment:} The general word for the event we evaluate the impact of. 
	That event can be receiving training or cash transfer from a program, experience a natural disaster etc.}
 to another, counterfactual group.\sidenote{
	\textbf{Counterfactual:} A statistical description of what would have happened to specific individuals in an alternative scenario,
	 for example, a different treatment assignment outcome.}
\index{counterfactual} 
The key assumption is that every
person, facility, or village (or whatever the unit of intervention is)
has two possible states: their outcomes if they do not receive some treatment
and their outcomes if they do receive that treatment.
The impact of the treatment is defined as the difference in these two states
,however, we can never observe the same unit in our data
in both their treated and untreated states simultaneously.

Instead, the treated observations are compared to observations
that are \textit{statistically similar} in a \textbf{control} group. 
Different research designs have different methods 
for how the \textit{statistically similar} control observation are identified. 
You need a PhD in economics to fully navigate this, 
but this section we will cover how that affect how you should plan your data accordingly.

\textit{Statistically similarity} can be tested using \textbf{balance checks} 
where the similarity between treatment and control groups can be formally tested. 
Since this test is so common, 
we have developed a Stata command called \texttt{iebaltab}\sidenote{
	\url{https://dimewiki.worldbank.org/iebaltab}}
 in the package \texttt{ietoolkit} that generates a table of balance checks.


\subsection{Identification of control groups in different research designs}

%%%%% Experimental 

In \textbf{experimental research designs} the research team can control which part of the studied population will get the treatment. 
This is often done by random assignment\sidenote{For example \textbf{randomized control trial (RCT) --}
	\url{https://dimewiki.worldbank.org/Randomized_Control_Trials}}
\index{randomized control trials}
where a subset of the eligible population is randomized to receive the treatment (see later in this chapter for how to implement randomization). 
The intuition is that if everyone in the eligible population is assigned group randomly, then they will, on average, be statistically similar.

To randomly assign treatment you need data over all individuals in the eligible population. 
This can be a census when running a traditional household survey, 
but it can also be anything from all companies in a country's tax records
to all Twitter accounts that liked a tweet.
It is important that the completeness of the eligible population in your data has no bias, 
for example missing many poor households, 
as that bias will then be included in your research design and your results will have the same bias.

%%%%% Quasi Experimental 

\textbf{Quasi-experimental} research designs,\sidenote{
	\url{https://dimewiki.worldbank.org/Quasi-Experimental_Methods}}
by contrast, are causal inference methods based on events not controlled by the research team. Instead, they rely on ``experiments of nature'',
in which natural variation can be argued to approximate randomization. 

Unlike carefully planned experimental designs,
quasi-experimental designs typically require the extra luck
of having access to data collected at the right times and places
to exploit events that occurred in the past,
or having the ability to collect data in a time and place
where an event that produces causal identification occurred or will occur.

Therefore, these methods often use either secondary data,
or they use primary data in a cross-sectional retrospective method,
including administrative data or other new classes of routinely-collected information.

%%%%% Regression discontinuity

\textbf{Regression discontinuity (RD)} designs exploit sharp breaks or limits
in policy designs to separate a single group of potentially eligible recipients
into comparable groups of individuals who do and do not receive a treatment.\sidenote{
	\url{https://dimewiki.worldbank.org/Regression_Discontinuity}} \index{regression discontinuity}
In an RD design, there is typically some program or event
that has limited availability due to practical considerations or policy choices
and is therefore made available only to individuals who meet a certain threshold requirement.

Common examples are test score thresholds and income thresholds.\sidenote{
	\url{https://blogs.worldbank.org/impactevaluations/regression-discontinuity-porn}}

The key assumption here is that the running variable cannot be directly manipulated
by the potential recipients.

%%%%% IV regression

\textbf{Instrumental variables (IV)} designs, unlike the previous approaches,
assume that the treatment not directly identifiable.
Instead, similar to regression discontinuity designs,
IV focus on a subset of the variation in treatment take-up, 
but whereas regression discontinuity designs are ``sharp'' --
treatment status is completely determined by which side of a cutoff an individual is on --
IV designs are ``fuzzy'', meaning that they do not completely determine
the treatment status but instead influence the \textit{probability} of treatment.

In practice, there are a variety of packages that can be used
to analyse data and report results from instrumental variables designs.
While the built-in Stata command \texttt{ivregress} will often be used
to create the final results, the built-in packages are not sufficient on their own.

%%%%% Matching

\textbf{Matching}\sidenote{
	\url{https://dimewiki.worldbank.org/Matching}}
 methods use observable characteristics to construct pairs of treatment and control groups so that the observations in each group is as similar as possible. \index{matching}
This can be done before randomization where the matching group is the unit used in the randomization,
or it can be done after the treatment is assigned to get a counterfactual observations for each treatment observation.

Matching are often done in one-to-one pair but can also be done in one-to-many or many-to-many groups. DIME's \texttt{iematch} command in the \texttt{ietoolkit} package produces matchings based on a single continuous matching variable.\sidenote{
	\url{https://dimewiki.worldbank.org/iematch}}


%-----------------------------------------------------------------------------------------------


\subsection{One observation or multiple observations over time}

Most of the research designs in the previous section can be implemented 
using data collected only after the treatment, 
or using data collected at multiple time periods, 
for example before and after the treatment. 
The advantage of multiple points in time is 
that you can control for each observations initial status.

A study that observes data in only one time period is called 
a \textbf{cross-sectional study} is any type of study. 
This type of data is easy to collect and handle because
you do not need to track individuals across time. 
Instead, the challenge in a cross-sectional study is to
show that the control group is indeed a valid counterfactual to the treatment group.

A study that observes data in multiple time periods can either be a 
\textbf{repeated cross-sections study} or a \textbf{panel data study} 
depending on if the same sample is used in the multiple observations.

In repeated cross-sections, each successive round of data collection contains a random sample
of observations from the treatment and control groups;
as in cross-sectional designs, both the randomization and sampling processes
are critically important to maintain alongside the data.

In panel data structures, we attempt to observe the exact same units
in different points in time, so that we see the same individuals
both before and after they have received treatment (or not).\sidenote{
	\url{https://blogs.worldbank.org/impactevaluations/what-are-we-estimating-when-we-estimate-difference-differences}}

When tracking individuals over time for this purpose,
maintaining sampling and tracking records is especially important,
because attrition will remove that unit's information
from all points in time, not just the one they are unobserved in.
Panel-style studies therefore require a lot more effort in field work
for studies that use original data.\sidenote{
	\url{https://www.princeton.edu/~otorres/Panel101.pdf}}

Where cross-sectional designs draw their estimates of treatment effects
from differences in outcome levels in a single measurement,
\textbf{differences-in-differences}\sidenote{
	\url{https://dimewiki.worldbank.org/Difference-in-Differences}}
designs (abbreviated as DD, DiD, diff-in-diff, and other variants)
estimate treatment effects from \textit{changes} in outcomes
between two or more rounds of measurement.
\index{difference-in-differences}
In these designs, three control groups are used –
the baseline level of treatment units,
the baseline level of non-treatment units,
and the endline level of non-treatment units.\sidenote{
	\url{https://www.princeton.edu/~otorres/DID101.pdf}}

ANCOVA?

As with cross-sectional designs, difference-in-differences designs are widespread.
Therefore there exist a large number of standardized tools for analysis.
Our \texttt{ietoolkit} Stata package includes the \texttt{ieddtab} command
which produces standardized tables for reporting results.\sidenote{
  \url{https://dimewiki.worldbank.org/ieddtab}}

