\section{Cleaning data for analysis}

% What is data cleaning
\textbf{Data cleaning} is a widely used expression used to refer to different tasks.
The cleaning process, as defined in this book, involves
(1) making the dataset easy to use and understand, and 
(2) carefully exploring each variable to document their distributions and identify patterns that may bias the analysis.
The resulting dataset will contain only the variables collected in the field, and
no modifications to data points will be made, 
except for corrections of mistaken entries.
Apart from the \textbf{cleaned dataset} (or datasets) itself,
cleaning will also yield extensive documentation describing  it.

% Section overview
During data cleaning, you will acquire in-depth understanding of the contents and structure of your data.
This knowledge will be key to correctly construct final indicators and analyze them.
So don't rush through this step.
Explore the dataset using tabulations, summaries, and descriptive plots.
It is common for cleaning to be the most time-consuming task in a project.
In this section, we will introduce some concepts and tools to make it more efficient and productive.

%\subsection{Correcting data points}

\subsection{Recoding and annotating data}

% Why recoding and annotating data are important
The cleaned dataset is the starting point of data analysis.
It will be extensively manipulated to construct analysis indicators,
so it is important for it to be easily processed by statistical software.
To make the analysis process smoother, 
anyone opening it for the first time should have all the information needed to interact with it,
even if they were not involved in the acquisition or cleaning process.
This will save them time going back as forward between the dataset and its accompanying documentation. 

% Encoding variables
Often times, datasets are not imported into statistical software in the most efficient format.
The most common example are string variables:
categorical variables and open-ended responses are often read as strings.
However, variables in these format cannot be used for quantitative analysis.
Therefore, categorical variables must be transformed into easier to use formats,
such as \texttt{factors} in R and \texttt{labeled integers}\sidenote{https://dimewiki.worldbank.org/wiki/Data_Cleaning#Value_Labels} in Stata.
Additionally, open-ended responses stored as strings usually have a high risk of being identifiers, 
so cleaning them requires extra attention.
The option names in categorical variables
(called \textit{value labels} in Stata and \textit{levels} in R)
should be accurate and concise, 
and correspond exactly to the data collection instrument.
Adding option names to categorical variables 
makes it easier to understand your data as you explore it,
and thus reduces the risk of small errors making their way through into the analysis stage.

% Recoding missing values
In survey data, it is common for non-responses such as ``Don't know'' and ``Declined to answer''
to be represented by negative survey codes. 
The presence of these negative values could bias your analysis,
since they don't represent actual observations of a variable.
So they need to be turned into \textit{missing values}.
However, the fact that a respondent didn't know how to answer a question is also useful information,
that would be lost by this transformation.
In Stata, this information can be elegantly conserved using extended missing values.\sidenote{
	\url{https://dimewiki.worldbank.org/Data\_Cleaning\#Survey\_Codes\_and\_Missing\_Values}}

% Labelling variables
We recommend that the cleaned data set by kept as similar to the raw data as possible.
This is particularly important regarding variable names:
keeping them consistent with the raw data makes data processing and construction more transparent.
Unfortunately, not all variable names are informative.
In such cases, one important piece of documentation,
the variable dictionary, makes the data easier to handle.
When a data collection instrument is available, 
it is often the best dictionary one could ask for.
But even in this cases, going back-and-forth between files can be inefficient,
so annotating variables in a dataset is extremely useful.
In Stata, \textit{variable labels}\sidenote{\url{
	https://dimewiki.worldbank.org/wiki/Data_Cleaning#Variable_Labels}} must always be present in a cleaned dataset.
They should include a short and clear description of the variable.
A lengthier description, that may include, for example,
the exact wording of a question, may be added through \textit{variable notes}.
In R, it is less common to use variable labels,
and a separate dataset with a variable dictionary is often preferred,
but \texttt{dataframe attributes} can be used for the same purpose.

% tools: iecodebook, tidyverse
Although all these tasks are key to making the data easy to use,
implementing them can be quite repetitive and create convoluted scripts.
The \texttt{iecodebook} command suite, part of the \texttt{iefieldkit} Stata package,
is designed to make some of the most tedious components of this process more efficient.\sidenote{
	\url{https://dimewiki.worldbank.org/iecodebook}}
\index{iecodebook}
It also creates a self-documenting workflow,
so your data cleaning documentation is created alongside that code,
with no extra steps.
As far as we know, there are no similar resources in R.
However, the \texttt{tidyverse}\sidenote{https://www.tidyverse.org/} packages
compose a consistent and useful grammar to perform the same tasks.

