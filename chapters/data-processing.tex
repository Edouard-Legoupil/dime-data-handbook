\section{Data quality assurance}

Whether you are acquiring data from a partner or collecting it directly,
it is important to make sure that data faithfully reflects ground realities.
Data quality assurance requires a combination of real-time data checks
and back-checks or validation audits, which often means tracking down
the people whose information is in the dataset.

\subsection{Implementing high frequency quality checks}

% What are HFCs
A key advantage of continuous electronic data intake methods,
as compared to traditional paper surveys and one-time data dumps,
is the ability to access and analyze the data while the project is ongoing.
Data issues can be identified and resolved in real-time.
Designing systematic data checks and running them routinely throughout data intake
simplifies monitoring and improves data quality.
As part of data acquisition preparation,
the research team should develop a \textbf{data quality assurance plan}\sidenote{
	\url{https://dimewiki.worldbank.org/Data\_Quality\_Assurance\_Plan}}.
While data acquisition is ongoing,
a research assistant or data analyst should work closely with the field team or partner
to ensure that the data collection is progressing correctly,
and set up and perform \textbf{high-frequency checks (HFCs)} with the incoming data.

% Why they should be made in real-time
High-frequency checks (HFCs) should carefully inspect key treatment and outcome variables
so that the data quality of core experimental variables is uniformly high,
and that additional effort is centered where it is most important.
Data quality checks should be run on the data every time it is received from the field or partner
to flag irregularities in the aquisition progress, in sample completeness, or in response quality.
The faster issues are identified, the more likely they are to be solved.
\texttt{ipacheck}\sidenote{
	\url{https://github.com/PovertyAction/high-frequency-checks/wiki}}
is a very useful command that automates some of these tasks,
regardless of the source of the data.

% Completeness
It is important to check continuously that the observations in the data match the intended sample.
In surveys, the software often provides some form of case management features
through which sampled units are directly assigned to individual enumerators.
For data received from partners, this may be harder to validate,
since they are the authoritative source of the data,
so cross-referencing with other data sources may be necessary to ensure completeness of the data.
Even with careful management, it is often the case that raw data includes duplicate or missing entries,
which may occur due to data entry errors or failed submissions to data servers.\sidenote{
	\url{https://dimewiki.worldbank.org/Duplicates_and_Survey_Logs}}
\texttt{ieduplicates}\sidenote{
	\url{https://dimewiki.worldbank.org/ieduplicates}}
provides a workflow for collaborating on the resolution of duplicate entries between you and the provider.
Then, observed units in the data must be validated against the expected sample:
this is as straightforward as merging the sample list with the survey data and checking for mismatches.
Reporting errors and duplicate observations in real-time allows the field team to make corrections efficiently.
Tracking data collection progress is important for monitoring attrition,
so that it is clear early on if a change in protocols or additional tracking will be needed.
It is also important to check data collection completion rate
and sample compliance by surveyor and survey team, if applicable,
or compare data missingness across administrative regions,
to identify any clusters that may be providing data of suspect quality.

% Consistency
High frequency checks should also include content-specific data checks.
Electronic survey and data entry software often incorporates many quality control features,
so these checks should focus on issues survey software cannot check automatically.
As most of these checks are project specific,
it is difficult to provide general guidance.
An in-depth knowledge of the questionnaire and a careful examination of the analysis plan
is the best preparation.
Examples include verifying consistency across multiple response fields or data sources,
validation of complex calculations like crop yields or medicine stocks (which require unit conversions),
suspicious patterns in survey timing,
or atypical response patters from specific data sources or enumerators.\sidenote{
	\url{https://dimewiki.worldbank.org/Monitoring_Data_Quality}}
Electronic data entry software typically provides rich metadata,
which can be useful in assessing data quality.
For example, automatically collected timestamps show when data was submitted
and (for surveys) how long enumerators spent on each question,
and trace histories show how many
times answers were changed before or after the data was submitted.

% Following up on flagged issues
High-frequency checks will only improve data quality
if the issues they catch are communicated to the data provider.
There are lots of ways to do this;
what's most important is to find a way to create actionable information for your team.
\texttt{ipacheck}, for example, generates a spreadsheet with flagged errors;
these can be sent directly to the data collection teams.
Many teams choose other formats to display results,
such as online dashboards created by custom scripts.
It is also possible to automate communication of errors to the field team
by adding scripts to link the HFCs with a messaging program such as WhatsApp.
Any of these solutions are possible:
what works best for your team will depend on such variables as
cellular networks in field work areas, whether field supervisors have access to laptops,
internet speed, and coding skills of the team preparing the HFC workflows.

\subsection{Conducting back-checks and data validation}

% Conducting back-checks
Careful validation of data is essential for high-quality data.
Since we cannot control natural measurement error
that comes from variation in the realization of key outcomes,
original data collection provides the opportunity to make sure
that there is no error arising from inaccuracies in the data itself.
\textbf{Back-checks}\sidenote{\url{https://dimewiki.worldbank.org/Back\_Checks}} and
other validation audits help ensure that data collection is following established protocols,
and that data is not fasified, incomplete, or otherwise suspect.
For back-checks and validation audits, a random subset of the main data is selected,
and a subset of information from the full survey is
verified through a brief targeted survey with the original respondent
or a cross-referenced dataset from another source (if the original data is not a field survey).
Design of the back-checks or validations follows the same survey design
principles discussed above: you should use the analysis plan
or list of key outcomes to establish which subset of variables to prioritize,
and similarly focus on errors that would be major flags for poor quality data.

% How to implement back-checks
Real-time access to the data massively increases the potential utility of validation,
and both simplifies and improves the rigor of the associated workflows.
You can use the raw data to draw the back-check or validation sample;
this ensures that the validation is correctly apportioned across observations.
As soon as checking is complete, the validation data can be tested against
the original data to identify areas of concern in real-time.
The \texttt{bcstats} command is a useful tool for analyzing back-check data in Stata.\sidenote{
	\url{https://ideas.repec.org/c/boc/bocode/s458173.html}}
Some electronic surveys surveys also provide a unique opportunity
to do audits through audio recordings of the interview,
typically short recordings triggered at random throughout the questionnaire.
\textbf{Audio audits} are a useful means to assess whether enumerators are conducting interviews as expected.
Do note, however, that audio audits must be included in the informed consent for the respondents.

\section{De-identifying research data}
\subsection{Protecting privacy as researcher}

% Dealing with human subjects
Most of the field research done in development involves human subjects.\sidenote{
	\url{https://dimewiki.worldbank.org/Human_Subjects_Approval}}
\index{human subjects}
As a researcher, you may have access to personal information about your subjects:
where they live, how rich they are, whether they have committed or been victims of crimes,
their names, their national identity numbers, among other sensitive data.
PII data carries strict expectations about data storage and handling,
and it is the responsibility of the research team to satisfy these expectations.\sidenote{
	\url{https://dimewiki.worldbank.org/Research\_Ethics}}
Your donor or employer will most likely require you to hold a certification from a source
such as Protecting Human Research Participants\sidenote{
	\url{https://phrptraining.com}}
or the CITI Program;\sidenote{
	\url{https://about.citiprogram.org/en/series/human-subjects-research-hsr}}
and your team will need to present a plan to handle this data securely for IRB approval.

% Options for dealing with PII data: only collect it if extremely necessary, encrypt it, restrict access, de-identify it
In general, though, you shouldn't need to handle PII data very often
once the data collection processes are completed.
You can take simple steps to avoid risks by minimizing the handling of PII.
First, only collect information that is strictly needed for the research.
Second, avoid the proliferation of copies of identified data.
There should never be more than one copy of the raw identified dataset in the project folder,
and it must always be encrypted.
Even within the research team,
access to PII data should be limited to team members who require it for their specific tasks.
Data analysis that requires identifying information is rare
and in most cases can be avoided by properly linking identifiers to research information
such as treatment statuses and weights, then removing identifiers.

% De-identification vs anonymization
Therefore, once data is securely collected and stored,
the first thing you will generally do is \textbf{de-identify} it,
that is, remove direct identifiers of the individuals in the dataset.\sidenote{
	\url{https://dimewiki.worldbank.org/De-identification}}
\index{de-identification}
Note, however, that it is in practice impossible to \textbf{anonymize} data.
There is always some statistical chance that an individual's identity
will be re-linked to the data collected about them
-- even if that data has had all directly identifying information removed --
by using some other data that becomes identifying when analyzed together.
For this reason, we recommend de-identification in two stages.
The \textbf{initial de-identification} process strips the data of direct identifiers
as early in the process as possible,
to create a working de-identified dataset that
can be shared \textit{within the research team} without the need for encryption.
This simplifies workflows.
The \textbf{final de-identification} process involves
making a decision about the trade-off between
risk of disclosure and utility of the data
before publicly releasing a dataset.\sidenote{
	\url{https://sdcpractice.readthedocs.io/en/latest/SDC\_intro.html\#need-for-sdc}}
The following section describes how to implement these processes.


\subsection{De-identification in practice}

% Initial de-identification
To simplify workflows, the raw data should be de-identified as early as possible.
Once you create a de-identified version of the dataset,
you no longer need to interact directly with the encrypted raw data.
During the initial round of de-identification, 
dataset must be stripped of personally identifying information.\sidenote{
	\url{https://dimewiki.worldbank.org/De-identification}}
To do so, you will need to identify all variables that contain
such information.\sidenote{\url{
		https://www.povertyactionlab.org/sites/default/files/resources/J-PAL-guide-to-deidentifying-data.pdf}}
For data collection, where the research team designs the survey instrument,
flagging all potentially identifying variables in the questionnaire design stage
simplifies the initial de-identification process.
If you did not do that, or you received original data by another means,
there are a few tools to help flag variables with personally-identifying data.
JPAL's \texttt{PII scan}, as indicated by its name,
scans variable names and labels for common string patterns associated with identifying information.\sidenote{
	\url{https://github.com/J-PAL/PII-Scan}}
The World Bank's \texttt{sdcMicro}
lists variables that uniquely identify observations, 
but it is less efficient for large datasets.\sidenote{
	\url{https://sdctools.github.io/sdcMicro/articles/sdcMicro.html}}
The \texttt{iefieldkit} command \texttt{iecodebook}
lists all variables in a dataset and exports an Excel sheet
where you can easily select which variables to keep or drop.\sidenote{
	\url{https://dimewiki.worldbank.org/Iecodebook}}

% Initial de-identification in practice
Once you have a list of variables that contain confidential information,
assess them against the analysis plan and first ask yourself for each variable:
\textit{will this variable be needed for the analysis?}
If not, the variable should be dropped.
Don't be afraid to drop too many variables the first time,
as you can always go back and remove variables from the list of variables to be dropped,
but you can not go back in time and drop a PII variable that was leaked
because it was incorrectly kept.
Examples include respondent names and phone numbers, enumerator names, taxpayer 
numbers, and addresses.
For each confidential variable that is needed in the analysis, ask yourself:
\textit{can I encode or otherwise construct a variable that masks the confidential component, and
	then drop this variable?}
This is typically the case for most identifying information.
Examples include geocoordinates
(after constructing measures of distance or area,
drop the specific location)
and names for social network analysis (can be encoded to secret and unique IDs).
If the answer to either of the two questions above is yes,
all you need to do is write a script to drop the variables that are not required for analysis,
encode or otherwise mask those that are required,
and save a working version of the data.
If confidential information strictly required for the analysis itself and can not be
masked or encoded,
it will be necessary to keep at least a subset of the data encrypted through
the data analysis process.

% Final de-identification: sdcMicro
The dataset created by the initial de-identification will be the underlying source for all cleaned and constructed data.
This is the dataset that you will interact with directly during the remaining tasks described in this chapter.
Because identifying information is typically only used during data collection,
when teams need to find and confirm the identity of interviewees,
de-identification should not affect the usability of the data.
However, access to this data should still be restricted to the research team only,
as it's possible that combining a group of variables,
or taking into account information that is not the dataset,
may lead to the identification of the individuals in the sample.
It is common, and even desirable, for teams to make data publicly available
once the tasks discussed in this chapter are concluded.
This will allow other researchers to conduct additional analysis and to reproduce your finding.
Before that can be done, however,
a careful analysis of the risk of disclosure and a final round of de-identification must be conducted.
The World Bank's \texttt{sdcMicro} allos for sophisticated disclosure risk calculations.\sidenote{
	\url{https://sdctools.github.io/sdcMicro/articles/sdcMicro.html}}
There is no clear cut about what is an acceptable level of disclosure,
so your team must weight the trade-off between transparency and privacy,
and make a decision that is specific to the data.
In general, it's preferable to be conservative and include additional layers of security
to access variable that represent a higher disclosure risk.


\section{Cleaning data for analysis}

% What is data cleaning
\textbf{Data cleaning} is a widely used expression used to refer to different tasks.
The cleaning process, as defined in this book, involves
(1) making the dataset easy to use and understand, and 
(2) carefully exploring each variable to document their distributions and identify patterns that may bias the analysis.
The resulting dataset will contain only the variables collected in the field, and
no modifications to data points will be made, 
except for corrections of mistaken entries.
Apart from the \textbf{cleaned dataset} (or datasets) itself,
cleaning will also yield extensive documentation describing  it.

% Section overview
During data cleaning, you will acquire in-depth understanding of the contents and structure of your data.
This knowledge will be key to correctly construct final indicators and analyze them.
So don't rush through this step.
Explore the dataset using tabulations, summaries, and descriptive plots.
It is common for cleaning to be the most time-consuming task in a project.
In this section, we will introduce some concepts and tools to make it more efficient and productive.
Values make sense with the variable

\subsection{Getting to know your data}

% What to look for when exploring the data
The first time you interact with your data is during quality checks.
However, these checks are often programmed before you receive all the data.
Furthermore, because quality checks are usually time-sensitive, 
there may not be time to explore the data at length.
During data cleaning, on the other hand, 
you will need to inspect each variable closely.
Use tabulations, summary statistics, histograms and density plots to understand the structure of data,
and look for irregularities.
Think critically about what you see:
are the values shown consistent with the information the variable represents?
How do distributions look? 
Are there any outliers, or missing values?
Could distributional patterns be caused by data entry errors?
Are variables consistent with each other?

% Document patterns rather than fix them
At this point, it is more important to document your findings
than to try to address any irregularities found.
There is a very limited set of changes that should be made to the raw data during cleaning.
They are described in the next two sections,
and are usually applied to each variable as you explore them.
Most transformation resulting in data points that were not directly observed in the original data
will be done during \textbf{construction}, a process discussed in the next chapter.
For now, focus on creating a record of what you observe,
and save it in the \texttt{Documentation} folder that corresponds to the data being explored.
You will use this documentation when discussing with your team
how to address irregularities when you get to the construction stage.
This material will also be valuable during exploratory data analysis.

%\subsection{Correcting data points}

\subsection{Recoding and annotating data}

% Why recoding and annotating data are important
The cleaned dataset is the starting point of data analysis.
It will be extensively manipulated to construct analysis indicators,
so it is important for it to be easily processed by statistical software.
To make the analysis process smoother, 
anyone opening it for the first time should have all the information needed to interact with it,
even if they were not involved in the acquisition or cleaning process.
This will save them time going back as forward between the dataset and its accompanying documentation. 

% Encoding variables
Often times, datasets are not imported into statistical software in the most efficient format.
The most common example are string variables:
categorical variables and open-ended responses are often read as strings.
However, variables in these format cannot be used for quantitative analysis.
Therefore, categorical variables must be transformed into easier to use formats,
such as \texttt{factors} in R and \texttt{labeled integers}\sidenote{https://dimewiki.worldbank.org/wiki/Data\_Cleaning\#Value\_Labels} in Stata.
Additionally, open-ended responses stored as strings usually have a high risk of being identifiers, 
so cleaning them requires extra attention.
The option names in categorical variables
(called \textit{value labels} in Stata and \textit{levels} in R)
should be accurate and concise, 
and correspond exactly to the data collection instrument.
Adding option names to categorical variables 
makes it easier to understand your data as you explore it,
and thus reduces the risk of small errors making their way through into the analysis stage.

% Recoding missing values
In survey data, it is common for non-responses such as ``Don't know'' and ``Declined to answer''
to be represented by negative survey codes. 
The presence of these negative values could bias your analysis,
since they don't represent actual observations of a variable.
So they need to be turned into \textit{missing values}.
However, the fact that a respondent didn't know how to answer a question is also useful information,
that would be lost by this transformation.
In Stata, this information can be elegantly conserved using extended missing values.\sidenote{
	\url{https://dimewiki.worldbank.org/Data\_Cleaning\#Survey\_Codes\_and\_Missing\_Values}}

% Labeling variables
We recommend that the cleaned data set by kept as similar to the raw data as possible.
This is particularly important regarding variable names:
keeping them consistent with the raw data makes data processing and construction more transparent.
Unfortunately, not all variable names are informative.
In such cases, one important piece of documentation,
the variable dictionary, makes the data easier to handle.
When a data collection instrument is available, 
it is often the best dictionary one could ask for.
But even in this cases, going back-and-forth between files can be inefficient,
so annotating variables in a dataset is extremely useful.
In Stata, \textit{variable labels}\sidenote{\url{
	https://dimewiki.worldbank.org/wiki/Data\_Cleaning\#Variable\_Labels}} must always be present in a cleaned dataset.
They should include a short and clear description of the variable.
A lengthier description, that may include, for example,
the exact wording of a question, may be added through \textit{variable notes}.
In R, it is less common to use variable labels,
and a separate dataset with a variable dictionary is often preferred,
but \texttt{dataframe attributes} can be used for the same purpose.

Although all these tasks are key to making the data easy to use,
implementing them can be quite repetitive and create convoluted scripts.
The \texttt{iecodebook} command suite, part of the \texttt{iefieldkit} Stata package,
is designed to make some of the most tedious components of this process more efficient.\sidenote{
	\url{https://dimewiki.worldbank.org/iecodebook}}
\index{iecodebook}
It also creates a self-documenting workflow,
so your data cleaning documentation is created alongside that code,
with no extra steps.
As far as we know, there are no similar resources in R.
However, the \texttt{tidyverse}\sidenote{https://www.tidyverse.org/} packages
compose a consistent and useful grammar to perform the same tasks.

