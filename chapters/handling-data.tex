%------------------------------------------------

\begin{fullwidth}

	Development research does not just involve real people -- it also affects real people.
	Policy decisions are made every day using the results of briefs and studies,
	and these can have wide-reaching consequences on the lives of millions.
	As the range and importance of the policy-relevant questions asked by development researchers grow,
	so too does the (rightful) scrutiny under which methods and results are placed.
	It is useful to think of research as a public service,
	one that requires you to be accountable to both research participants and research consumers.
	On the research participant side,
	it is essential to respect individual privacy and ensure data security.
	Researchers look deeply into real people's personal lives, financial conditions, and other sensitive subjects.
	Respecting the respondents' right to privacy,
	by intelligently assessing and proactively averting risks they might face,
	is a core tenet of research ethics.
	On the consumer side, it is important to protect confidence in development research
	by following modern practices for transparency and reproducibility.

  Across the social sciences, the open science movement has been fueled by discoveries of low-quality research practices,
	data and code that are inaccessible to the public, analytical errors in major research papers,
	and in some cases even outright fraud. While the development research community has not yet
	experienced any major scandals, it has become clear that there are necessary incremental improvements
	in the way that code and data are handled as part of research.
	Neither privacy nor transparency is an all-or-nothing objective:
	the most important thing is to report the transparency and privacy measures you have taken
  and always strive to do the best that you are capable of with current technology.
	In this chapter, we outline a set of practices that help to ensure
	research participants are appropriately protected and
	research consumers can be confident in the conclusions reached.
  Later chapters will provide more hands-on guides to implementing those practices.

\end{fullwidth}

%------------------------------------------------


%------------------------------------------------

\section{Ensuring privacy and security in research data}

Anytime you are working with original data in a development research project,
you are almost certainly handling data that include \textbf{personally-identifying
	information (PII)}.\index{personally-identifying information}\index{primary data}\sidenote{
	\textbf{Personally-identifying information:} any piece or set of information
	that can be used to identify an individual research subject.
	\url{https://dimewiki.worldbank.org/De-identification\#Personally\_Identifiable\_Information}}
PII data contains information that can, without any transformation, be used to identify
individual people, households, villages, or firms that were part of data collection.
\index{data collection}
This includes names, addresses, and geolocations, and extends to personal information
such as email addresses, phone numbers, and financial information.\index{geodata}\index{de-identification}
It is important to keep in mind data privacy principles not only for the  respondent
but also the PII data of their household members or other individuals who are covered under the survey.
\index{privacy}
In some contexts this list may be more extensive --
for example, if you are working in an environment that is either small, specific,
or has extensive linkable data sources available to others,
information like someone's age and gender may be sufficient to identify them
even though these would not be considered PII in general.
There is no one-size-fits-all solution to determine what is PII, and you will have to use careful judgment in each case
to decide which pieces of information fall into this category.\sidenote{
	\url{https://sdcpractice.readthedocs.io}}

In all cases where this type of information is involved,
you must make sure that you adhere to several core principles.
These include ethical approval, participant consent, data security, and participant privacy.
If you are a US-based researcher, you will become familiar
with a set of governance standards known as ``The Common Rule''.\sidenote{
	\url{https://www.hhs.gov/ohrp/regulations-and-policy/regulations/common-rule/index.html}}
If you interact with European institutions or persons,
you will also become familiar with the General Data Protection Regulation (GDPR),\sidenote{
	\url{http://blogs.lshtm.ac.uk/library/2018/01/15/gdpr-for-research-data}}
a set of regulations governing \textbf{data ownership} and privacy standards.\sidenote{
	\textbf{Data ownership:} the set of rights governing who may access, alter, use, or share data, regardless of who possesses it.}
\index{data ownership}

In all settings, you should have a clear understanding of
who owns your data (it may not be you, even if you collect or possess it),
the rights of the people whose information is reflected there,
and the necessary level of caution and risk involved in
storing and transferring this information.
Even if your research does not involve PII,
it is a prerogative of the data owner to determine who may have access to it.
Therefore, if you are using data that was provided to you by a partner,
they have the right to request that you hold it to uphold the same data security safeguards as you would to PII.
For the purposes of this book,
we will call all any data that may not be freely accessed for these or other reasons \textbf{confidential data}.
Given the increasing scrutiny on many organizations
from recently advanced data rights and regulations,
these considerations are critically important.
Check with your organization if you have any legal questions;
in general, you are responsible for any action that
knowingly or recklessly ignores these considerations.

\subsection{Obtaining ethical approval and consent}

For almost all data collection and research activities that involve
human subjects or PII data,
you will be required to complete some form of \textbf{Institutional Review Board (IRB)} process.\sidenote{
	\textbf{Institutional Review Board (IRB):} An institution formally responsible for ensuring that research meets ethical standards.}
\index{Institutional Review Board}
Most commonly this consists of a formal application for approval of a specific
protocol for consent, data collection, and data handling.\sidenote{
  \url{https://dimewiki.worldbank.org/IRB_Approval}}
Which IRB has sole authority over your project is not always apparent,
particularly if some institutions do not have their own.
It is customary to obtain an approval from a university IRB
where at least one PI is affiliated,
and if work is being done in an international setting,
approval is often also required
from an appropriate local institution subject to the laws of the country where data originates.

One primary consideration of IRBs
is the protection of the people about whom information is being collected
and whose lives may be affected by the research design.
Some jurisdictions (especially those responsible to EU law) view all personal data
as intrinsically owned by the persons who they describe.
This means that those persons have the right to refuse to participate in data collection
before it happens, as it is happening, or after it has already happened.
It also means that they must explicitly and affirmatively consent
to the collection, storage, and use of their information for any purpose.
Therefore, the development of appropriate consent processes is of primary importance.
All survey instruments must include a module in which the sampled respondent grants informed consent to participate.
Research participants must be informed of the purpose of the research,
what their participation will entail in terms of duration and any procedures,
any foreseeable benefits or risks,
and how their identity will be protected.\sidenote{
	\url{https://www.icpsr.umich.edu/icpsrweb/content/datamanagement/confidentiality/conf-language.html}}
There are special additional protections in place for vulnerable populations,
such as minors, prisoners, and people with disabilities,
and these should be confirmed with relevant authorities if your research includes them.

IRB approval should be obtained well before any data is acquired.
IRBs may have infrequent meeting schedules
or require several rounds of review for an application to be approved.
If there are any deviations from an approved plan or expected adjustments,
report these as early as possible so that you can update or revise the protocol.
Particularly at universities, IRBs have the power to retroactively deny
the right to use data which was not acquired in accordance with an approved plan.
This is extremely rare, but shows the seriousness of these considerations
since the institution itself may face legal penalties if its IRB
is unable to enforce them. As always, as long as you work in good faith,
you should not have any issues complying with these regulations.

\subsection{Transmitting and storing data securely}

Secure data storage and transfer are ultimately your personal responsibility.\sidenote{
	\url{https://dimewiki.worldbank.org/Data_Security}}
There are several precautions needed to ensure that your data is safe.
First, all online and offline accounts
-- including personal accounts like computer logins and email --
need to be protected by strong and unique passwords.
There are several services that create and store these passwords for you,
and some provide utilities for sharing passwords with others
inside that secure environment.
However, password-protection alone is not sufficient,
because if the underlying data is obtained through a leak the information itself remains usable.
Datasets that include confidential information
\textit{must} therefore be \textbf{encrypted}\sidenote{
	\textbf{Encryption:} Methods which ensure that files are unreadable even if laptops are stolen, databases are hacked, or any other type of unauthorized access is obtained.
	\url{https://dimewiki.worldbank.org/Encryption}}
during data collection, storage, and transfer.\index{encryption}\index{data transfer}\index{data storage}

Most modern data collection software has features that,
if enabled, make secure transmission straightforward.\sidenote{
	\url{https://dimewiki.worldbank.org/Encryption\#Encryption\_in\_Transit}}
Many also have features that ensure data is encrypted when stored on their servers,
although this usually needs to be actively enabled and administered.\sidenote{
	\url{https://dimewiki.worldbank.org/Encryption\#Encryption\_at\_Rest}}
When files are properly encrypted,
the information they contain will be completely unreadable and unusable
even if they were to be intercepted my a malicious
``intruder'' or accidentally made public.
When the proper data security precautions are taken,
no one who is not listed on the IRB may have access to the decryption key.
This means that is it usually not
enough to rely service providers' on-the-fly encryption as they need to keep a copy
of the decryption key to make it automatic. When confidential data is stored on a local
computer it must always remain encrypted, and confidential data may never be sent unencrypted
over email, WhatsApp, or other chat services.

The easiest way to reduce the risk of leaking confidential information is to use it as rarely as possible.
It is often very simple to conduct planning and analytical work
using a subset of the data that does not include this type of information.
We encourage this approach, because it is easy.
However, when confidential data is absolutely necessary to a task,
such as implementing an intervention
or submitting survey data,
you must actively protect that information in transmission and storage.

There are plenty of options available to keep your data safe,
at different prices, from enterprise-grade solutions to free software.
It may be sufficient to hold identifying information in an encrypted service,
or you may need to encrypt information at the file level using a special tool.
(This is in contrast to using software or services with disk-level or service-level encryption.)
Data security is important not only for identifying, but all confidential information,
especially when a worst-case scenario could potentially lead to re-identifying subjects.
Extremely confidential information may be required to be held in a ``cold'' machine
which does not have internet access -- this is most often the case with
government records such as granular tax information.
What data security protocols you employ will depend on project needs and data sources,
but agreeing on a protocol from the start of a project will make your life easier.
Finally, having an end-of-life plan for data is essential:
you should always know how to transfer access and control to a new person if the team changes,
and what the expiry of the data and the planned deletion processes are.

\subsection{De-identifying data}

Most of the field research done in development involves human subjects.\sidenote{
	\url{https://dimewiki.worldbank.org/Human_Subjects_Approval}}
\index{human subjects}
As a researcher, you are asking people to trust you with personal information about themselves:
where they live, how rich they are, whether they have committed or been victims of crimes,
their names, their national identity numbers, and all sorts of other data.
PII data carries strict expectations about data storage and handling,
and it is the responsibility of the research team to satisfy these expectations.\sidenote{
	\url{https://dimewiki.worldbank.org/Research_Ethics}}
Your donor or employer will most likely require you to hold a certification from a source
such as Protecting Human Research Participants\sidenote{
	\url{https://phrptraining.com}}
or the CITI Program.\sidenote{
	\url{https://about.citiprogram.org/en/series/human-subjects-research-hsr}}

In general, though, you shouldn't need to handle PII data very often
once the data collection processes are completed.
You can take simple steps to avoid risks by minimizing the handling of PII.
First, only collect information that is strictly needed for the research.
Second, avoid the proliferation of copies of identified data.
There should never be more than one copy of the raw identified dataset in the project folder,
and it must always be encrypted.
Even within the research team,
access to PII data should be limited to team members who require it for specific analysis
(most analysis will not depend on PII).
Analysis that requires PII data is rare
and can be avoided by properly linking identifiers to research information
such as treatment statuses and weights, then removing identifiers.

Therefore, once data is securely collected and stored,
the first thing you will generally do is \textbf{de-identify} it,
that is, remove direct identifiers of the individuals in the dataset.\sidenote{
	\url{https://dimewiki.worldbank.org/De-identification}}
\index{de-identification}
Note, however, that it is in practice impossible to \textbf{anonymize} data.
There is always some statistical chance that an individual's identity
will be re-linked to the data collected about them
-- even if that data has had all directly identifying information removed --
by using some other data that becomes identifying when analyzed together.
For this reason, we recommend de-identification in two stages.
The \textbf{initial de-identification} process strips the data of direct identifiers
as early in the process as possible,
to create a working de-identified dataset that
can be shared \textit{within the research team} without the need for encryption.
This simplifies workflows.
The \textbf{final de-identification} process involves
making a decision about the trade-off between
risk of disclosure and utility of the data
before publicly releasing a dataset.\sidenote{
	\url{https://sdcpractice.readthedocs.io/en/latest/SDC\_intro.html\#need-for-sdc}}
We will provide more detail about the process and tools available
for initial and final de-identification in Chapters 6 and 7, respectively.
