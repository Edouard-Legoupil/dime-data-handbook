%----------------------------------------------------------------------------------------
%	PACKAGES AND OTHER DOCUMENT CONFIGURATIONS
%----------------------------------------------------------------------------------------

\documentclass{tufte-book} % Use the tufte-book class which in turn uses the tufte-common class
%Tufte docs: http://ftp.math.purdue.edu/mirrors/ctan.org/macros/latex/contrib/tufte-latex/sample-book.pdf
\usepackage{geometry}

  \geometry{
    % showframe,
   % paperwidth=8.5in,
    %paperheight=11in,
   % left=0.55in,
   % right=0.45in,
    %top=.5in,
    %bottom=.5in,
    %marginparsep=0.25in,
    %marginparwidth=1in,
   % includemp,
    %includehead,
    % The text width and height are calculated automatically.
  }

\hypersetup{colorlinks} % Comment this line if you don't wish to have colored links
\expandafter\def\expandafter\UrlBreaks\expandafter{\UrlBreaks  \do\-} %  Allow URLs to wrap on dash


\usepackage{microtype} % Improves character and word spacing

\usepackage{lipsum} % Inserts dummy text

\usepackage{booktabs} % Better horizontal rules in tables

\usepackage{graphicx} % Needed to insert images into the document
\graphicspath{{graphics/}} % Sets the default location of pictures
\setkeys{Gin}{width=\linewidth,totalheight=\textheight,keepaspectratio} % Improves figure scaling

\usepackage{upquote}
\usepackage{fancyvrb} % Allows customization of verbatim environments
%Fancyvrb docs: http://mirrors.ibiblio.org/CTAN/macros/latex/contrib/fancyvrb/doc/fancyvrb-doc.pdf
\fvset{fontsize=\small} % The font size of all verbatim text can be changed here

\renewcommand{\FancyVerbFormatLine}{\color{violet}}

\newcommand{\hangp}[1]{\makebox[0pt][r]{(}#1\makebox[0pt][l]{)}} % New command to create parentheses around text in tables which take up no horizontal space - this improves column spacing
\newcommand{\hangstar}{\makebox[0pt][l]{*}} % New command to create asterisks in tables which take up no horizontal space - this improves column spacing

\usepackage{xspace} % Used for printing a trailing space better than using a tilde (~) using the \xspace command

\newcommand{\monthyear}{\ifcase\month\or January\or February\or March\or April\or May\or June\or July\or August\or September\or October\or November\or December\fi\space\number\year} % A command to print the current month and year

\newcommand{\openepigraph}[2]{ % This block sets up a command for printing an epigraph with 2 arguments - the quote and the author
\begin{fullwidth}
\sffamily\large
\begin{doublespace}
\noindent\allcaps{#1}\\ % The quote
\noindent\allcaps{#2} % The author
\end{doublespace}
\end{fullwidth}
}

\newcommand{\blankpage}{\newpage\hbox{}\thispagestyle{empty}\newpage} % Command to insert a blank page

\usepackage{makeidx} % Used to generate the index
\makeindex % Generate the index which is printed at the end of the document

%So we can use option FloatBarrier, which is similar to [H] but is an
%alternative solition when the algorithm can't solce [H] as too many
%settings are going on. [H] seems to get stuck in infinite loop
%https://tex.stackexchange.com/questions/2275/keeping-tables-figures-close-to-where-they-are-mentioned
\usepackage{placeins}
\newcommand{\codeexample}[2]{
	\begin{figure*}[h]
		\VerbatimInput[
			framesep=3mm,
			frame=lines, % line above and below code section
			numbers=left, %Line number
			label= #1, %name of code section
			baselinestretch=0.75, %Use line space more similat to line space in code editors
		]{#2} %Write the relative file path and the name of the file to be included
	\end{figure*}
	\FloatBarrier
}

%----------------------------------------------------------------------------------------
%	BOOK META-INFORMATION
%----------------------------------------------------------------------------------------

\title{Data for \\ \noindent Development Impact: \\ \bigskip
\noindent The DIME Analytics \\ \noindent Resource Guide} % Title of the book

\author{Kristoffer Bj{\"a}rkefur \\ \noindent Lu{\'i}za Cardoso de Andrade \\ \noindent Benjamin Daniels \\ \noindent Maria Jones \\} % Author

\publisher{DIME Analytics} % Publisher

%----------------------------------------------------------------------------------------

\begin{document}

\frontmatter

%----------------------------------------------------------------------------------------
%	EPIGRAPH
%----------------------------------------------------------------------------------------

%----------------------------------------------------------------------------------------

\maketitle % Print the title page

%----------------------------------------------------------------------------------------
%	COPYRIGHT PAGE
%----------------------------------------------------------------------------------------

\newpage
\begin{fullwidth}
~\vfill
\thispagestyle{empty}
\setlength{\parindent}{0pt}
\setlength{\parskip}{\baselineskip}
Copyright \copyright\ \the\year\ \thanklessauthor

\par\smallcaps{Published by \thanklesspublisher}

\par\smallcaps{\url{http://worldbank.github.com/d4di}}

\par Released under a Creative Commons CC BY 2.0 license.

\url{https://creativecommons.org/licenses/by/2.0/}

\par\textit{First printing, \monthyear}
\end{fullwidth}

%----------------------------------------------------------------------------------------
%	Edition notes
%----------------------------------------------------------------------------------------

\cleardoublepage
\chapter*{Notes on this edition} % The asterisk leaves out this chapter from the table of contents

This is a draft peer review edition of
\textit{Data for Development Impact:
The DIME Analytics Resource Guide}.
This version of the book has been substantially revised
since the first release in June 2019
with feedback from readers and other experts.
It now contains most of the major content
that we hope to include in the finished version,
and we are in the process of making final additions
and polishing the materials to formally publish it.

This book is intended to remain a living product
that is written and maintained in the open.
The raw code and edit history are online at:
\url{https://github.com/worldbank/d4di}.
You can get a PDF copy at:
\url{https://worldbank.github.com/d4di}.
The website also includes the most updated instructions
for providing feedback, as well as
a log of errata and updates that have been made to the content.

\subsection{Feedback}

Whether you are a DIME team member or you work for the World Bank
or another organization or university,
we ask that you read the contents of this book carefully and critically.
We encourage feedback and corrections
so that we can improve the contents of the book
in future editions. Please visit
\url{https://worldbank.github.com/d4di/feedback} to
see different options on how to provide feedback.
You can also email us at \url{dimeanalytics@worldbank.org}
with input or comments, and we will be very thankful.
We hope you enjoy the book!


%----------------------------------------------------------------------------------------
%	Abbreviations
%----------------------------------------------------------------------------------------

\cleardoublepage
\chapter*{Abbreviations} % The asterisk leaves out this chapter from the table of contents

\noindent\textbf{2SLS} -- Two-Stage Least Squares

\noindent\textbf{AEA} -- American Economic Association

\noindent\textbf{CAPI} -- Computer-Assisted Personal Interviewing

\noindent\textbf{CI} -- Confidence Interval

\noindent\textbf{DEC} -- Development Economics

\noindent\textbf{DD or DiD} -- Differences-in-Differences

\noindent\textbf{DGP} -- Data-Generating Process

\noindent\textbf{DIME} -- Development Impact Evaluations

\noindent\textbf{FC} -- Field Coordinator

\noindent\textbf{FE} -- Fixed Effects

\noindent\textbf{HFC} -- High-Frequency Checks

\noindent\textbf{IRB} -- Instituional Review Board

\noindent\textbf{IV} -- Instrumental Variables

\noindent\textbf{MDE} -- Minimum Detectable Effect

\noindent\textbf{NGO} -- Non-Governmental Organization

\noindent\textbf{ODK} -- Open Data Kit

\noindent\textbf{OLS} -- Ordinary Least Squares

\noindent\textbf{OSF} -- Open Science Foundation

\noindent\textbf{PI} -- Principal Investigator

\noindent\textbf{PII} -- Personally-Identifying Information

\noindent\textbf{QA} -- Quality Assurance

\noindent\textbf{RA} -- Research Assistant

\noindent\textbf{RD} -- Regression Discontinuity

\noindent\textbf{RCT} -- Randomized Control Trial

\noindent\textbf{SSC} -- Statistical Software Components

\noindent\textbf{WBG} -- World Bank Group


%----------------------------------------------------------------------------------------

\tableofcontents % Print the table of contents

%----------------------------------------------------------------------------------------

% \listoffigures % Print a list of figures

%----------------------------------------------------------------------------------------

% \listoftables % Print a list of tables

%----------------------------------------------------------------------------------------
%	DEDICATION PAGE
%----------------------------------------------------------------------------------------

\cleardoublepage
~\vfill
\begin{doublespace}
\noindent\fontsize{18}{22}\selectfont\itshape
\nohyphenation
Dedicated to all the research assistants who have
wrangled data without being taught how,
hustled to get projects done on time,
wondered if they really should get their PhD after all,
and in doing so made this knowledge necessary and possible.
\end{doublespace}
\vfill
\vfill
