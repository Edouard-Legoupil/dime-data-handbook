%-----------------------------------------------------------------------------------------------

\begin{fullwidth}

In this chapter we will show how you can save a lot of time
and increase the quality of your research by planning your project's data requirements
based on your research design.
There are many published resources about 
the theories behind different research designs.
This chapter will instead focus on how the design 
impacts a project's data requirements.
We assume you have a working familiarity
with the research designs mentioned here.
If needed, you can reference Appendix XYZ,
where you will find more details 
and specific references for each design.

Planning data requirements is more than just listing the data your project will need. 
It requires understanding how to structure the project's data to best answer the research questions, 
and creating the tools to share this understanding across your team.
The first step of this process is to create master datasets 
for every unit of observation in your study,
and to create a data map, 
detailing how the master datasets are interlinked.
These master datasets will serve three key functions.
First, they will list all of the units who are eligible for the study,
and enable you to map data collected or received from the field to the research design.
Second, in designs where your team has direct control over interventions or other field work,
they will allow you to complete sampling and treatment assignment tasks
before those tasks are implemented in the field.
Finally, they will function as a single unambiguous location where all information 
related to the implementation and validity of your research is stored,
as well as all information needed to correctly identify any observation in any of your project's datasets.

Almost all research designs rely on a random component 
for the results of the research to be a valid interpretation of the real world.
This includes both how a sample is representative to the population studied,
and how the counterfactual observations in experimental design are statistically indistinguishable
from the treatment observations.
The second half of this chapter covers random sampling and assignment
and the necessary practices to ensure that 
these and other random processes are reproducible.
It concludes with a discussion of power calculations and randomization inference,
and how both are important tools to make optimal choices when planning data work.


\end{fullwidth}

%-----------------------------------------------------------------------------------------------

\section{Translating research design to master data}

In most projects, different data sources are needed to answer the research question.
These could be multiple survey rounds,
data acquired from different  partners (e.g. administrative data,
web scraping, implementation monitoring, etc)
or complex combinations of these.
For example, you may have different \textbf{units of observation}\sidenote{
	The \textbf{unit of observation} is the unit at or for which data is collected. See
	\url{https://dimewiki.worldbank.org/Unit\_of\_Observation}},
and their level may vary from round to round.

However your study is structured, you need to know how to link data from all sources
and analyze the relationships between the units that appear in them
to answer all your research questions.
You might think that you are able to keep all of the relevant details in your head,
but your whole research team is unlikely to have the same understanding,
at all times, of all the datasets required.
The only way to make sure that the full team shares the same understanding
is to create \textit{master datasets} and a \textit{data map}.


\subsection{Creating master datasets and a data map}

A \textbf{master dataset}\sidenote{
	\url{https://dimewiki.worldbank.org/Master\_Data\_Set}}
\index{master datasets}
details all project-wide time-invariant information
about all observations encountered,
as well as their relationship to the research design,
typically summarized by sampling and treatment status.
Having a plan for how to get the raw data into analysis shape 
before you acquire it,
and making sure that the full research team knows where to find this information,
will save you a ton of time during the course of the project
and increase the quality of your research.

You should create a master dataset
for each unit of observation
relevant to the research.
This includes all units used for significant research activity,
like data collection or data analysis.
Therefore, any unit
that will be used in sampling or treatment assignment,
must have a master dataset,
and that master dataset -- not field data --
should be used when sampling or assigning treatment.

You also need to record how all datasets for each unit of observation
will link or merge with each other as needed.
This linking scheme is called a \textbf{data map}\sidenote{
	\url{https://dimewiki.worldbank.org/data\_map} (TO BE CREATED)}.
\index{data maps} 
A data map is more than just a list of datasets.
Its purpose is to specify the characteristics and linkages of those datasets.
To link properly, the master datasets must include fully unique ID variables.\sidenote{
	\url{https://dimewiki.worldbank.org/ID\_Variable\_Properties}}
The master datasets should indicate whether datasets should be merge one-to-one,
for example, merging baseline data and endline data that use the same unit of observation,
or whether two datasets should be merged many-to-one, 
for example, school administrative data merged with student data.
Your data map must indicate which ID variables can be used and how to merge datasets.
It is common that administrative data use IDs 
that are different than the project IDs, 
and the linkage between those should be clear from your master dataset.

The data map should also include metadata about the handling of all information.
These characteristics may be updated as the project progresses.
For example, you will need to note the original source of each dataset,
as well as the project folder where 
the raw original data and codebooks are stored
and where the back-ups for the each raw dataset are stored.

Some of the characteristics in your master datasets and your data map
should be filled in during the planning stage,
but both of them should be active resources 
that are updated as your project evolves.
Finally, your master data should not include any unlabeled missing values. 
If the information is missing for one unit, 
then the reason should always be indicated with a code.
An example for such reason could be that a unit was not included in the treatment assignment
as it was not sampled in the first place,
or was not located in the data collection at a given round.

\subsection{Defining study comparisons using master data}

Your research design will determine what statistical comparisons you need
to estimate in your analytical work.
The research designs discussed here compare a group that received
some kind of \textbf{treatment}\sidenote{
	\textbf{Treatment:} The general word for the evaluated intervention or event.
	This includes being offered training or cash transfer from a program, experiencing a natural disaster etc.}
against a counterfactual control group.\sidenote{
	\textbf{Counterfactual:} A statistical description of what would have happened
	to specific individuals in an alternative scenario,
	for example, a different treatment assignment outcome.}
\index{counterfactual}
The key assumption is that each
person, facility, or village (or whatever the unit of intervention is)
had two possible states: their outcome if they did receive the treatment
and their outcome if they did not receive that treatment.
The average impact of the treatment is defined as
the difference between these two states averaged over all units.

However, we can never observe the same unit
in both the treated and untreated state simultaneously,
so we cannot calculate these differences directly.
Instead, the treatment group is compared to a control group
that is statistically indistinguishable,
which makes the average impact of the treatment
mathematically equivalent to the difference in averages between the groups.
Statistical similarity is often defined
as \textbf{balance} between two or more groups.	
This test is so common,	
DIME Analytics created the Stata command \texttt{iebaltab}\sidenote{	
	\url{https://dimewiki.worldbank.org/iebaltab}}	
to generates tables of balance checks	
as part of the package \texttt{ietoolkit}.	
Each research design has a different method	for identifying the statistically-similar control group. 
for how the statistically similar control group is identified.

The rest of this section covers how data requirements differ
between different research designs.
What does not differ, however, is that the authoritative source
for which units are in the treatment group and which are in the control group
should always be one or several variables in your master dataset. 
You will often have to merge that data to other datasets, 
but that is an easy task if you created a data map.

%%%%% Experimental design

In \textbf{experimental research designs}, 
such as \textbf{randomized control trials (RCTs)},\sidenote{
	\url{https://dimewiki.worldbank.org/Randomized\_Control\_Trials}}
\index{randomized control trials} \index{experimental research designs}
the research team determines which members
of the studied population will receive the treatment.
This is typically done by a randomized process
in which a subset of the eligible population
is randomly assigned to receive the treatment
(see later in this chapter for how to implement this).
The intuition is that if everyone in the eligible population
is assigned at random to either the treatment or control group,
then the two groups will, on average, be statistically indistinguishable.
The treatment will therefore not be correlated with anything
but the impact of that treatment.\cite{duflo2007using}
The randomized assignment should be done using the master data,
and the result should be saved there before being merged to other datasets.

%%%%% Quasi-experimental design

\textbf{Quasi-experimental research designs},\sidenote{
	\url{https://dimewiki.worldbank.org/Quasi-Experimental\_Methods}}
\index{quasi-experimental research designs}
by contrast, are based on events not controlled by the research team.
Instead, they rely on ``experiments of nature'',
in which natural variation in treatment can be argued to approximate randomization.
You must have a way to measure this natural variation,
and how the variation is categorized as outcomes of a naturally randomized assignment
must be documented in your master dataset.
Unlike carefully planned experimental designs,
quasi-experimental designs typically require the luck
of having access to data collected at the right times and places
to exploit events that occurred in the past.
Therefore, these methods often use either secondary data,
including administrative data or other classes of routinely-collected information,
and it is important that your data map documents 
how that data is merged to any other data.


%%%%% Regression discontinuity

\textbf{Regression discontinuity (RD)}\sidenote{
	\url{https://dimewiki.worldbank.org/Regression\_Discontinuity}}
\index{regression discontinuity}
designs exploit sharp breaks or limits
in policy designs to separate a single group of potentially eligible recipients
into comparable groups of individuals who do and do not receive a treatment.
Common examples are test score thresholds and income thresholds,
where the individuals on one side of some threshold receive
a treatment but those on the other side do not.\sidenote{
	\url{https://blogs.worldbank.org/impactevaluations/regression-discontinuity-porn}}
The intuition is that, on average,
individuals immediately on one side of the threshold
are statistically indistinguishable from the individuals on the other side,
and the only difference is receiving the treatment.
In your data you need an unambiguous way
to define which observations were above or below the cutoff.
The cutoff determinant, or running variable,
is often a continuous variable 
that is used to divide the sample into two or more groups. 
Both the running variable and a categorical cutoff variable,
should be saved in your master dataset.

%%%%% IV regression

\textbf{Instrumental variables (IV)}\sidenote{
	\url{https://dimewiki.worldbank.org/Instrumental\_Variables}}
\index{instrumental variables}
designs, unlike the previous approaches,
assume that the treatment effect is not directly identifiable.
Similar to RD designs,
IV designs focus on a subset of the variation in treatment take-up.
Where RD designs use a \textit{sharp} or binary cutoff,
IV designs are \textit{fuzzy}, meaning that the input does not completely determine
the treatment status, but instead influence the \textit{probability of treatment}.
You will need variables in your data
that can be used to estimate the probability of treatment for each unit.
These variables are called \textbf{instruments}.
In IV designs, instead of the ordinary regression estimator,
a special version called two-stage least squares (2SLS) is used
to estimate the impact of the treatment.
Stata has a built-in command called \texttt{ivregress},
and another popular implementation is the user-written command \texttt{ivreg2}.


%%%%% Matching

\textbf{Matching}\sidenote{
	\url{https://dimewiki.worldbank.org/Matching}}
methods use observable characteristics to construct
sets of treatment and control units
where the observations in each set 
are as similar as possible. \index{matching}
These sets can either consist of exactly one treatment and one control observation (one-to-one),
a set of observations where
both groups have more than one observation represented (many-to-many),
or where only one group has more than one observation included (one-to-many).
By now you can probably guess that 
the result of the matching needs to be saved in the master dataset.
This is best done by assigning an ID to each matching set, 
and create a variable in the master dataset 
with the ID for the set each unit belongs to.
The matching can even be done before the randomized assignment,
so that treatment can be randomized within each matching set.
This is a type of experimental design.
Furthermore, if no control observations were identified before the treatment,
then matching can be used to ex-post identify a control group.
Many matching algorithms can only match on a single variable,
so you first have to turn many variables into a single variable
by using an index or a propensity score.\sidenote{
	\url{https://dimewiki.worldbank.org/Propensity\_Score\_Matching}}
DIME Analytics developed a command to match observations 
based on this single continuous variable: \texttt{iematch}\sidenote{
	\url{https://dimewiki.worldbank.org/iematch}},
part of the \texttt{ietoolkit} package.

%-----------------------------------------------------------------------------------------------
\subsection{Structuring complex data}

Your data map and master dataset requirements also depends on
whether you are using data from one time period or several.
A study that observes data in only one time period is called
a \textbf{cross-sectional study}.
\index{cross-sectional data}
This type of data is relatively easy to collect and handle because
you do not need to track individuals across time.
Instead, the challenge in a cross-sectional study is to
show that the control group is indeed a valid counterfactual to the treatment group.

Observations over multiple time periods, 
referred to as \textbf{longitudinal data},
\index{longitudinal data}
can consist of either \textbf{repeated cross-sections}
\index{repeated cross-sectional data}
or \textbf{panel data}.
\index{panel data} 
In repeated cross-sections,
each successive round of data collection uses a new random sample
of observations from the treatment and control groups,
but in a panel data study the same observations are tracked and included each round.
If you are using panel data,
then your data map must document how the different rounds will be merged or appended,
and your master dataset will be the authoritative source of which ID that will be used. 

You must keep track of the \textit{attrition rate} in panel data,
which is the share of observations not observed in follow-up data.
It is common that the observations not possible to track
can be correlated with the outcome you study.
For example, poorer households may live in more informal dwellings,
patients with worse health conditions might not survive to follow-up,
and so on.
If this is the case, then your results might only be an effect of your remaining sample
being a subset of the original sample that were better or worse off from the beginning.
You should have a variable in your master dataset that indicates attrition.
A balance check using the attrition variable can provide insights
as to whether the lost observations were systematically different
compared to the rest of the sample,
and there are a variety of methods for estimating treatment effects
with selective attrition.

%-----------------------------------------------------------------------------------------------
\subsection{Monitoring data}

For any study with an ex-ante design, 
\textbf{monitoring data}\sidenote{\url{
		https://dimewiki.worldbank.org/Monitoring\_Data}}
is very important for understanding whether field realities match the research design.
\index{monitoring data}
Monitoring data is used to understand if the
assumptions made during the research design corresponds to what is true in reality.
The most typical example is to make sure that, in an experimental design,
the treatment was implemented according to your treatment assignment.

Treatment implementation is often carried out by partners,
and field realities may be more complex realities than foreseen during research design.
Furthermore, the field staff of your partner organization,
might not be aware that their actions are the implementation of your research.
Therefore, you must acquire monitoring data that tells you how well the treatment assignment in the field
corresponds to your intended treatment assignment,
for nearly all research designs.

Another example of a research design where monitoring data is important
are regression discontinuity (RD) designs
where the discontinuity is a cutoff for eligibility of the treatment. 
For example, 
let's say your project studies the impact of a program for students that scored under 50\% at a test.
We might have the exact results of the tests for all students, 
and therefore know who should be offered the program, 
however that is not the same as knowing who attended the program. 
A teacher might offer the program to someone that scored 51\% at the test,
and someone that scored 49\% at the might decline to participate in the program.
We should not pass judgment on a teacher that offers a program to a student
they think can benefit from it, 
but if that was not inline with our research assumptions,
then we need to understand how common that was.
Otherwise the result of our research will not be helpful
in evaluating the program.

Monitoring data is particularly prone to errors 
relating to merging with other data set. 
It is too late to think about how monitoring data will be merged
with the rest of the data, after the monitoring data is already collected.
If you send a team to simply record the name of all people attending a training,
then potentially different spellings of names 
-- especially when names have to be transcribed from other alphabets to the Latin alphabet --
will be the biggest source of error in your monitoring activity.
The time to discuss and document how monitoring data will be merged with precision
to the rest of your data, is when you are creating your data map.
The solution is often very simple, it is just a matter of solving it before it is too late.
An example of a solution is to provide the monitoring teams 
with lists of the people's names spelled the same way as in your master dataset. 
Include both the people you expect to attend the training,
and people that you do not expect to attend,
but do not tell the monitoring team which person you expect to attend the training.

Additionally, instruct the monitoring teams to record the name of any
participant not in your lists. 
Add those names to your master datasets, 
as the most complete information possible will help you 
if you at any point in your project end up without an ID to merge on,
and will have to compare names when merging data.
Finally, while it is always better to monitor all activities,
it might be to costly. 
In those cases you can sample a smaller number of critical activities and monitor them.
This will not be detailed enough to be used as a control in your analysis,
but it will still give an idea of the validity of your research design assumptions.


%-----------------------------------------------------------------------------------------------
%-----------------------------------------------------------------------------------------------
\section{Implementing random sampling and treatment assignments}

Random sampling and treatment assignment are two core elements of research design.
In experimental methods, random sampling and treatment assignment directly determine
the set of individuals who are going to be observed
and what their status will be for the purpose of effect estimation.
In quasi-experimental methods, random sampling determines what populations the study
will be able to make meaningful inferences about,
and random treatment assignment creates counterfactuals.
Randomization\sidenote{
	\textbf{Randomization} is often used interchangeably to mean random treatment assignment.
	In this book however, \textit{randomization} will only be used to describe the process of generating
	a sequence of unrelated numbers, i.e. a random process. 
	\textit{Randomization} will never be used to mean the process of assigning units in treatment and control groups,
	that will always be called \textit{random treatment assignment},
	or a derivative thereof.} 
is used to ensure that a sample is representative and
that any treatment and control groups are statistically indistinguishable
after treatment assignment.

Randomization in statistical software is non-trivial
and its mechanics are unintuitive for the human brain.
The principles of randomization we will outline
apply not just to random sampling and random assignment,
but to all statistical computing processes that have random components
such as simulations and bootstrapping.
Furthermore, all random processes introduce statistical noise
or uncertainty into estimates of effect sizes.
Choosing one random sample from all the possibilities produces some probability of
choosing a group of units that are not, in fact, representative.
Similarly, choosing one random assignment produces some probability of
creating groups that are not good counterfactuals.
\textit{Power calculation} and \textit{randomization inference}
are the main methods by which these probabilities of error are assessed.
These analyses are particularly important in the initial phases of development research --
typically conducted before any actual field work occurs --
and have implications for feasibility, planning, and budgeting.

%-----------------------------------------------------------------------------------------------
\subsection{Randomizing sampling and treatment assignment}

% sampling universe: the master dataset
\textbf{Sampling} is the process of randomly selecting observations
from a list of individuals to create a representative or statistically similar sub-sample.\index{sampling}
This process can be used, for example, to select a subset from all eligible units
to be included in data collection when the cost of collecting data on everyone is prohibitive.\sidenote{
	\url{https://dimewiki.worldbank.org/Sample\_Size\_and\_Power\_Calculations}}
But it can also be used to select a sub-sample of your observations to test a computationally heavy process 
before running it on the full data.
\textbf{Randomized treatment assignment} is the process of assigning observations to different treatment arms.
This process is central to experimental research design.
Most of the code processes used for randomized assignment are the same as those used for sampling,
since it is also a process of randomly splitting a list of observations into groups.
Where sampling determines whether a particular individual
will be observed at all in the course of data collection,
randomized assignment determines if each individual will be observed
as a treatment observation or used as a counterfactual.
The list of units to sample or assign from may be called a \textbf{sampling universe},
a \textbf{listing frame}, or something similar.
This list should always be your \textbf{master dataset} when possible.
The rare exceptions when master datasets cannot be used is when sampling must be done in real time --
for example, randomly sampling patients as they arrive at a health facility.

% implement uniform-probability random sampling
The simplest form of sampling is 
\textbf{uniform-probability random sampling}.
This means that every eligible observation in the master dataset
has an equal probability of being selected.
The most explicit method of implementing this process
is to assign random numbers to all your potential observations,
order them by the number they are assigned,
and mark as ``sampled'' those with the lowest numbers, up to the desired proportion.
There are a number of shortcuts to doing this process,
but they all use this method as the starting point,
so you should become familiar with exactly how it works.
The do-file below provides an example of
how to implement uniform-probability sampling in practice.
This code uses a Stata built-in dataset and is fully reproducible
(more on reproducible randomization in next section),
so anyone that runs this code in any version of Stata later than 13.1
(the version set in this code)
will get the exact same, but still random, results.

\codeexample{simple-uniform-probability-sampling.do}{./code/simple-uniform-probability-sampling.do}

Sampling typically has only two possible outcomes: observed and unobserved.
Similarly, a simple randomized assignment has two outcomes: treatment and control,
and the logic in the code would be identical to the sampling code example.
However, randomized assignment often involves multiple treatment arms
which each represent different varieties of treatments to be delivered;
in some cases, multiple treatment arms are intended to overlap in the same sample.
Complexity can therefore grow very quickly in randomized assignment
and it is doubly important to fully understand the conceptual process
that is described in the experimental design,
and fill in any gaps before implementing it in code.
The do-file below provides an example of how to implement
a randomized assignment with multiple treatment arms.

\codeexample{simple-multi-arm-randomization.do}{./code/simple-multi-arm-randomization.do}

%-----------------------------------------------------------------------------------------------

\subsection{Programming reproducible random processes}

% what it means for randomization to be reproducible
For statistical programming to be reproducible,
you must be able to re-obtain its exact outputs in the future.\cite{orozco2018make}
We will focus on what you need to do to produce
truly random results for your project,
to ensure you can get those results again.
This takes a combination of strict rules, solid understanding, and careful programming.
This section introduces strict rules:
these are non-negotiable (but thankfully simple).
Stata, like most statistical software, uses a \textbf{pseudo-random number generator}
which, in ordinary research use, 
produces sequences of number that are as good as random.\sidenote{
	\url{https://dimewiki.worldbank.org/Randomization\_in\_Stata}}
However, for \textit{reproducible} randomization, we need two additional properties:
we need to be able to fix the sequence of numbers generated and
we need to ensure that the first number is independently randomized.
In Stata, this is accomplished through three concepts:
\textbf{versioning}, \textbf{sorting}, and \textbf{seeding}.
We again use Stata in our examples,
but the same principles translate to all other programming languages.

% rule 1: versioning
\textbf{Versioning} means using the same version of the software each time you run the random process.
If anything is different, the underlying list of random numbers may have changed,
and it may be impossible to recover the original result.
In Stata, the \texttt{version} command ensures that the list of random numbers is fixed.\sidenote{
	At the time of writing, we recommend using \texttt{version 13.1} for backward compatibility;
	the algorithm used to create this list of random numbers was changed after Stata 14 but the improvements do not matter in practice.}
The \texttt{ieboilstart} command in \texttt{ietoolkit} provides functionality to support this requirement.\sidenote{
	\url{https://dimewiki.worldbank.org/ieboilstart}}
We recommend you use \texttt{ieboilstart} at the beginning of your master do-file.\sidenote{
	\url{https://dimewiki.worldbank.org/Master\_Do-files}}
However, testing your do-files without running them
via the master do-file may produce different results,
since Stata's \texttt{version} setting expires after each time you run your do-files.

% rule 2: sorting
\textbf{Sorting} means that the actual data that the random process is run on is fixed.
Because random numbers are assigned to each observation row-by-row starting from
the top row,
changing their order will change the result of the process.
In Stata, the only way to guarantee a unique sorting order is to use
\texttt{isid [id\_variable], sort}.
(The \texttt{sort, stable} command is insufficient.)
Since the exact order must be unchanged,
the underlying data itself must be unchanged as well between runs.
This means that if you expect the number of observations to change
(for example to increase during ongoing data collection),
your randomization will not be reproducible unless you split your data up into
smaller fixed datasets where the number of observations does not change.
You can combine all
those smaller datasets after your randomization.


% rule 3: seeding
\textbf{Seeding} means manually setting the start point in the list of random numbers.
A seed is just a single number that specifies one of the possible start points.
It should be at least six digits long and you should use exactly
one unique, different, and randomly created seed per randomization process.\sidenote{You
	can draw a uniformly distributed six-digit seed randomly by visiting \url{https://bit.ly/stata-random}.
	(This link is a just shortcut to request such a random number on \url{https://www.random.org}.)
	There are many more seeds possible but this is a large enough set for most purposes.}
In Stata, \texttt{set seed [seed]} will set the generator
to the start point identified by the seed.
In R, the \texttt{set.seed} function does the same.
To be clear: you should not set a single seed once in the master do-file,
but instead you should set a new seed in code right before each random process.
The most important thing is that each of these seeds is truly random,
so do not use shortcuts such as the current date or a seed you have used before.
You should also describe in your code how the seed was selected.

% testing randomization reproducibility
Other commands may induce randomness in the data,
change the sorting order,
or alter the place of the random generator without you realizing it,
so carefully confirm exactly how your code runs before finalizing it.
To confirm that a randomization has worked correctly before finalizing its results,
save the outputs of the process in a temporary location,
re-run the code, and use \texttt{cf} or \texttt{datasignature} to ensure
nothing has changed. It is also advisable to let someone else reproduce your
randomization results on their machine to remove any doubt that your results
are reproducible.
Once the result of a randomization is used in the field,
there is no way to correct any mistakes.

\codeexample{reproducible-randomization.do}{./code/reproducible-randomization.do}

Some types of experimental designs require
that randomized assignment results be revealed in the field.
It is possible to do this using survey software or live events, such as a live lottery.
These methods typically do not leave a record of the randomization,
and as such are never reproducible. 
However, you can often run your randomization in advance 
even when you do not have list of eligible units in advance.
Let's say you want to, at various health facilities, 
randomly select a sub-sample of patients as they arrive.
You can then have a pre-generated list 
with a random order of ''in sample'' and ''not in sample''.
Your field staff would then go through this list in order
and cross off one randomized result as it is used for a patient.

This is especially beneficial if you are implementing a more complex randomization,
for example, sample 10\% of the patients, show a video for 50\% of the sample, 
and ask a longer version of the questionnaire to 20\% of both 
the group of patients that watch the video and those that did not.
The real time randomization is much more likely to be implemented correctly,
if your field staff simply can follow a list with the randomized categories
where you are in control fo the pre-determined proportions and the random order.
This way, you can also control with precision,
how these categories are evenly distributed across all health facilities.

Finally, if this real-time randomization implementation is done using survey software,
then the pre-generated list of randomized categories can be preloaded
into the questionnaire.
Then the field team can follow a list of respondent IDs 
that are randomized into the appropriate categories,
and the survey software can show a video and control which version of the questionnaire is asked.
This way, you reduce the risk of errors in field randomization.


%-----------------------------------------------------------------------------------------------

\subsection{Clustering or stratifying a random sample or assignment}

% the cases discussed so far are the most simple, but not the most common
For a variety of reasons, random sampling and random treatment assignment
are rarely as straightforward as a uniform-probability draw.
The most common variants are \textbf{clustering} and \textbf{stratification}.\cite{athey2017econometrics}
\textbf{Clustering} occurs when your unit of analysis is different
than the unit of sampling or treatment assignment.\sidenote{
	\url{https://dimewiki.worldbank.org/Unit\_of\_Observation}}
For example, a policy may be implemented at the village level,
or you may only be able to send enumerators to a limited number of villages,
or the outcomes of interest for the study are measured at the household level.\sidenote{
	\url{https://dimewiki.worldbank.org/Multi-stage\_(Cluster)\_Sampling}}
\index{clustered randomization}
The groups in which units are assigned to treatment are called clusters.
% what is stratification
\textbf{Stratification} breaks the full set of observations into subgroups
before performing randomized assignment within each subgroup.\sidenote{
	\url{https://dimewiki.worldbank.org/Stratified\_Random\_Sample}}
\index{stratification}
The subgroups are called \textbf{strata}.
This ensures that members of every subgroup
are included in all groups of the randomized assignment process,
or that members of all groups are observed in the sample.
Without stratification, it is possible that the randomization
would put all the members of a subgroup into just one of the treatment arms,
or fail to select any of them into the sample.
For both clustering and stratification,
implementation is nearly identical in both sampling and randomized assignment.

% How to implement randomization with clusters
Clustering is procedurally straightforward in Stata,
although it typically needs to be performed manually.
To cluster a sampling or randomized assignment,
randomize on the master dataset for the unit of observation of the cluster,
and then merge the results to the master dataset for the unit of analysis.
This is a many-to-one merge and your data map should document
how those datasets can be merged correctly.
If you do not have a master dataset yet for the unit of observation of the cluster,
then you first have to create that and update your data map accordingly.
When sampling or randomized assignment is conducted using clusters,
the clustering variable should be clearly identified in the master dataset
for the unit of analysis
since it will need to be used in subsequent statistical analysis.
Namely, standard errors for these types of designs must be clustered
at the level at which the randomization was clustered.\sidenote{
	\url{https://blogs.worldbank.org/impactevaluations/when-should-you-cluster-standard-errors-new-wisdom-econometrics-oracle}}
This accounts for the design covariance within the cluster --
the information that if one individual was observed or treated there,
the other members of the clustering group were as well.

% using randtreat for stratified randomized assignment
By contrast, implementing stratified designs in statistical software is prone to error.
Even for a typical multi-armed design, the basic method of randomly ordering the observations
will often create very skewed assignments.\sidenote{\url
	{https://blogs.worldbank.org/impactevaluations/tools-of-the-trade-doing-stratified-randomization-with-uneven-numbers-in-some-strata}}
The user-written \texttt{randtreat} command properly implements stratification.\cite{carril2017dealing}
The options and outputs (including messages) from the command should be carefully reviewed
so that you understand exactly what has been implemented.
Notably, it is extremely hard to target precise numbers of observations
in stratified designs, because exact allocations are rarely round fractions
and the process of assigning the leftover ``misfit'' observations
imposes an additional layer of randomization above the specified division.
Strata variables must also be included in your master datasets.

%-----------------------------------------------------------------------------------------------

\subsection{Doing power calculations for research design}

% sampling error, randomization noise and the need for power calcs
Random sampling and treatment assignment are noisy processes:
it is impossible to predict the result in advance.
By design, we know that the exact choice of sample or treatment
will be uncorrelated with our key outcomes,
but this lack of correlation is only true ``in expectation'' --
that is, across a large number of randomizations.
In any \textit{particular} randomization,
the correlation between the sampling or randomized assignments and the outcome variable
is guaranteed to be \textit{nonzero}:
this is called the \textbf{in-sample} or \textbf{finite-sample correlation}.

Since we know that the true correlation
(over the ``population'' of potential samples or randomized assignments)
is zero, we think of the observed correlation as an \textbf{error}.
In sampling, we call this the \textbf{sampling error},
and it is defined as the difference between a true population parameter
and the observed mean due to chance selection of units.\sidenote{
	\url{https://economistjourney.blogspot.com/2018/06/what-is-sampling-noise.html}}
In randomized assignment, we call this the \textbf{randomization noise},
and define it as the difference between a true treatment effect
and the estimated effect due to placing units in groups.
The intuition for both measures is that from any group,
you can find some possible subsets that have higher-than-average values of some measure;
similarly, you can find some that have lower-than-average values.
Your random sample or treatment assignment will fall in one of these categories,
and we need to assess the likelihood and magnitude of this occurrence.\sidenote{
	\url{https://davegiles.blogspot.com/2019/04/what-is-permutation-test.html}}
\textit{Power calculation} and \textit{randomization inference} are the two key tools to doing so.

% why to do power calculations
\textbf{Power calculations} report the likelihood that your experimental design
\index{power calculations}
will be able to detect the treatment effects you are interested in
given these sources of noise.\sidenote{
	\url{https://dimewiki.worldbank.org/Sample\_Size\_and\_Power\_Calculations}}
This measure of \textbf{power} can be described in various different ways,
each of which has different practical uses.\sidenote{
	\url{https://www.stat.columbia.edu/~gelman/stuff\_for\_blog/chap20.pdf}}
The purpose of power calculations is to identify where the strengths and weaknesses
of your design are located, so you know the relative tradeoffs you will face
by changing your randomization scheme for the final design.

The \textbf{minimum detectable effect (MDE)}\sidenote{
	\url{https://dimewiki.worldbank.org/Minimum\_Detectable\_Effect}}
is the smallest true effect that a given research design can reliably detect.
This is useful as a check on whether a study is worthwhile.
If, in your field, a ``large'' effect is just a few percentage points
or a small fraction of a standard deviation,
then it is nonsensical to run a study whose MDE is much larger than that.
This is because, given the sample size and variation in the population,
the effect needs to be much larger to possibly be statistically detected,
so such a study would never be able to say anything about the effect size that is practically relevant.
Conversely, the \textbf{minimum sample size} pre-specifies expected effect sizes
and tells you how large a study's sample would need to be to detect that effect,
which can tell you what resources you would need 
to implement a useful study.

% what is randomization inference
\textbf{Randomization inference}\sidenote{
	\url{https://dimewiki.worldbank.org/Randomization\_Inference}} 
is used to analyze the likelihood
\index{randomization inference}
that the randomized assignment process, by chance,
would have created a false treatment effect as large as the one you observed.
Randomization inference is a generalization of placebo tests,
because it considers what the estimated results would have been
from a randomized assignment that did not in fact happen in reality.
Randomization inference is particularly important
in quasi-experimental designs and in small samples,
where the number of possible \textit{randomizations} is itself small.
Randomization inference can therefore be used proactively during experimental design,
to examine the potential spurious treatment effects your exact design is able to produce.
If there is significant heaping at particular result levels,
or if results seem to depend dramatically on the outcomes for a small number of individuals,
randomization inference will flag those issues before the experiment is fielded
and allow adjustments to the design to be made.
