\begin{fullwidth}
Welcome to Data for Development Impact.
This book is intended to serve as a resource guide
for people who collect or use data for development research.
In particular, the book is intended to guide the reader
through the process of research using primary survey data,
from research design to fieldwork to data management to analysis.
This book will not teach you econometrics or epidemiology or agribusiness.
This book will not teach you how to design an impact evaluation.
This book will not teach you how to do data analysis, or how to code.
There are lots of really good resources out there for all of these things,
and they are much better than what we would be able to squeeze into this book.

What this book will teach you is how to think about quantitative data,
keeping in mind that you are not going to be the only person
collecting it, using it, or looking back on it.
We hope to provide you two key tools by the time you finish this book.
First, we want you to form a mental model of data collection as a ``social process'',
in which many people need to have the same idea about what is to be done, and when and where and by whom,
so that they can collaborate effectively on large, long-term projects.
Second, we want to provide a map of concrete resources for supporting these processes in practice.
As research teams and timespans have grown dramatically over the last decade,
it has become inefficient for everyone to have their own personal style
dictating how they use different functions, how they store data, and how they write code.
\end{fullwidth}

%------------------------------------------------

\section{Doing credible research at scale}

The team responsible for this book is known as \textbf{DIME Analytics}.\sidenote{
\url{http://www.worldbank.org/en/research/dime/data-and-analytics}}
The DIME Analytics team works within the \textbf{Development Impact Evaluation  (DIME)} group \sidenote{
\url{http://www.worldbank.org/en/research/dime}}
at the World Bank's \textbf{Development Economics group (DEC)}.\sidenote{
\url{http://www.worldbank.org/en/research/}}
After years of supporting hundreds of projects and staff in total,
DIME Analytics has built up a set of ideas, practices, and software tools
that support all of the research projects operated at DIME.

In the time that we have been working in the development field,
the proportion of projects that rely on \textbf{primary data} has soared.\cite{angrist2017economic}
Today, the scope and scale of those projects continue to expand rapidly,
meaning that more and more people are working on the same data over longer and longer timeframes.
This is because, while administrative datasets
and \textbf{big data} have important uses,
primary data\sidenote{\textbf{Primary data:} data collected from first-hand sources.}
is critical for answering specific research questions.\cite{levitt2009field}
As the ambition of development researchers grows, so too has the complexity of the data
on which they rely to make policy-relevant research conclusions from \textbf{field experiments}.\sidenote{\textbf{Field experiment:} experimental intervention in the real world, rather than in a laboratory.}
Unfortunately, this seems to have happened (so far) without the creation of
guidelines for practitioners to handle data efficiently and transparently,
which could provide relevant and objective quality markers for research consumers.

One important lesson we have learned from doing field work over this time is that
the most overlooked parts of primary data work are reproducibility and collaboration.
You will be working with people
who have very different skillsets and mindsets than you,
from and in a variety of cultures and contexts, and you will have to adopt workflows
that everyone can agree upon, and that save time and hassle on every project.
This is not easy. But for some reason, the people who agreed to write this book enjoy doing it.
(In part this is because it has saved ourselves a lot of time and effort.)
As we have worked with more and more DIME recruits
we have realized that we barely have the time to give everyone the attention they deserve.
This book itself is therefore intended to be a vehicle to document our experiences and share it with with future DIME team members.

The \textbf{DIME Wiki} is one of our flagship resources for project teams,
as a free online collection of our resources and best practices.\sidenote{\url{http://dimewiki.worldbank.org/}}
This book therefore complements the detailed-but-unstructured DIME Wiki
with a guided tour of the major tasks that make up primary data collection.\sidenote{Like this: \url{https://dimewiki.worldbank.org/wiki/Primary_Data_Collection}}
We will not give a lot of highly specific details in this text,
but we will point you to where they can be found,
and give you a sense of what you need to find next.
Each chapter will focus on one task,
and give a primarily narrative account of:
what you will be doing; where in the workflow this task falls;
when it should be done; who you will be working with;
why this task is important; and how to implement the most basic form of the task.

We will use broad terminology throughout this book
to refer to different team members:
\textbf{principal investigators (PIs)} who are responsible for
the overall success of the project;
\textbf{field coordinators (FCs)} who are responsible for
the operation of the project on the ground;
and \textbf{research assistants (RAs)} who are responsible for
handling technical capacity and analytical tasks.

\section{Writing reproducible code in a collaborative environment}

Research reproduciblity and data quality follow naturally from
good code and standardized practices.
Process standardization means that there is
little ambiguity about how something ought to be done,
and therefore the tools that are used to do it are set in advance.
Good code is easier to read and replicate, making it easier to spot mistakes.
The resulting data contains substantially fewer sampling, randomization, and cleaning errors. And all the data work can de easily reviewed before it's published and replicated afterwards.

Code is good when it is both correct and easily understood by whoever reads it.
Most research assistants that join our unit have only been trained in how to code correctly.
While correct results are extremely important, we usually tell our new research assistants that
\textit{when your code runs on your computer and you get the correct results then you are only half-done writing \underline{good} code.}

Just as data collection and management processes have become more social and collaborative,
code processes have as well.\sidenote{\url{https://dimewiki.worldbank.org/wiki/Stata_Coding_Practices}} This means other people need to be able to read your code.
Not only are things like documentation and commenting important,
but code should be readable in the sense that others can:
(1) quickly understand what a portion of code is supposed to be doing;
(2) evaluate whether or not it does that thing correctly; and
(3) modify it efficiently either to test alternative hypotheses
or to adapt into their own work.\sidenote{\url{https://kbroman.org/Tools4RR/assets/lectures/07_clearcode.pdf}}

To accomplish that, you should think of code in terms of three major elements:
\textbf{structure}, \textbf{syntax}, and \textbf{style}.
We always tell people to ``code as if a stranger would read it'',
from tomorrow, that stranger will be you.
The \textbf{structure} is the environment your code lives in:
good structure means that it is easy to find individual pieces of code that correspond to tasks.
Good structure also means that functional blocks are sufficiently independent from each other
that they can be shuffled around, repurposed, and even deleted without damaging other portions.
The \textbf{syntax} is the literal language of your code.
Good syntax means that your code is readable
in terms of how its mechanics implement ideas --
it should not require arcane reverse-engineering
to figure out what a code chunk is trying to do.
\textbf{Style}, finally, is the way that the non-functional elements of your code convey its purpose.
Elements like spacing, indentation, and naming can make your code much more (or much less)
accessible to someone who is reading it for the first time and needs to understand it quickly and correctly.

For some implementation portions where precise code is particularly important,
we will provide minimal code examples either in the book or on the DIME Wiki.
In the book, they will be presented like the following:

\codeexample{code.do}{./code/code.do}

We have tried really hard to make sure that all the Stata code runs,
and that each block is well-formatted and uses built-in functions.
We will also point to user-written functions when they provide important tools.
In particular, we have written two suites of Stata commands,
\texttt{ietoolkit}\sidenote{\url{https://dimewiki.worldbank.org/wiki/ietoolkit}} and \texttt{iefieldkit},\sidenote{\url{https://dimewiki.worldbank.org/wiki/iefieldkit}}
that standardize some of our core data collection workflows.
Providing some standardization to Stata code style is also a goal of this team,
since groups are collaborating on code in Stata more than ever before.
We will not explain Stata commands unless the behavior we are exploiting
is outside the usual expectation of its functionality;
we will comment the code generously (as you should),
but you should reference Stata help-files \texttt{h [command]}
whenever you do not understand the functionality that is being used.
We hope that these snippets will provide a foundation for your code style.
Alongside the collaborative view of data that we outlined above,
good code practices are a core part of the new data science of development research.
Code today is no longer a means to an end (such as a paper),
but it is part of the output itself: it is a means for communicating how something was done,
in a world where the credibility and transparency of data cleaning and analysis is increasingly important.

While adopting the workflows and mindsets described in this book
requires an up-front cost,
it should start to save yourself and others a lot of time and hassle very quickly.
In part this is because you will learn how to do the essential things directly;
in part this is because you will find tools for the more advanced things;
and in part this is because you will have the mindset to doing everything else in a high-quality way.
We hope you will find this book helpful for accomplishing all of the above,
and that you will find that mastery of data helps you make an impact!

\textbf{-- The DIME Analytics Team}

\mainmatter
