% ----------------------------------------------------------------------------------------------
% Setting the stage for successful data work

\begin{fullwidth}
Preparation for collaborative data work begins long before you acquire any data,
and involves planning both software tools and collaboration platforms and processes for your team.
In order to be prepared to do effective data work in a team environment,
you need to structure your workflow in advance.
This means knowing what types of data you'll acquire,
whether the data will require special handling due to size or privacy considerations,
which datasets and outputs you will need at the end of the process,
and how all data files and versions will stay organized throughout.
It's important to plan data workflows in advance because
changing software or protocols halfway through a project can be costly and time-consuming.
Seemingly small decisions such as sharing services, folder structures,
and filenames can be extremely painful to alter down the line in any project.

This chapter will guide you in setting up an effective environment for collaborative data work,
structuring your data work to be well-organized and clearly documented,
and setting up processes to handle confidential data securely.
The first section outlines hows to set up your working environment
to effectively collaborate on technical tasks with others,
and how to document tasks and decisions.
The second section discusses how to organize your code and data so that others
will be able to understand and interact with it easily.
The third section provides guidelines for ensuring
privacy and security when working with confidential data.

\end{fullwidth}

% ----------------------------------------------------------------------------------------------
% ----------------------------------------------------------------------------------------------

\section{Preparing a collaborative work environment}

This section introduces the core concepts and tools
for organizing data work in an efficient, collaborative and reproducible manner.
Some of these skills may seem elementary,
but thinking about simple things from a workflow perspective
can help you make marginal improvements every day you work;
those add up to substantial gains over the course of multiple years and projects.
Together, these processes form a collaborative workflow
that will greatly accelerate your team's ability to get tasks done
on all your projects.

Teams often develop workflows in an ad hoc fashion,
solving new challenges as they arise.
Adaptation is good, of course.
But it is important to recognize
that there are a number of tasks that exist for every project,
and it is more efficient to agree on the corresponding workflows in advance.
For example, documentation, naming schema, folder and output organization,
coding, managing revisions to files, and reviewing each other's work.
These tasks are common to almost every project,
and solutions translate well between projects.
Therefore, there are large efficiency gains to
thinking in advance about the best way to do these tasks,
instead of throwing together a solution when the task arises.
This section outlines the main points to discuss within the team,
and suggests best practice solutions for these tasks.

% ----------------------------------------------------------------------------------------------
\subsection{Setting up your computer}

First things first:
almost all your data work will be done on your computer,
so make sure it's set up for success.
The operating system should be fully updated,
it should be in good working order,
and you should have a \textbf{password-protected} login.
  \index{password protection}

Make sure your computer is backed up to prevent information loss.
  \index{backup}
Follow the \textbf{3-2-1 rule}: maintain 3 copies of all original or irreplaceable data,
on at least 2 different hardware devices you have access to,
with 1 offsite storage method.\sidenote{
  \url{https://www.backblaze.com/blog/the-3-2-1-backup-strategy}}
One example of this setup is having one copy on your primary computer,
one copy on an external hard drive stored in a safe place,
and one copy in the cloud.
In this case, Dropbox and other automatic file sync services do not count as a cloud copy,
since other users can alter or delete them
unless you create a specific folder for this purpose that is not shared with anyone else.
  \index{Dropbox}

Ensure you know how to get the \textbf{absolute file path} for any given file.
Using the absolute file path, starting from the filesystem root,
means that the computer will never accidentally load the wrong file.
  \index{file paths}
On MacOS this will be something like \path{/users/username/git/project/...},
and on Windows, \path{C:/users/username/git/project/...}.
Use forward slashes (\texttt{/}) in file paths for folders,
and whenever possible use only A-Z (the 26 English characters),
dashes (\texttt{-}), and underscores (\texttt{\_}) in folder names and filenames.
For emphasis: \textit{always} use forward slashes (\texttt{/}) in file paths in code,
just like in internet addresses. Do this even if you are using a Windows machine where
both forward and backward slashes are allowed, as your code will otherwise break
if anyone tries to run it on a Mac or Linux machine.
Making the structure of your directories a core part of your workflow is very important,
since otherwise you will not be able to reliably transfer the instructions
for replicating or carrying out your analytical work.

When you are working with others, you will most likely be using
some kind of \textbf{file sharing} software.
  \index{file sharing}
The exact services you use will depend on your tasks,
but in general, there are several approaches to file sharing,
and the three discussed here are the most common.
\textbf{File syncing} is the most familiar method,
and is implemented by software like Dropbox and OneDrive.
  \index{file syncing}
Sync forces everyone to have the same version of every file at the same time,
which makes simultaneous editing difficult but other tasks easier.
They also have some security concerns which we will address later.
\textbf{Version control} is another method,
commonly implemented by tools like Git\sidenote{
	\textbf{Git:} a multi-user version control system for collaborating on and tracking changes to code as it is written.} and GitHub\sidenote{
	\textbf{GitHub:} the biggest publicly available platform for hosting Git projects.}.
  \index{version control}
Version control allows everyone to access different versions of files at the same time,
making simultaneous editing easier but some other tasks harder.
It is also only optimized for specific types of files
(for example, any type of code files).
Finally, \textbf{server storage} is the least-common method,
because there is only one version of the materials,
and simultaneous access must be carefully regulated.
  \index{server storage}
Server storage ensures that everyone has access
to exactly the same files and environment, and it also enables
high-powered computing processes for large and complex data.
All three file sharing methods are used for collaborative workflows,
and you should review the types of data work
that you will be doing, and plan which types of files
will live in which types of sharing services.
It is important to note that they are, in general, not interoperable,
meaning you should not have version-controlled files inside a syncing service,
or vice versa, without setting up complex workarounds,
and you cannot shift files between them without losing historical information.
Therefore, choosing the correct sharing service at the outset is essential.

% ----------------------------------------------------------------------------------------------
\subsection{Documenting decisions and tasks}

Once your technical and sharing workspace is set up,
you need to decide how you are going to communicate with your team.
The first habit that many teams need to break
is using instant communication for management and documentation.
Email is, simply put, not a system. It is not a system for anything. Neither is WhatsApp.\index{email}\index{WhatsApp}
These tools are developed for communicating ``now'' and that is what they do well.
They are not structured to manage group membership or to present the same information
across a group of people, or to remind you when old information becomes relevant.
They are not structured to allow people to collaborate over a long time or to review old discussions.
It is therefore easy to miss or lose communications from the past when they have relevance in the present.
Everything with future relevance that is communicated over e-mail or any other instant medium
-- such as, for example, decisions about sampling --
should immediately be recorded in a system that is designed to keep permanent records.
We call these systems collaboration tools, and there are several that are very useful.\sidenote{
  \url{https://dimewiki.worldbank.org/Collaboration_Tools}}
  \index{collaboration tools}

Good collaboration tools are task-oriented systems
that allow the team to create and assign tasks,
carry out discussions related to single tasks,
track task progress across time, and quickly see the overall project status.
They are web-based so that everyone on your team can access them simultaneously
and have live discussions about tasks and processes.
Such systems link communication to specific tasks so that
related decisions are permanently recorded
and easy to find in the future when questions about that task come up.
Choosing the right tool for your team's needs is essential to designing an effective workflow.
What is important is that your team chooses a system and commits to using it,
so that decisions, discussions, and tasks are easily reviewable long after they are completed.

Some popular and free collaboration tools that meet these criteria are
GitHub project boards, GitHub issues and Dropbox Paper.
Any specific list of software will quickly be outdated;
we mention these as examples that have worked for our team.
Different collaboration tools can be used different types of tasks.
Our team, for example, uses GitHub Issues for code-related tasks,
and Dropbox Paper for more managerial and office-related tasks.
GitHub creates incentives for writing down why changes were made
in response to specific discussions
as they are completed, creating naturally documented code.
It is useful also because tasks in Issues can clearly be tied to file versions.
On the other hand, Dropbox Paper provides a clean interface with task notifications,
assignments, and deadlines,
and is very intuitive for people with non-technical backgrounds.
Therefore, it is a useful tool for managing non-code-related tasks.

% ----------------------------------------------------------------------------------------------
\subsection{Choosing software}

Choosing the right software environments can make your work significantly easier.
  \index{software environments}
It may be difficult or costly to switch halfway through a project, so
think ahead about the various software your team will use.
Take into account the technical abilities of team members,
how important it is to access files offline constantly,
the type of data you will need to access,
and the level of security required.
Big datasets require additional infrastructure and may overburden
the tools commonly used for small datasets,
particularly if you are trying to sync or collaborate on them.
Also consider the cost of licenses, the time to learn new tools,
and the stability of the tools.
There are few strictly right or wrong choices for software,
but what is important is that you plan in advance
and understand how the chosen tools will interact with your workflows.

Ultimately, the goal is to hold your code environment constant
over the lifecycle of a single project.
While this means you will inevitably have different projects
with different code environments, each one will be better than the last,
and you will avoid the costly process of migrating an ongoing project
into a new code enviroment.
Code environment should be constant down to the software level:
the specific versions of software and the individual packages you use
should be referenced or maintained so that they can be reproduced going forward,
even if different releases contain changes that would break your code
or change your results.
  \index{software versions}
DIME Analytics developed the command \texttt{ieboilstart} in the \texttt{ietoolkit} package
to support Stata version stability.\sidenote{
  \url{https://dimewiki.worldbank.org/ieboilstart}}

Next, think about how and where you write and execute code.
This book is intended to be agnostic to the size or origin of your data,
but we are going to broadly assume that you are using
one of the two most popular statistical software packages: R or Stata.
(If you are using another language, like Python,
many of the same principles apply but the specifics will be different.)
Most of your code work will be done in a code editor.
This does not need to be the same program the code runs in;
using an external editor can be preferable since your editor will not crash if your code does,
and the editor may offer additional features aimed at writing code well.
If you are working in R, \textbf{RStudio} is the typical choice.\sidenote{
  \url{https://www.rstudio.com}}
For Stata, the built-in do-file editor is the most widely adopted code editor,
but we recommend using an external editor such as
\textbf{Atom}\sidenote{\url{https://atom.io}} or
\textbf{Sublime}\sidenote{\url{https://www.sublimetext.com}}.
These editors can be configured to run Stata code externally
and offer great accessibility and quality features.
For example, they can access an entire directory -- rather than a single file --
which gives you directory-level views and file management actions,
such as folder management, Git integration,
and simultaneous work with other types of files, without leaving the editor.

% ----------------------------------------------------------------------------------------------
% ----------------------------------------------------------------------------------------------
\section{Organizing code and data}

We assume you are going to do your analytical work through code,
and that you want all your processes to be documented and replicable.
Though it is possible to interact with some statistical software
through the user interface without writing any code,
we strongly advise against it.
Writing code creates a record of every task you performed.
It also prevents direct interaction with the data files that could lead to non-reproducible steps.
You may do some exploratory tasks by point-and-click or typing directly into the console,
but anything that is included in a research output
must be coded up in an organized fashion so that you can release
the exact code recipe that goes along with your final results.
Still, organizing code and data into files and folders is not a trivial task.
What is intuitive to one person rarely comes naturally to another,
and searching for files and folders is everybody's least favorite task.
As often as not, you come up with the wrong one,
and then it becomes very easy to create problems that require complex resolutions later.
This section provides basic tips on managing the folder
that stores your project's data work.

Maintaining an organized file structure for data work is the best way
to ensure that you, your teammates, and others
are able to easily advance, edit, and replicate your work in the future.
It also ensures that automated processes from code and scripting tools
are able to interact well with your work,
whether they are yours or those of others.
File organization makes your own work easier as well as more transparent,
and will make your code easier to combine with tools like version control systems
that aim to cut down on the amount of repeated tasks you have to perform.
It is worth thinking in advance about how to store, name, and organize
the different types of files you will be working with,
so that there is no confusion down the line
and everyone has the same expectations.

% ----------------------------------------------------------------------------------------------
\subsection{Organizing files and folder structures}

Once you start a research project,
the number of scripts, datasets, and outputs that you have to manage will grow very quickly.
This can get out of hand just as quickly,
so it's important to organize your data work and follow best practices from the beginning.
You should agree with your team on a specific directory structure,
and set it up at the beginning of the project.
You should also agree on a file naming convention.\sidenote{
	\url{https://dimewiki.worldbank.org/wiki/Naming_Conventions}}
This will help you to easily find project files and
ensure that all team members can easily run the same code.

To support consistent folder organization at DIME,
DIME Analytics created \texttt{iefolder}\sidenote{
  \url{https://dimewiki.worldbank.org/iefolder}}
as a part of our \texttt{ietoolkit} package.\index{\texttt{iefolder}}\index{\texttt{ietoolkit}}
This Stata command sets up a pre-standardized folder structure
for what we call the \texttt{DataWork} folder.\sidenote{
  \url{https://dimewiki.worldbank.org/DataWork_Folder}}
The \texttt{DataWork} folder includes folders for all the steps of a typical project.
  \index{\texttt{DataWork} folder}
Since each project will always have unique needs,
we have tried to make the structure easy to adapt.
Having a universally standardized folder structure
across the entire portfolio of projects
means that everyone can easily move between projects
without having to reorient on file and folder organization.

If you do not already have a standard file structure across projects,
\texttt{iefolder} is an easy template to start from.
The command creates a \texttt{DataWork} folder at the project level,
and within that folder, creates standardized directory structures
for each data source or survey round.
Within each subdirectory, \texttt{iefolder} creates folders for raw encrypted data,
raw deidentified data, cleaned data, final data, outputs, and documentation.
It creates a parallel folder structure for the code files
that move the data through this progression,
and for the final analytical outputs.
The \texttt{ietoolkit} package also includes the \texttt{iegitaddmd} command,
which can place \texttt{README.md} placeholder files in your folders so that
your folder structure can be shared using Git.\sidenote{
	Git only tracks files, so empty folders
	-- which most folders are in the beginning of a project --
	are ignored if placeholder files are not used,
	leading to only parts of the folder structure being shared across the team.}
Since these placeholder files are written in a plaintext language called \textbf{Markdown},
they also provide an easy way to document the contents of every folder in the structure.
  \index{Markdown}

The \texttt{DataWork} folder may be created either inside
an existing project folder, or it may be created as a separate root folder.
We advise keeping project management materials
(such as contracts, Terms of Reference, briefs and other administrative or management work)
separate from the datawork folder structure.
It is useful to maintain project management materials in a synced location like Dropbox,
whereas the code folder should be maintained in a version-controlled location like GitHub.
(Remember, a version-controlled folder \textit{should not}
be stored in a synced folder that is shared with other people.
Combining these two types of collaboration tools
almost always creates undesired functionalities.)

\subsection{Setting up a version control system}
We recommend using a \textbf{version control system} to
maintain control of file history and functionality.
A good version control system tracks who edited each file and when,
allows you to revert to previous versions,
and provides a protocol for ensuring that conflicting versions are avoided.
This is important, for example, for your team
to be able to find the version of a presentation that you delivered to a donor,
or to understand why the significance level of your estimates has changed.
Everyone who has ever encountered a file named something like \texttt{final\_report\_v5\_LJK\_KLE\_jun15.docx}
can appreciate how useful such a system can be.

Most syncing services offer some kind of rudimentary version control;
these are usually enough to manage changes to binary files (such as Word and PowerPoint documents)
without needing to rely on dreaded filename-based versioning conventions.
For code files, however, a more detailed version control system is usually desirable.
We recommend using Git for version-control of all data work.
Git manages changes to all \textbf{plaintext} files,
this includes all code files, most raw outputs,
and written outputs that use plaintext tools such as
{\LaTeX} and dynamic documents.\index{{\LaTeX}}\index{dynamic documents}
Git tracks all the changes you make to each plaintext file,
and allows you to go back to previous versions without losing the information on changes made.
It also makes it possible to work on multiple parallel versions of the file,
so you don't risk breaking code for other team members as you try something new.


% ----------------------------------------------------------------------------------------------
\subsection{Documenting and organizing code}
Good code is written in a way that is easily understood and run by others.
Below we discuss a few crucial steps to code organization.
They all come from the principle that code is an output by itself,
not just a means to an end,
and should be written thinking of how easy it will be for someone to read it later.
At the end of this section, we include a template for a master script in Stata,
to provide a concrete example of the required elements and structure.
Throughout this section, we refer to lines of the example do-file
to give concrete examples of the required code elements, organization and structure.

To be readable, code must be well-documented.
Start by adding a code header to every file.
A code header is a long \textbf{comment}\sidenote{
  \textbf{Comments:} Code components that have no function,
  but describe in plain language what the code is supposed to do.
}
that details the functionality of the entire script;
refer to lines 5-10 in the example do-file.
This should include simple things such as
the purpose of the script and the name of the person who wrote it.
If you are using a version control software,
the last time a modification was made and the person who made it will be recorded by that software.
Otherwise, you should include it in the header.
You should always track the inputs and outputs of the script,
as well as the uniquely identifying variable;
refer to lines 49-51 in the example do-file.
When you are trying to track down which code creates which dataset, this will be very helpful.
While there are other ways to document decisions related to creating code,
the information that is relevant to understand the code should always be written in the code file.

Two types of comments should be included in the script itself.
The first type of comment describes what is being done;
refer to line 35 in the example do-file.
This might be easy to understand from the code itself
if you know the language well enough and the code is clear,
but often it is still a great deal of work to reverse-engineer the code's intent.
Writing the task in plain English (or whichever language you use to communicate with your team)
will make it easier for everyone to read and understand the code's purpose.
It can also help you organize your own work and ensure you are following logical steps.
The second type of comment explains why the code is performing a task in a particular way.
As you are writing code, you are making a series of decisions that
(hopefully) make perfect sense to you at the time.
These are often highly specialized and may exploit a functionality
that is not obvious or has not been seen by others before.
Even you will probably not remember the exact choices that were made in a couple of weeks.
Therefore, you must document your precise processes in your code.

Code files should be stored in an easy-to-find location and named in a meaningful way.
  \index{code organization}
Breaking your code into independently readable ``chunks'' is good practice for code organization.
You should write each functional element as a chunk that can run completely on its own.
This ensures that each code component is independent;
it does not rely on a complex program state
created by other code chunks that are not obvious from the immediate context.
One way to do this is by creating sections in your script to identify where a specific task is completed.
For example, if you want to find the line in your code where the directory is set,
you can go straight to \texttt{PART 2: Prepare folder paths and define programs},
instead of reading line by line through the entire code.
RStudio makes it very easy to create sections,
and it compiles them into an interactive script index for you.
In Stata, you can use comments to create section headers
(see line 24 of the example do-file),
though they're just there to make the reading easier and don't have functionality.
Since an index is not automated,
create this manually in the code header by copying and pasting section titles
(see lines 8-10 in the example do-file).
You can then add and navigate through them using the \texttt{find} functionality.
Since Stata code is harder to navigate, as you will need to scroll through the document,
it's particularly important to avoid writing very long scripts.
Therefore, in Stata at least, we recommend breaking code tasks down
into separate do-files, since there is no limit on how many you can have,
how detailed their names can be, and no advantage to writing longer files.
One reasonable rule of thumb is to not write do-files that have more than 200 lines.
This is an arbitrary limit, just like the common practice of limiting code lines to 80 characters:
it seems to be ``enough but not too much'' for most purposes.

\subsection{Working with a master script}
To bring all these smaller code files together, you must maintain a master script.\sidenote{
	\url{https://dimewiki.worldbank.org/Master\_Do-files}}
  \index{master do-file}
A master script is the map of all your project's data work
which serves as a table of contents for the instructions that you code.
Anyone should be able to follow and reproduce all your work from
raw data to all outputs by simply running this single script.
By follow, we mean someone external to the project who has the master script and all the input data can
(i) run all the code and recreate all outputs,
(ii) have a general understanding of what is being done at every step, and
(iii) see how codes and outputs are related.
The master script is also where all the settings are established,
such as versions, folder paths, functions, and constants used throughout the project.

Try to create the habit of running your code from the master script.
Creating ``section switches'' using macros or objects to run only the codes related to a certain task
should always be preferred to manually open different scripts to run them in a certain order
(see Part 1 of \texttt{stata-master-dofile.do} for an example of how to do this).
Furthermore, running all scripts related to a particular task through the master whenever one of them is edited
helps you identify unintended consequences of the changes you made.
Say, for example, that you changed the name of a variable created in one script.
This may break another script that refers to this variable.
But unless you run both of them when the change is made, it may take time for that to happen,
and when it does, it may take time for you to understand what's causing an error.
The same applies to changes in datasets and results.

To link code, data and outputs,
the master script reflects the structure of the \texttt{DataWork} folder in code
through globals (in Stata) or string scalars (in R);
refer to lines 35-40 of the example do-file.
These coding shortcuts can refer to subfolders,
so that those folders can be referenced without repeatedly writing out their absolute file paths.
Because the \texttt{DataWork} folder is shared by the whole team,
its structure is the same in each team member's computer.
The only difference between machines should be
the path to the project root folder, i.e. the highest-level shared folder,
which in the context of \texttt{iefolder} is the \texttt{DataWork} folder.
This is reflected in the master script in such a way that
the only change necessary to run the entire code from a new computer
is to change the path to the project folder to reflect the filesystem and username;
refer to lines 27-32 of the example do-file.
The code in \texttt{stata-master-dofile.do} shows how folder structure is reflected in a master do-file.
Because writing and maintaining a master script can be challenging as a project grows,
an important feature of the \texttt{iefolder} is to write sub-master do-files
and add to them whenever new subfolders are created in the \texttt{DataWork} folder.

In order to maintain well-documented and organized code,
you should agree with your team on a plan to review code as it is written.
  \index{code review}
Reading other people's code is the best way to improve your coding skills.
And having another set of eyes on your code will make you more comfortable with the results you find.
It's normal (and common) to make mistakes as you write your code.
Reading it again to organize and comment it as you prepare it to be reviewed will help you identify them.
Try to have a code review scheduled frequently,
every time you finish writing a piece of code, or complete a small task.
If you wait for a long time to have your code reviewed, and it gets too complex,
preparation and code review will require more time and work,
and that is usually the reason why this step is skipped.
One other important advantage of code review is that
making sure that the code is running properly on other machines,
and that other people can read and understand the code easily,
is the easiest way to be prepared in advance for a smooth project handover
or for release of the code to the general public.

\codeexample{stata-master-dofile.do}{./code/stata-master-dofile.do}
% ----------------------------------------------------------------------------------------------

\section{Preparing to handle confidential data}

Anytime you are working with original data in a development research project,
you are almost certainly handling data that include
\textbf{personally-identifying	information (PII)}.\index{personally-identifying information}\sidenote{
	\textbf{Personally-identifying information:} any piece or set of information
	that can be used to identify an individual research subject.
	\url{https://dimewiki.worldbank.org/De-identification\#Personally\_Identifiable\_Information}}
PII is information which can, without any transformation or linkage,
be used to identify individual people, households, villages,
or firms (or other units) that were part of a data collection.
\index{data collection}
In all cases where this type of information is involved,
you must make sure that you adhere to several core principles.
These include ethical approval, participant consent,
data security, and participant privacy.
If you are a US-based researcher, you will become familiar
with a set of governance standards known as ``The Common Rule''.\sidenote{
	\url{https://www.hhs.gov/ohrp/regulations-and-policy/regulations/common-rule/index.html}}
If you interact with European institutions or persons,
you will also become familiar with the General Data Protection Regulation (GDPR),\sidenote{
	\url{http://blogs.lshtm.ac.uk/library/2018/01/15/gdpr-for-research-data}}
a set of regulations governing \textbf{data ownership} and privacy standards.
No matter who you are or what exact legal requirements you face,
the core principles and practices you need to consider will always be similar.

\subsection{Obtaining ethical approval and consent}

Most of the field research done in development involves human subjects.\sidenote{
	\url{https://dimewiki.worldbank.org/Human_Subjects_Approval}}
\index{human subjects}
As a researcher, you are asking people to trust you with personal information about themselves:
where they live, how rich they are, whether they have committed or been victims of crimes,
their names, their national identity numbers, and all sorts of other data.
PII data carries strict expectations about data storage and handling,
and it is the responsibility of the research team to satisfy these expectations.\sidenote{
	\url{https://dimewiki.worldbank.org/Research_Ethics}}
Your donor or employer will most likely require you to hold a certification from a source
such as Protecting Human Research Participants\sidenote{
	\url{https://phrptraining.com}}
or the CITI Program.\sidenote{
	\url{https://about.citiprogram.org/en/series/human-subjects-research-hsr}}

For almost all such data collection and research activities,
you will be required to complete some form of \textbf{Institutional Review Board (IRB)} process.\sidenote{
	\textbf{Institutional Review Board (IRB):} An institution formally responsible for ensuring that research meets ethical standards.}
\index{Institutional Review Board}
Most commonly this consists of a formal application for approval of a specific
protocol for consent, data collection, and data handling.\sidenote{
  \url{https://dimewiki.worldbank.org/IRB_Approval}}
Which IRB has sole authority over your project is not always apparent,
particularly if some institutions do not have their own.
It is customary to obtain an approval from a university IRB
where at least one PI is affiliated,
and if work is being done in an international setting,
approval is often also required
from an appropriate local institution subject to the laws of the country where data originates.

One primary consideration of IRBs
is the protection of the people about whom information is being collected
and whose lives may be affected by the research design.
Some jurisdictions (especially those responsible to EU law) view all personal data
as intrinsically owned by the persons who they describe.
This means that those persons have the right to refuse to participate in data collection
before it happens, as it is happening, or after it has already happened.
It also means that they must explicitly and affirmatively consent
to the collection, storage, and use of their information for any purpose.
Therefore, the development of appropriate consent processes is of primary importance.
All survey instruments must include a module in which the sampled respondent grants informed consent to participate.
Research participants must be informed of the purpose of the research,
what their participation will entail in terms of duration and any procedures,
any foreseeable benefits or risks,
and how their identity will be protected.\sidenote{
	\url{https://www.icpsr.umich.edu/icpsrweb/content/datamanagement/confidentiality/conf-language.html}}
There are special additional protections in place for vulnerable populations,
such as minors, prisoners, and people with disabilities,
and these should be confirmed with relevant authorities if your research includes them.

IRB approval should be obtained well before any data is acquired.
IRBs may have infrequent meeting schedules
or require several rounds of review for an application to be approved.
If there are any deviations from an approved plan or expected adjustments,
report these as early as possible so that you can update or revise the protocol.
Particularly at universities, IRBs have the power to retroactively deny
the right to use data which was not acquired in accordance with an approved plan.
This is extremely rare, but shows the seriousness of these considerations
since the institution itself may face legal penalties if its IRB
is unable to enforce them. As always, as long as you work in good faith,
you should not have any issues complying with these regulations.

\subsection{Ensuring privacy and security in research data}

In order to safeguard PII and protect respondent privacy,
you must set up a secure data system from the outset of a project.\sidenote{
	\url{https://dimewiki.worldbank.org/Data_Security}}
There are several components to this.
First, you need a system for managing strong and unique passwords for
all accounts -- including personal accounts like computer logins and email.
There are several services that create and store these passwords for you,
and some provide utilities for securely sharing passwords with others.
Second, all machines should have \textbf{hard disk encryption} enabled.\sidenote{
	\textbf{Encryption:} Methods which ensure that files are unreadable even if laptops are stolen, databases are hacked, or any other type of unauthorized access is obtained.
	\url{https://dimewiki.worldbank.org/Encryption}}
\index{encryption}
Disk encryption is available on many modern operating systems;
it prevents your files from ever being accessed
without first entering the system password.
As with all critical passwords, your system password should be strong,
memorable, and backed up in a separate secure location.
Third, you must encrypt datasets that include confidential information
during data collection, storage, and transfer.\index{encryption}\index{data transfer}\index{data storage}
When files are properly encrypted,
the information they contain will be completely unreadable and unusable
even if they were to be intercepted my a malicious
``intruder'' or accidentally made public.
There are many encryption options available, from enterprise-grade solutions to free software.
DIME uses Veracrypt for file encryption,
our protocols are available on GitHub\sidenote{\url{https://github.com/worldbank/dime-standards/blob/master/dime-research-standards/pillar-4-data-security/data-security-resources/veracrypt-guidelines.md}}

The easiest way to reduce the risk of leaking confidential information
is to use it as rarely as possible.
It is often very simple to conduct planning and analytical work
using a subset of the data that does not include this type of information.
\index{privacy}
Some examples of PII variables include names, addresses, and geolocations,
email addresses, phone numbers,
and bank accounts or other financial details.\index{geodata}\index{de-identification}
If you are working in a context or population that is either small, specific,
or has extensive linkable data sources available to others,
information like someone's age and gender may be sufficient to
disclose their identify, even though those variables would not be considered PII in general.
There is no one-size-fits-all solution to determine what is PII,
research teams have to use careful judgment in each case to avoid statistical disclosure.\sidenote{
	\url{https://sdcpractice.readthedocs.io}}\index{statistical disclosure}
	%WIKI need to create wiki page
It is important to keep in mind that data privacy principles apply
not only for the respondent giving you the information
but also for their household members or other individuals who are included in the data.

It is in practice impossible to \textbf{anonymize} data.
There is always some statistical chance that an individual's identity
will be re-linked to the data collected about them
-- even if that data has had all directly identifying information removed --
by using some other data that becomes identifying when analyzed together.
For this reason, we recommend de-identification in two stages.
The \textbf{initial de-identification} process strips the data of direct identifiers
as early in the process as possible,
to create a working de-identified dataset that
can be shared \textit{within the research team} without the need for encryption.
This simplifies workflows.
The \textbf{final de-identification} process involves
making a decision about the trade-off between
risk of disclosure and utility of the data
before publicly releasing a dataset.\sidenote{
	\url{https://sdcpractice.readthedocs.io/en/latest/SDC\_intro.html\#need-for-sdc}}

Secure data storage and transfer are ultimately your personal responsibility.\sidenote{
	\url{https://dimewiki.worldbank.org/Data_Security}}
There are several precautions needed to ensure that your data is safe.
First, all online and offline accounts
-- including personal accounts like computer logins and email --
need to be protected by strong and unique passwords.
There are several services that create and store these passwords for you,
and some provide utilities for sharing passwords with others
inside that secure environment.
However, password-protection alone is not sufficient,
because if the underlying data is obtained through a leak the information itself remains usable.
Datasets that include confidential information
\textit{must} therefore be \textbf{encrypted}\sidenote{
	\textbf{Encryption:} Methods which ensure that files are unreadable even if laptops are stolen, databases are hacked, or any other type of unauthorized access is obtained.
	\url{https://dimewiki.worldbank.org/Encryption}}
during data collection, storage, and transfer.\index{encryption}\index{data transfer}\index{data storage}
What data security protocols you employ will depend on project needs and data sources,
but agreeing on a protocol from the start of a project will make your life easier.
Finally, having an end-of-life plan for data is essential:
you should always know how to transfer access and control to a new person if the team changes,
and what the expiry of the data and the planned deletion processes are.
