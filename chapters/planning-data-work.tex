%------------------------------------------------

\begin{fullwidth}
Preparation for data work begins long before you collect any data,
and involves planning both the software tools you will use yourself
as well as the collaboration platforms and processes for your team.
In order to be prepared to work on the data you receive with a group,
you need to know what you are getting into.
This means knowing which data sets and output you need at the end of the process,
how they will stay organized and linked,
what different types and levels of data you'll handle,
and whether the data will require special handling due to volume or privacy considerations.
Identifying these details creates a \textbf{data map} for your project,
giving you and your team a sense of how information resources should be organized.
It's okay to update this map once the project is underway --
the point is that everyone knows what the plan is.

Then, you must identify and prepare your collaborative tools and workflow.
Changing software and protocols half-way through a project can be costly and time-consuming,
so it's important to think ahead about decisions that may seem of little consequence
(think: creating a new folder and moving files into it).
Similarly, having a self-documenting discussion platform
makes working together on outputs much easier from the very first discussion.
This chapter will discuss some tools and processes that
will help prepare you for collaboration and replication.
We will try to provide free, open-source, and platform-agnostic tools wherever possible,
and point to more detailed instructions when relevant.
However, most have a learning and adaptation process,
meaning you will become most comfortable with each tool
only by using it in real-world work.
Get to know them well early on,
so that you do not spend a lot of time later figuring out basic functions.
\end{fullwidth}

%------------------------------------------------

\section{Preparing a collaborative work environment}

Being comfortable using your computer and having the tools you need in reach is key.
This section provides a brief introduction to key concepts and toolkits
that can help you take on the work you will be primarily responsible for.
Some of these skills may seem elementary,
but thinking about simple things from a workflow perspective
can help you make marginal improvements every day you work.
These add up, and together form a collaborative workflow
that will greatly accelerate your team's ability to get tasks done
on every project you take on together.

Teams often develop their workflows as they go,
solving new challenges as they arise.
This is broadly okay -- but it is important to recognize
that there are a number of tasks that will always have to be completed during any project,
and that the corresponding workflows can be agreed on in advance.
These include documentation methods, software choices,
naming schema, organizing folders and outputs, collaborating on code,
managing revisions to files, and reviewing each other's work.
These tasks appear in almost every project,
and also translate well between projects.
Therefore, there are large efficiency gains to
thinking about the best way to do these tasks ahead of time,
instead of just doing it quickly as needed.
This chapter will outline the main points to discuss within the team,
and suggest some common solutions.

\subsection{Setting up your computer}

First things first: turn on your computer.
Make sure you have fully updated the operating system,
that it is in good working order,
and that you have a \textbf{password-protected} login.
All machines should have hard disk encryption enabled.
Disk encryption is built in to most modern operating systems;
the service is currently called BitLocker on Windows or FileVault on MacOS.
Disk encryption prevents your files from ever being accessed
without first entering the system password.
This is different from file-level encryption,
which makes individual files unreadable without a specific key.
We will address that in more detail later.
As with all critical passwords, your system password should be strong,
memorable, and backed up in a separate secure location.

Make sure your computer is backed up to prevent information loss.
Follow the \textbf{3-2-1 rule}: maintain 3 copies of everything,
on at least 2 different hardware devices you have access to,
with 1 offsite storage method.\sidenote{
  \url{https://www.backblaze.com/blog/the-3-2-1-backup-strategy/}}
One reasonable setup is having your primary computer,
then a local hard drive managed with a tool like Time Machine
(alternatively, a fully synced secondary computer),
and either a remote copy maintained by a cloud backup service
or all original files stored on a remote server.
Dropbox and other synced files count only as local copies and never as remote backups,
because other users can alter them.

Find your \textbf{home folder}. It is never your desktop.
On MacOS, this will be a folder with your username.
On Windows, this will be something like ``This PC''.
Nearly everything we talk about will assume you are starting from here.
Ensure you know how to get the \textbf{absolute file path} for any given file.
On MacOS this will be something like \path{/users/username/dropbox/project/...},
and on Windows, \path{C:/users/username/github/project/...}.
We will write file paths such as \path{/Dropbox/project-title/DataWork/EncryptedData/}
using forward slashes (\texttt{/}), and mostly use only A-Z, dash (\texttt{-}), and underscore (\texttt{\_}).
You should \textit{always} use forward slashes (\texttt{/}) in file paths,
just like an internet address, and no matter how your computer writes them,
because the other type will cause your work to break many systems.
You can use spaces in names of non-technical files, but not technical ones.\sidenote{
  \url{http://www2.stat.duke.edu/~rcs46/lectures_2015/01-markdown-git/slides/naming-slides/naming-slides.pdf}}
Making the structure of your files part of your workflow is really important,
as is naming them correctly so you know what is where.

If you are working with others, you will most likely be using some kind
of file collaboration method.
The exact method you use will depend on your tasks,
but three methods are the most common.
\textbf{File syncing} is the most familiar method,
and is implemented by software like Dropbox and OneDrive.
Sync forces everyone to have the same version of every file at the same time,
which makes simultaneous editing difficult but other tasks easier.
(They also have some security concerns which we will address later.)
\textbf{Version control} is another method,
and is implemented by tools like GitHub.
Version control allows everyone to have different versions at the same time,
making simultaneous editing easier but other tasks harder.
Finally, \textbf{server storage} is the least-used method,
because there is only one version of the materials,
and simultaneous access must be carefully regulated.
However, server storage ensures that everyone has access
to exactly the same files, and also enables
high-powered computing processes for large and complex data.
All three methods are used for sharing and collaborating,
and you should review the types of data work
that you are going to be doing, and plan which processes
will live in which types of locations.

\subsection{Documenting decisions and tasks}

Once your technical workspace is set up,
you need to decide how you are going to communicate with your team.
The first habit that many teams need to break is using e-mail for management tasks.
E-mail is, simply put, not a system. It is not a system for anything.
E-mail was developed for communicating ``now'' and this is what it does well.
It is not structured to manage group membership or to present the same information
across a group of people, or to remind you when old information becomes relevant.
It is not structured to allow people to collaborate over a long time or to review old discussions.
It is therefore easy to miss or lose communications from the past when they have relevance in the present.
Everything that is communicated over e-mail or any other medium should
immediately be transferred into a system that is designed to keep records.
We call these systems collaboration tools, and there are several that are very useful.\sidenote{
  \url{https://dimewiki.worldbank.org/wiki/Collaboration_Tools}}

Many task management tools are online or web-based,
so that everyone on your team can access them simultaneously.
Many of them are based on an underlying system known as ``Kanban boards''.\sidenote{
  \url{https://en.wikipedia.org/wiki/Kanban_board}}
This task-oriented system allows the team to create and assign tasks,
to track progress across time, and to quickly see the project state.
These systems also link communications to specific tasks so that
the records related to decision making on those tasks is permanently recorded.
A common and free implementation of this system is the one found in GitHub project boards.
You may also use a system like GitHub Issues or task assignment on Dropbox Paper,
which has a more chronological structure, if this is appropriate to your project.
What is important is that you have a system and you stick to it,
so that decisions, discussions, and tasks are easily reviewable long after they are completed.

Just like we use different file sharing tools for different types of files,
we use different collaboration tools for different types of tasks.
Our team, for example, uses GitHub Issues for code-related tasks,
and Dropbox Paper for more managerial and office-related tasks.
GitHub creates incentives for writing down why changes were made
as they are saved, creating naturally documented code.
It is useful also because tasks in Issues can clearly be tied to file versions.
Thus, GitHub serves as a great tool for managing code-related tasks.
On the other hand, Dropbox Paper provides a good interface with notifications,
and is very intuitive for people with non-technical backgrounds.
It is useful because tasks can be easily linked to other documents saved in Dropbox.
Thus, it is a great tool for managing non-code-related tasks.
Neither of these tools require much technical knowledge;
they merely require an agreement and workflow design
so that the people assigning the tasks are sure to set them up in the appropriate system.

\subsection{Choosing software}

Choosing the right personal and team working environment can also make your work easier.
It may be difficult or costly to switch halfway through a project, so
think ahead about the different software to be used.
Take into account the different levels of techiness of team members,
how important it is to access files offline constantly,
as well as the type of data you will need to access and the security needed.
Big datasets require additional infrastructure and may overburden
the traditional tools used for small datasets,
particularly if you are trying to sync or collaborate on them.
Also consider the cost of licenses, the time to learn new tools,
and the stability of the tools.
There are few strictly right or wrong answers,
but what is important is that you have a plan in advance
and understand how your tools with interact with your work.

Ultimately, the goal is to ensure that you will be able to hold
your code environment constant over the life cycle of a single project.
While this means you will inevitably have different projects
with different code environments, each one will be better than the last,
and you will avoid the extremely costly process of migrating a project
into a new code enviroment.
This can be set up down to the software level:
you need to ensure that even specific versions of software
and the individual packages you use
are referenced or maintained so that they can be reproduced going forward
even if their most recent version contains changes that would break your code.
(For example, our command \texttt{ieboilstart} in the \texttt{ietoolkit} package
provides functionality to support Stata version stability.\sidenote{
  \url{https://dimewiki.worldbank.org/wiki/ieboilstart}})

Next, think about how and where you write code.
The rest of this book focuses mainly on primary survey data,
so we are going to broadly assume that you are using ``small'' data
in one of the two popular desktop-based packages for that kind of work: R or Stata.
(If you are using another language, like Python,
or working with big data projects on a server installation,
you can skip this section.)
The most important part of working with code is a code editor.
This does not need to be the same program as the code runs in.
This can be preferable since your editor will not crash if your code does,
and may offer additional features aimed at writing code well.
If you are working in R, \textbf{RStudio} is the typical choice.\sidenote{
  \url{https://www.rstudio.com}}
For Stata, the built-in do-file editor is the most widely adopted code editor,
and \textbf{Atom}\sidenote{\url{https://atom.io}} and \textbf{Sublime}\sidenote{\url{https://www.sublimetext.com/}} can also be configured to run Stata code externally, while offering great accessibility features.
For example, these tools can work on an entire directory -- rather than a single file --
which gives you access to directory views and file management actions,
such as folder management, Git integration, and simultaneous work with other types of files without leaving the editor.

In our field of development economics,
Stata is by far the most commonly used programming language,
and the Stata do-file editor the most common editor.
We focus on Stata-specific tools and instructions in this book.
Hence, we will use the terms `script' and `do-file'
interchangeably to refer to Stata code throughout.
This is only in part due to its popularity.
Stata is primarily a scripting language for statistics and data,
meaning that its users often come from economics and statistics backgrounds
and understand Stata to be encoding a set of tasks as a record for the future.
We believe that this must change somewhat:
in particular, we think that practitioners of Stata
must begin to think about their workflows more as programmers do,
and that people who adopt this approach will be dramatically
more capable in their analytical ability.
This means that they will be more productive when managing teams,
and more able to focus on the challenges of experimental design
and econometric analysis, rather than spending excessive time
re-solving problems on the computer.
Stata also has relatively few resources of this type available,
and the ones that we have created and shared here
we hope will be an asset to all its users.

\section{Organizing code and data}

Organizing files and folders is not a trivial task.
What is intuitive to one person rarely comes naturally to another,
and searching for files and folders is everybody's least favorite task.
As often as not, you come up with the wrong one,
and then it becomes very easy to create problems that require complex resolutions later.
This section will provide basic tips on managing the folder
that will store your project's data work.

We assume you will be working with code and data throughout your project.
We further assume you will want all your processes to be recorded
and easily findable at any point in time.
Maintaining an organized file structure for data work is the best way
to ensure that you, your teammates, and others
are able to easily work on, edit, and replicate your work in the future.
It also ensures that core automation processes like script tools
are able to interact well will your work,
whether they are yours or those of others.
File organization makes your own work easier as well as more transparent,
and plays well with tools like version control systems
that aim to cut down on the amount of repeated tasks you have to perform.
It is worth thinking in advance about how to store, name, and organize
the different types of files you will be working with,
so that there is no confusion down the line
and everyone has interoperable expectations.

\subsection{File and folder management}

Agree with your team on a specific folder structure, and
set it up at the beginning of the research project
to prevent folder re-organization that may slow down your workflow and,
more importantly, prevent your code files from running.
DIME Analytics created and maintains
\texttt{iefolder}\sidenote{\url{https://dimewiki.worldbank.org/wiki/iefolder}}
as a part of our \texttt{ietoolkit} suite.
This command sets up a standardized folder structure for what we call the \texttt{/DataWork/} folder.\sidenote{\url{https://dimewiki.worldbank.org/wiki/DataWork_Folder}}
It includes folders for all the steps of a typical DIME project.
However, since each project will always have its own needs,
we tried to make it as easy as possible to adapt when that is the case.
The main advantage of having a universally standardized folder structure
is that changing from one project to another requires less
time to get acquainted with a new organization scheme.

The first thing your team will need to create is a shared folder.
If every team member is working on their local computers,
there will be a lot of crossed wires when collaborating on any single file,
and e-mailing one document back and forth is not efficient.
Your folder will contain all your project's documents.
It will be the living memory of your work.
The most important thing about this folder is for everyone in the team to know how to navigate it.
Creating folders with self-explanatory names will make this a lot easier.
Naming conventions may seem trivial,
but often times they only make sense to whoever created them.
It will often make sense for the person in the team who uses a folder the most to create it.

\texttt{iefolder} also creates master do-files.\sidenote{\url{https://dimewiki.worldbank.org/wiki/Master\_Do-files}}
Master scripts are a key element of code organization and collaboration,
and we will discuss some important features soon.
With regard to folder structure, it's important to keep in mind
that the master script should mimic the structure of the \texttt{/DataWork/} folder.
This is done through the creation of globals (in Stata) or string scalars (in R).
These are ``variables'' -- coding shortcuts that refer to subfolders,
so that those folders can be referenced without repeatedly writing out their complete filepaths.
Because the \texttt{/DataWork/} folder is shared by the whole team,
its structure is the same in each team member's computer.
What may differ is the path to the project folder (the highest-level shared folder).
This is reflected in the master script in such a way that
the only change necessary to run the entire code from a new computer
is to change the path to the project folder.
The code in \texttt{stata-master-dofile.do} shows how folder structure is reflected in a master do-file.


\subsection{Version control}

A \textbf{version control system} is required to manage changes to any computer file.
A good version control system tracks who edited each file and when,
and additionally providers a protocol for ensuring that conflicting versions are avoided.
This is important, for example, for your team to be able to find the version of a presentation that you delivered to a donor,
and also to understand why the significance level of your estimates has changed.
Everyone who has ever encountered a file named something like \texttt{final\_report\_v5\_LJK\_KLE\_jun15.docx}
can appreciate how useful such a system can be.
Most file sharing solutions offer some kind of version control.
These are usually enough to manage changes to binary files (such as Word and PowerPoint documents) without needing to appeal to these dreaded filename-based versioning conventions.
For code files, however, a more complex version control system is usually desirable.
We recommend using Git\sidenote{\textbf{Git:} a multi-user version control system for collaborating on and tracking changes to code as it is written.} for all plain text files.
Git tracks all the changes you make to your code,
and allows you to go back to previous versions without losing the information on changes made.
It also makes it possible to work on two parallel versions of the code,
so you don't risk breaking the code for other team members as you try something new,

Increasingly, we recommend the entire data work folder
to be created and stored separately in GitHub.
Nearly all code and outputs (except datasets) are better managed this way.
Code is written in its native language,
and it's becoming more and more common for written outputs such as reports,
presentations and documentations to be written using different \textbf{literate programming}
tools such as {\LaTeX} and dynamic documents.
You should therefore feel comfortable having both a project folder and a code folder.
Their structures can be managed in parallel by using \texttt{iefolder} twice.
The project folder can be maintained in a synced location like Dropbox,
and the code folder can be maintained in a version-controlled location like GitHub.
Keeping code in a version-controlled folder will allow you
to maintain better control of its history and functionality,
and because of the specificity with which code depends on file structure,
you will be able to enforce better practices there than in the project folder.




\subsection{Code management}

Once you start a project's data work,
the number of scripts, datasets, and outputs that you have to manage will grow very quickly.
This can get out of hand just as quickly,
so it's important to organize your data work and follow best practices from the beginning.
Adjustments will always be needed along the way,
but if the code is well-organized, they will be much easier to make.
Below we discuss a few crucial steps to code organization.
They all come from the principle that code is an output by itself,
not just a means to an end.
So code should be written thinking of how easy it will be for someone to read it later.

Code documentation is one of the main factors that contribute to readability,
if not the main one.
There are two types of comments that should be included in code.
The first one describes what is being done.
This should be easy to understand from the code itself if you know the language well enough and the code is clear.
But writing plain English (or whichever language you communicate with your team on)
will make it easier for everyone to read.
The second type of comment is what differentiates commented code from well-commented code:
it explains why the code is performing a task in a particular way.
As you are writing code, you are making a series of decisions that
(hopefully) make perfect sense to you at the time.
However, you will probably not remember how they were made in a couple of weeks.
So write them down in your code.
There are other ways to document decisions
(GitHub offers a lot of different documentation options, for example),
but information that is relevant to understand the code should always be written in the code itself.

Code organization is the next level.
Start by adding a code header.
This should include simple things such as stating the purpose of the script and the name of the person who wrote it.
If you are using a version control software,
the last time a modification was made and the person who made it will be recorded by that software.
Otherwise, you should include it in the header.
Finally, and more importantly, use it to track the inputs and outputs of the script.
When you are trying to track down which code creates a data set, this will be very helpful.

Breaking your code into readable steps is also good practice on code organization.
One way to do this is to create sections where a specific task is completed.
So, for example, if you want to find the line in your code where a variable was created,
you can go straight to \texttt{PART 2: Create new variables},
instead of reading line by line of the code.
RStudio makes it very easy to create sections, and compiles them into an interactive script index.
In Stata, you can use comments to create section headers,
though they're just there to make the reading easier.
Adding a code index to the header by copying and pasting section titles is the easiest way to create a code map.
You can then add and navigate through them using the find command.
Since Stata code is harder to navigate, as you will need to scroll through the document,
it's particularly important to avoid writing very long scripts.
One reasonable rule of thumb is to not write files that have more than 200 lines.
This is also true for other statistical software,
though not following it will not cause such a hassle.

\codeexample{stata-master-dofile.do}{./code/stata-master-dofile.do}

To bring all these smaller code files together, maintain a master script.
A master script is the map of all your project's data work,
a table of contents for the instructions that you code.
Anyone should be able to follow and reproduce all your work from
raw data to all outputs by simply running this single script.
By follow, we mean someone external to the project who has the master script can
(i) run all the code and recreate all outputs,
(ii) have a general understanding of what is being done at every step, and
(iii) see how codes and outputs are related.
The master script is also where all the settings are established,
such as folder paths, functions and constants used throughout the project.

Agree with your team on a plan to review code as it is written.
Reading other people's code is the best way to improve your coding skills.
And having another set of eyes on your code will make you more comfortable with the results you find.
It's normal (and common) to make mistakes as you write your code quickly.
Reading it again to organize and comment it as you prepare it to be reviewed will help you identify them.
Try to have code review scheduled frequently, as you finish writing a piece of code, or complete a small task.
If you wait for a long time to have your code review, and it gets too long,
preparing it for code review and reviewing them will require more time and work,
and that is usually the reason why this step is skipped.
Making sure that the code is running,
and that other people can understand the code is also the easiest way to ensure a smooth project handover.

\subsection{Output management}

Another task that needs to be discussed with your team is the best way to manage outputs.
A great number of them will be created during the course of a project,
from raw outputs such as tables and graphs to final products such as presentations, papers and reports.
When the first outputs are being created, agree on where to store them,
what software to use, and how to keep track of them.

% Where to store outputs
Decisions about storage of final outputs are made easier by technical constraints.
As discussed above, Git is a great way to control for different versions of
plain text files, and sync software such as Dropbox are better for binary files.
So storing raw outputs in formats like \texttt{.tex} and \texttt{.eps} in Git and
final outputs in PDF, PowerPoint or Word, makes sense.
Storing plain text outputs on Git makes it easier to identify changes that affect results.
If you are re-running all of your code from the master when significant changes to the code are made,
the outputs will be overwritten, and changes in coefficients and number of observations, for example,
will be highlighted.

% What software to use
Though formatted text software such as Word and PowerPoint are still prevalent,
more and more researchers are choosing to write final outputs using
\LaTeX.\sidenote{\url{https://www.latex-project.org}}
{\LaTeX} is a document preparation system that can create both text documents and presentations.
The main difference between them is that {\LaTeX} uses plain text,
and it's necessary to learn its markup convention to use it.
The main advantage of using {\LaTeX} is that you can write dynamic documents,
that import inputs every time they are compiled.
This means you can skip the copying and pasting whenever an output is updated.
Because it's written in plain text, it's also easier to control and document changes using Git.
Creating documents in {\LaTeX} using an integrated writing environment such as TeXstudio
is great for outputs that focus mainly on text,
but include small chunks of code and static code outputs.
This book, for example, was written in \LaTeX.

Another option is to use the statistical software's dynamic document engines.
This means you can write both text (in Markdown) and code in the script,
and the result will usually be a PDF or html file including code,
text and code outputs.
Dynamic document tools are better for including large chunks of code and dynamically created graphs and tables,
but formatting can be trickier.
So it's great for creating appendices,
or quick document with results as you work on them,
but not for final papers and reports.
RMarkdown\sidenote{\url{https://rmarkdown.rstudio.com/}} is the most widely adopted solution in R.
There are also different options for Markdown in Stata,
such as German Rodriguez' \texttt{markstat},\sidenote{\url{https://data.princeton.edu/stata/markdown}}
Stata 15 dynamic documents,\sidenote{\url{https://www.stata.com/new-in-stata/markdown/}}
and Ben Jann's \texttt{webdoc}\sidenote{\url{http://repec.sowi.unibe.ch/stata/webdoc/index.html}} and
\texttt{texdoc}.\sidenote{\url{http://repec.sowi.unibe.ch/stata/texdoc/index.html}}

Whichever options you choose,
agree with your team on what tools will be used for what outputs, and
where they will be stored before you start creating them.
Take into account ease of use for different team members, but
keep in mind that learning how to use a new tool may require some
time investment upfront that will be paid off as your project advances.

% Keeping track of outputs
Finally, no matter what choices you make regarding software and folder organization,
you will need to make changes to your outputs quite frequently.
And anyone who has tried to recreate a graph after a few months probably knows
that it can be hard to remember where you saved the code that created it.
Here, naming conventions and code organization play a key role in not re-writing scripts again and again.
Use intuitive and descriptive names when you save your code.
It's often desirable to have the names of your outputs and scripts linked,
so, for example, \texttt{merge.do} creates \texttt{merged.dta}.
Document output creation in the Master script,
meaning before the line that runs a script there are a few lines of comments listing
data sets and functions that are necessary for it to run,
as well as all outputs created by that script.
When performing data analysis,
it's ideal to write one script for each output,
as well as linking them through name.
This means you may have a long script with ``exploratory analysis'',
just to document everything you have tried.
But as you start to export tables and graphs,
you'll want to save separate scripts, where
\texttt{descriptive\_statistics.do} creates \texttt{descriptive\_statistics.tex}.
