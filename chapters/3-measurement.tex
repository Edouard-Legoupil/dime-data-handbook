%-----------------------------------------------------------------------------------------------

\begin{fullwidth}

In this chapter we will show how you can save a lot of time
and increase the quality of your research by
planning your project's data requirements in advance.
Planning data requirements requires more than
simply listing the key outcome variables.
You need to understand how to structure the project's data
to best answer the research questions,
and create the tools to share this understanding across your team.
We will discuss two types of variables:
variables that tie your research design
to the observations in the  data
which we call \textbf{research variables};
and variables that correspond to observations of the real world,
which we call \textbf{measurement variables}.
The project's data plan needs to account for both.

The first section of this chapter discusses how to
determine your project's data needs,
and introduces DIME's data plan template.
The template includes:
one data linkage table,
one or several master datasets, and
one or several data flowcharts.
These three tools will help to communicate the project's data requirements
both across the team and across time.
This section also discusses what specific research data you need
based on your project's research design,
and how to document those data needs in the data plan.

The second section of this chapter covers two activities where
research data is created by the research team
instead of being observed in the real world.
Those two activities are random sampling and random assignment.
Special focus is spent on how to ensure that
these and other random processes are reproducible,
which is critical for the credibility of your research.

The chapter concludes with a discussion of power calculations and randomization inference,
and how both are important tools to make optimal choices when planning data work.


\end{fullwidth}

%-----------------------------------------------------------------------------------------------

\section{Translating research design to a data plan}

In most projects, more than one data source is needed to answer the research question.
These could be data from multiple survey rounds,
data acquired from different partners (such as administrative data, implementation data, sensor data),
web scraping,
or complex combinations of these and other sources.
However your study is structured, you need to know how to link data from all sources
and analyze the relationships between the units that appear in them
to answer all your research questions.
You might think that you are able to keep all the relevant details in your head,
but your whole research team is unlikely to have the same understanding,
at all times, of all the datasets required.
The only way to make sure that the full team shares the same understanding
is to create a \textbf{data plan}\index{Data plan}.\sidenote{
	\url{https://dimewiki.worldbank.org/Data\_Plan}}
DIME's data plan template has three components:
one \textit{data linkage table},\index{Data linkage table}
one or several \textit{master datasets}\index{Master datasets}
and one or several \textit{data flowcharts}.\index{Data flowchart}

A \textbf{data linkage table}\sidenote{
	\url{https://dimewiki.worldbank.org/Data\_Linkage\_Table}}
lists all the datasets that will be used in the project.
Its most important function is to indicate
how all those datasets can be be linked when
combining information from multiple data sources.
\textbf{Master datasets}\sidenote{
	\url{https://dimewiki.worldbank.org/Master\_Data\_Set}}
list all observations your project ever encounter
and are the authoritative source for all research data
such as unique identifiers, sample status and treatment assignment
(the following two sections of this chapter discuss how to generate these variables).
\textbf{Data flowcharts}\sidenote{
	\url{https://dimewiki.worldbank.org/Data\_Flow\_Chart}}
list all data sources that are needed to create each analysis dataset,
and what manipulation of these data sources is necessary
to get to the final analysis dataset(s),
such as merging, appending, or other linkages.

\subsection{Creating a data plan}

The process of drafting the data plan is itself useful,
as it is an opportunity for the principal investigators
to communicate their vision of the data environment,
and for research assistants to communicate
their understanding of that vision.
The data plan should be drafted at the outset of a project,
before any data is acquired,
but it is not a static document;
it will need to be updated as the project evolves.

To create a data plan according to DIME's template,
the first step is to create a \textbf{data linkage table} by listing
all the data sources you know you will use in a spreadsheet.
If one source of data will result in two different datasets,
then list each dataset on its own row.
For each dataset, list the \textbf{unit of observation}\sidenote{
	\url{https://dimewiki.worldbank.org/Unit\_of\_Observation}},
and the name of the project ID variable for that unit of observation.
Your project should only have one project ID variable per unit of observation.
When you list a dataset in the data linkage table --
which should be done before that dataset is acquired --
you should always make sure that the dataset will
be fully and uniquely identified by the project ID,
or make a plan for how
the new dataset will be linked to the project ID.
It is very labor intensive to work with a dataset that
do not have an unambiguous way to link to the project ID,
and it is a major source of error.

The data linkage table should indicate whether
datasets can be merged one-to-one (for example,
merging baseline and endline datasets
that use the same unit of observation),
or whether two datasets need to be merged many-to-one
(for example, school administrative data merged with student data).
Your data map must indicate which ID variables
can be used -- and how -- when merging datasets.
The data linkage table is also a great place to list other metadata,
such as the source of your data, its backup locations,
the nature of the data license, and so on.

The second step in creating a data plan is to create one \textbf{master dataset}
for each unit of observation
that will be used in any significant research activity.
Examples of such activities are data collection, data analysis,
sampling, and treatment assignment.
The master dataset should include and be the authoritative source of
all \textbf{research variables}\sidenote{
	\textbf{Research variables:} Research data that identifies observations
	and maps research design information to those observations.
	Research variables are time-invariant and
	often, but not always, controlled by the research team.
	Examples include
	ID variables, sampling status, treatment status, and treatment uptake.}
but not include any \textbf{measurement variables}.\sidenote{
	\textbf{Measurement variables:} Data that
	measures attributes or records responses of research subjects.
	Measurement variables are not controlled by the research team
	and often vary over time.
	Examples include characteristics of the research subject,
	outcome variables, input variables among many others.}
Research variables and measurement variables
often come from the same source,
but should not be stored in the same way.
For example, if you acquire administrative data that both includes
information on eligibility to be included in the study (research variable)
and data on the topic of your study (measurement variable)
you should first decide which variables are research variables,
remove them during the data cleaning (see Chapter 6)
and instead store them in your master dataset.
It is common that you will have to update
your master datasets throughout your project.

The most important function of the master dataset
is to be the authoritative source for the project ID.
This means that all observations listed
should be uniquely and fully identified by the included project ID variable.\sidenote{
	\url{https://dimewiki.worldbank.org/ID\_Variable\_Properties}}
You should also list all other identifiers used in your project,
such as names, addresses, or other IDs used by partner organizations,
and the master datasets will then serve as
the linkage between those identifiers and the project ID.
Because of this, master datasets must,
with very few exceptions, always be encrypted.
Even when a partner organization has a unique identifier,
you should always create a project ID specific to your project only,
as you are otherwise not in control over
who can re-identify your de-identified dataset.

You should include all observations ever encountered
in your master datasets,
even if they are not eligible for your study.
This is because, if you ever need to perform a record linkage such as a fuzzy match
on string variables like proper names,
you will make fewer errors the more information you have.
If you ever need to do a fuzzy match,
you should always do that between the master dataset
and the dataset without an unambiguous identifier.
You should not do anything with that dataset until
you have successfully merged
the project IDs from the master dataset.

Since the master datasets is the authoritative source
of the project ID and all research variables,
it serves as an unambiguous method of mapping
the observations in your study to your research design.

The third and final step in creating the data plan is to create \textbf{data flowcharts}.
Each analysis dataset
(see Chapter 6 for discussion on why you likely need multiple analysis datasets)
should have a data flowchart showing how it was created.
The flowchart is a diagram
where each starting point is either a master dataset
or a dataset listed in the data linkage table.
The data flowchart should include instructions on how
the datasets can be combined to create the analysis dataset.
The operations used to combine the data could include:
appending, one-to-one merging,
many-to-one or one-to-many merging, collapsing, or a broad variety of others.
You must list which ID variable or set of ID variables
should be used in each operation,
and note whether the operation creates a new variable or combination of variables
to identify the newly linked data.
Once you have acquired the datasets listed in the flowchart,
you can add to the data flowcharts the number of observations that
the starting point dataset has
and the number of observation each resulting datasets
should have after each operation.
This is a great method to track attrition and to make sure that
the operations used to combine datasets did not create unwanted duplicates
or incorrectly drop any observations.

A data flowchart can be created in a flowchart drawing tool
(there are many free alternatives online) or
by using shapes in Microsoft PowerPoint.
You can also do this simply by drawing on a piece of paper and taking a photo,
but we recommend a digital tool
so that flowcharts can easily be updated over time.

\subsection{Defining research variables related to your research design}

After you have set up your data plan,
you need to carefully think about your research design
and which research variables you will need in the data analysis
to infer the relation between differences in measurement variables
and your research design.
We assume you have a working familiarity
with the research designs mentioned here.
If needed, you can reference \textcolor{red}{Appendix XYZ},
where you will find more details
and specific references for common impact evaluation methods.

As DIME primarily works on impact evaluations,
we focus our discussion here on research designs
that compare a group that received
some kind of \textbf{treatment}\index{Treatment}\sidenote{
	\textbf{Treatment:} The general word for the evaluated intervention or event.
	This includes things like being offered a training,
	a cash transfer from a program,
	or experiencing a natural disaster, among many others.}
against a counterfactual control group\index{Counterfactual}.\sidenote{
	\textbf{Counterfactual:} A statistical description of
	what would have happened
	to specific individuals in an alternative scenario,
	for example, a different treatment assignment outcome.}

The key assumption is that each
person, facility, or village
(or whatever the unit of treatment is)
had two possible states: their outcome if they did receive the treatment
and their outcome if they did not receive that treatment.
The average impact of the treatment, or the ATE\sidenote{
	The \textbf{average treatment effect (ATE)}
	is the expected average change in outcome
	that untreated units would have experienced
	had they been treated.},
is defined as the difference
between these two states averaged over all units.

However, we can never observe the same unit
in both the treated and untreated state simultaneously,
so we cannot calculate these differences directly.
Instead, the treatment group is compared to a control group
that is statistically indistinguishable,
which makes the average impact of the treatment
mathematically equivalent to
the difference in averages between the groups.
Statistical similarity is often defined
as \textbf{balance} between two or more groups.
Since balance tests are commonly run for impact evaluations,
DIME Analytics created a Stata command to
standardize and automate the creation of nicely-formatted balance tables:
\texttt{iebaltab}\sidenote{
	\url{https://dimewiki.worldbank.org/iebaltab}}.

Each research design has a different method for
identifying the statistically-similar control group.
The rest of this section covers how research data requirements
differ between those different methods.
What does not differ, however,
is that these data requirements are all research variables.
The source for the required research variables varies
between research designs and between projects,
but the authoritative source for that type of data should
always be a master dataset.
You will often have to merge
the research variables to other datasets,
but that is an easy task
if you created a data linkage table.


%%%%% Experimental design

In \textbf{experimental research designs},
such as\index{randomized control trials}\index{experimental research designs}
\textbf{randomized control trials (RCTs)},\sidenote{
	\url{https://dimewiki.worldbank.org/Randomized\_Control\_Trials}}
the research team determines which members
of the studied population will receive the treatment.
This is typically done by a randomized process
in which a subset of the eligible population
is randomly assigned to receive the treatment
(see later in this chapter for how to implement this).
The intuition is that if everyone in the eligible population
is assigned at random to either the treatment or control group,
then the two groups will, on average, be statistically indistinguishable.
The treatment will therefore not be correlated with anything
but the impact of that treatment.\cite{duflo2007using}
The randomized assignment should be done
using data from the master dataset,
and the result should be saved back to the master dataset,
before being merged to other datasets.

%%%%% Quasi-experimental design

\textbf{Quasi-experimental research designs},\sidenote{
	\url{https://dimewiki.worldbank.org/Quasi-Experimental\_Methods}}
\index{quasi-experimental research designs}
by contrast, are based on events not controlled by the research team.
Instead, they rely on ``experiments of nature'',
in which natural variation in treatment can be argued to approximate randomization.
You must have a way to measure this natural variation,
and how the variation is categorized as outcomes of a naturally randomized assignment
must be documented in your master dataset.
Unlike carefully planned experimental designs,
quasi-experimental designs typically require the luck
of having access to data collected at the right times and places
to exploit events that occurred in the past.
Therefore, these methods often use either secondary data,
including administrative data or other classes of routinely-collected information,
and it is important that your data linkage table documents
how this data can be linked to the rest of the data in your project.


%%%%% Research variables

No matter the design, you should be very clear about
which data points you observe or collect are research variables.
For example,
\textbf{regression discontinuity (RD)}\sidenote{
	\url{https://dimewiki.worldbank.org/Regression\_Discontinuity}}
\index{regression discontinuity}
designs exploit sharp breaks or limits
in policy designs.
The cutoff determinant, or running variable,
should be saved in your master dataset.
In \textbf{instrumental variables (IV)}\sidenote{
	\url{https://dimewiki.worldbank.org/Instrumental\_Variables}}
\index{instrumental variables}
designs, the \textbf{instruments} influence the \textit{probability} of treatment.
These research variables should be collected and stored in master data.
In \textbf{matching} designs, observations are often grouped
by a strata, grouping, index, or propensity score.\sidenote{
	\url{https://dimewiki.worldbank.org/Matching}}
While you need not include the undelying variables in master data,
the match information itself is part of the experimental design
and should usually be recorded there.
In all these cases, fidelity to the design is important to note as well.
A program intended for students that scored under 50\% on a test
might have some cases where the program is offered to someone that scored 51\% at the test,
or someone that scored 49\% at the might decline to participate in the program.
Differences between assignments and realizations
should also be recorded in the master datasets.

%-----------------------------------------------------------------------------------------------
\subsection{Monitoring data}

For any study with an ex-ante design,
\textbf{monitoring data}\index{monitoring data}\sidenote{\url{
		https://dimewiki.worldbank.org/Monitoring\_Data}}
is very important for understanding if the
assumptions made during the research design
corresponds to what is true in reality.
The most typical example is to make sure that,
in an experimental design,
the treatment was implemented according to your treatment assignment.
While it is always better to monitor all activities,
it might be to costly.
In those cases you can sample a smaller number of critical activities and monitor them.
This will not be detailed enough to be used as a control in your analysis,
but it will still give a way to
estimate the validity of your research design assumptions.

Treatment implementation is often carried out by partners,
and field realities may be more complex realities than foreseen during research design.
Furthermore, the field staff of your partner organization,
might not be aware that their actions are the implementation of your research.
Therefore, you must acquire monitoring data that
tells you how well the treatment assignment in the field
corresponds to your intended treatment assignment,
for nearly all experimental research designs.

Monitoring data is particularly prone to errors
when linking it with the rest of the data in your project.
Often monitoring activities are done by
sending a team to simply record the name of all people attending a training,
or by a partner organization sharing their administrative data,
which is rarely maintained in the same format or structure as your research data.
In both those cases it can be difficult to make sure that
the project ID or any other unambiguous identifiers in our master datasets
is used to record who is who.
Planning ahead for this when the monitoring activity is added to the data linkage table
is the best protection from ending up with poor correlation
between treatment uptake and treatment assignment,
without a way to tell if the poor correlation is just
a result of a fuzzy link between monitoring data and the rest of your data.


%-----------------------------------------------------------------------------------------------
%-----------------------------------------------------------------------------------------------
\section{Research variables created by randomization}

Random sampling and treatment assignment are two research activities
at the core elements of research design,
that generated research variables.
In experimental methods,
random sampling and treatment assignment directly determine
the set of individuals who are going to be observed
and what their status will be for the purpose of effect estimation.
In quasi-experimental methods,
random sampling determines what populations the study
will be able to make meaningful inferences about,
and random treatment assignment creates counterfactuals.
\textbf{Randomization}\sidenote{
	\textbf{Randomization} is often used interchangeably
	to mean random treatment assignment.
	In this book however, \textit{randomization} will only
	be used to describe the process of generating
	a sequence of unrelated numbers, i.e. a random process.
	\textit{Randomization} will never be used to mean
	the process of assigning units in treatment and control groups,
	that will always be called \textit{random treatment assignment},
	or a derivative thereof.}
is used to ensure that a sample is representative and
that any treatment and control groups are statistically indistinguishable
after treatment assignment.

Randomization in statistical software is non-trivial
and its mechanics are unintuitive for the human brain.
The principles of randomization we will outline
apply not just to random sampling and random assignment,
but to all statistical computing processes that have random components
such as simulations and bootstrapping.
Furthermore, all random processes introduce statistical noise
or uncertainty into estimates of effect sizes.
Choosing one random sample from all the possibilities produces some probability of
choosing a group of units that are not, in fact, representative.
Similarly, choosing one random assignment produces some probability of
creating groups that are not good counterfactuals.
\textit{Power calculation} and \textit{randomization inference}
are the main methods by which these probabilities of error are assessed.
These analyses are particularly important in the initial phases of development research --
typically conducted before any data acquisition or field work occurs --
and have implications for feasibility, planning, and budgeting.

%-----------------------------------------------------------------------------------------------
\subsection{Randomizing sampling and treatment assignment}

% sampling universe: the master dataset
\textbf{Sampling} is the process of randomly selecting observations
from a list of individuals to create a representative or statistically similar sub-sample.\index{sampling}
This process can be used, for example, to select a subset from all eligible units
to be included in data collection when the cost of collecting data on everyone is prohibitive.\sidenote{
	\url{https://dimewiki.worldbank.org/Sample\_Size\_and\_Power\_Calculations}}
But it can also be used to select a sub-sample of your observations to test a computationally heavy process
before running it on the full data.
\textbf{Randomized treatment assignment} is the process of assigning observations to different treatment arms.
This process is central to experimental research design.
Most of the code processes used for randomized assignment are the same as those used for sampling,
since it is also a process of randomly splitting a list of observations into groups.
Where sampling determines whether a particular individual
will be observed at all in the course of data collection,
randomized assignment determines if each individual will be observed
as a treatment observation or used as a counterfactual.

The list of units to sample or assign from may be called a \textbf{sampling universe},
a \textbf{listing frame}, or something similar.
This list should always be your \textbf{master dataset} when possible,
and the result should always be saved in the master dataset
before merged to any other data.
One example of the rare exceptions
when master datasets cannot be used is
when sampling must be done in real time --
for example, randomly sampling patients
as they arrive at a health facility.
In those cases,
it is important that you collect enough data
during the real time sampling,
such that you can add these individuals,
and the result of the sampling,
to your master dataset afterwards.

% implement uniform-probability random sampling
The simplest form of sampling is
\textbf{uniform-probability random sampling}.
This means that every eligible observation in the master dataset
has an equal probability of being selected.
The most explicit method of implementing this process
is to assign random numbers to all your potential observations,
order them by the number they are assigned,
and mark as ``sampled'' those with the lowest numbers, up to the desired proportion.
There are a number of shortcuts to doing this process,
but they all use this method as the starting point,
so you should become familiar with exactly how it works.
The do-file below provides an example of
how to implement uniform-probability sampling in practice.
This code uses a Stata built-in dataset and is fully reproducible
(more on reproducible randomization in next section),
so anyone that runs this code in any version of Stata later than 13.1
(the version set in this code)
will get the exact same, but still random, results.

\codeexample{simple-uniform-probability-sampling.do}{./code/simple-uniform-probability-sampling.do}

Sampling typically has only two possible outcomes: observed and unobserved.
Similarly, a simple randomized assignment has two outcomes: treatment and control,
and the logic in the code would be identical to the sampling code example.
However, randomized assignment often involves multiple treatment arms
which each represent different varieties of treatments to be delivered;
in some cases, multiple treatment arms are intended to overlap in the same sample.
Complexity can therefore grow very quickly in randomized assignment
and it is doubly important to fully understand the conceptual process
that is described in the experimental design,
and fill in any gaps before implementing it in code.
The do-file below provides an example of how to implement
a randomized assignment with multiple treatment arms.

\codeexample{simple-multi-arm-randomization.do}{./code/simple-multi-arm-randomization.do}

%-----------------------------------------------------------------------------------------------

\subsection{Programming reproducible random processes}

% what it means for randomization to be reproducible
For statistical programming to be reproducible,
you must be able to re-obtain its exact outputs in the future.\cite{orozco2018make}
We will focus on what you need to do to produce
truly random results for your project,
to ensure you can get those results again.
This takes a combination of strict rules, solid understanding, and careful programming.
This section introduces strict rules:
these are non-negotiable (but thankfully simple).
Stata, like most statistical software, uses a \textbf{pseudo-random number generator}
which, in ordinary research use,
produces sequences of number that are as good as random.\sidenote{
	\url{https://dimewiki.worldbank.org/Randomization\_in\_Stata}}
However, for \textit{reproducible} randomization, we need two additional properties:
we need to be able to fix the sequence of numbers generated and
we need to ensure that the first number is independently randomized.
In Stata, this is accomplished through three concepts:
\textbf{versioning}, \textbf{sorting}, and \textbf{seeding}.
We again use Stata in our examples,
but the same principles translate to all other programming languages.

% rule 1: versioning
\textbf{Versioning} means using the same version of the software each time you run the random process.
If anything is different, the underlying list of random numbers may have changed,
and it may be impossible to recover the original result.
In Stata, the \texttt{version} command ensures that the list of random numbers is fixed.\sidenote{
	At the time of writing, we recommend using \texttt{version 13.1} for backward compatibility;
	the algorithm used to create this list of random numbers was changed after Stata 14 but the improvements do not matter in practice.}
The \texttt{ieboilstart} command in \texttt{ietoolkit} provides functionality to support this requirement.\sidenote{
	\url{https://dimewiki.worldbank.org/ieboilstart}}
We recommend you use \texttt{ieboilstart} at the beginning of your master do-file.\sidenote{
	\url{https://dimewiki.worldbank.org/Master\_Do-files}}
However, testing your do-files without running them
via the master do-file may produce different results,
since Stata's \texttt{version} setting expires after each time you run your do-files.

% rule 2: sorting
\textbf{Sorting} means that the actual data that the random process is run on is fixed.
Because random numbers are assigned to each observation row-by-row starting from
the top row,
changing their order will change the result of the process.
In Stata, the only way to guarantee a unique sorting order is to use
\texttt{isid [id\_variable], sort}.
(The \texttt{sort, stable} command is insufficient.)
Since the exact order must be unchanged,
the underlying data itself must be unchanged as well between runs.
This means that if you expect the number of observations to change
(for example to increase during ongoing data collection),
your randomization will not be reproducible unless you split your data up into
smaller fixed datasets where the number of observations does not change.
You can combine all
those smaller datasets after your randomization.


% rule 3: seeding
\textbf{Seeding} means manually setting the start point in the list of random numbers.
A seed is just a single number that specifies one of the possible start points.
It should be at least six digits long and you should use exactly
one unique, different, and randomly created seed per randomization process.\sidenote{You
	can draw a uniformly distributed six-digit seed randomly by visiting \url{https://bit.ly/stata-random}.
	(This link is a just shortcut to request such a random number on \url{https://www.random.org}.)
	There are many more seeds possible but this is a large enough set for most purposes.}
In Stata, \texttt{set seed [seed]} will set the generator
to the start point identified by the seed.
In R, the \texttt{set.seed} function does the same.
To be clear: you should not set a single seed once in the master do-file,
but instead you should set a new seed in code right before each random process.
The most important thing is that each of these seeds is truly random,
so do not use shortcuts such as the current date or a seed you have used before.
You should also describe in your code how the seed was selected.

% testing randomization reproducibility
Other commands may induce randomness in the data,
change the sorting order,
or alter the place of the random generator without you realizing it,
so carefully confirm exactly how your code runs before finalizing it.
To confirm that a randomization has worked correctly before finalizing its results,
save the outputs of the process in a temporary location,
re-run the code, and use \texttt{cf} or \texttt{datasignature} to ensure
nothing has changed. It is also advisable to let someone else reproduce your
randomization results on their machine to remove any doubt that your results
are reproducible.
Once the result of a randomization is used in the field,
there is no way to correct any mistakes.

\codeexample{reproducible-randomization.do}{./code/reproducible-randomization.do}

Some types of experimental designs require
that randomized assignment results be revealed in the field.
It is possible to do this using survey software or live events, such as a live lottery.
These methods typically do not leave a record of the randomization,
and as such are never reproducible.
However, you can often run your randomization in advance
even when you do not have list of eligible units in advance.
Let's say you want to, at various health facilities,
randomly select a sub-sample of patients as they arrive.
You can then have a pre-generated list
with a random order of ''in sample'' and ''not in sample''.
Your field staff would then go through this list in order
and cross off one randomized result as it is used for a patient.

This is especially beneficial if you are implementing a more complex randomization,
for example, sample 10\% of the patients, show a video for 50\% of the sample,
and ask a longer version of the questionnaire to 20\% of both
the group of patients that watch the video and those that did not.
The real time randomization is much more likely to be implemented correctly,
if your field staff simply can follow a list with the randomized categories
where you are in control fo the pre-determined proportions and the random order.
This way, you can also control with precision,
how these categories are evenly distributed across all health facilities.

Finally, if this real-time randomization implementation is done using survey software,
then the pre-generated list of randomized categories can be preloaded
into the questionnaire.
Then the field team can follow a list of respondent IDs
that are randomized into the appropriate categories,
and the survey software can show a video and control which version of the questionnaire is asked.
This way, you reduce the risk of errors in field randomization.


%-----------------------------------------------------------------------------------------------

\subsection{Clustering or stratifying a random sample or assignment}

% the cases discussed so far are the most simple, but not the most common
For a variety of reasons, random sampling and random treatment assignment
are rarely as straightforward as a uniform-probability draw.
The most common variants are \textbf{clustering} and \textbf{stratification}.\cite{athey2017econometrics}
\textbf{Clustering} occurs when your unit of analysis is different
than the unit of sampling or treatment assignment.\sidenote{
	\url{https://dimewiki.worldbank.org/Unit\_of\_Observation}}
For example, a policy may be implemented at the village level,
or you may only be able to send enumerators to a limited number of villages,
or the outcomes of interest for the study are measured at the household level.\sidenote{
	\url{https://dimewiki.worldbank.org/Multi-stage\_(Cluster)\_Sampling}}
\index{clustered randomization}
The groups in which units are assigned to treatment are called clusters.
% what is stratification
\textbf{Stratification} breaks the full set of observations into subgroups
before performing randomized assignment within each subgroup.\sidenote{
	\url{https://dimewiki.worldbank.org/Stratified\_Random\_Sample}}
\index{stratification}
The subgroups are called \textbf{strata}.
This ensures that members of every subgroup
are included in all groups of the randomized assignment process,
or that members of all groups are observed in the sample.
Without stratification, it is possible that the randomization
would put all the members of a subgroup into just one of the treatment arms,
or fail to select any of them into the sample.
For both clustering and stratification,
implementation is nearly identical in both sampling and randomized assignment.

% How to implement randomization with clusters
Clustering is procedurally straightforward in Stata,
although it typically needs to be performed manually.
To cluster a sampling or randomized assignment,
randomize on the master dataset for the unit of observation of the cluster,
and then merge the results to the master dataset for the unit of analysis.
This is a many-to-one merge and your data map should document
how those datasets can be merged correctly.
If you do not have a master dataset yet for the unit of observation of the cluster,
then you first have to create that and update your data map accordingly.
When sampling or randomized assignment is conducted using clusters,
the clustering variable should be clearly identified in the master dataset
for the unit of analysis
since it will need to be used in subsequent statistical analysis.
Namely, standard errors for these types of designs must be clustered
at the level at which the randomization was clustered.\sidenote{
	\url{https://blogs.worldbank.org/impactevaluations/when-should-you-cluster-standard-errors-new-wisdom-econometrics-oracle}}
This accounts for the design covariance within the cluster --
the information that if one individual was observed or treated there,
the other members of the clustering group were as well.

% using randtreat for stratified randomized assignment
By contrast, implementing stratified designs in statistical software is prone to error.
Even for a typical multi-armed design, the basic method of randomly ordering the observations
will often create very skewed assignments.\sidenote{\url
	{https://blogs.worldbank.org/impactevaluations/tools-of-the-trade-doing-stratified-randomization-with-uneven-numbers-in-some-strata}}
The user-written \texttt{randtreat} command properly implements stratification.\cite{carril2017dealing}
The options and outputs (including messages) from the command should be carefully reviewed
so that you understand exactly what has been implemented.
Notably, it is extremely hard to target precise numbers of observations
in stratified designs, because exact allocations are rarely round fractions
and the process of assigning the leftover ``misfit'' observations
imposes an additional layer of randomization above the specified division.
Strata variables must also be included in your master datasets.

%-----------------------------------------------------------------------------------------------

\section{Doing power calculations for research design}

% sampling error, randomization noise and the need for power calcs
Random sampling and treatment assignment are noisy processes:
it is impossible to predict the result in advance.
By design, we know that the exact choice of sample or treatment
will be uncorrelated with our key outcomes,
but this lack of correlation is only true ``in expectation'' --
that is, across a large number of randomizations.
In any \textit{particular} randomization,
the correlation between the sampling or randomized assignments and the outcome variable
is guaranteed to be \textit{nonzero}:
this is called the \textbf{in-sample} or \textbf{finite-sample correlation}.

Since we know that the true correlation
(over the ``population'' of potential samples or randomized assignments)
is zero, we think of the observed correlation as an \textbf{error}.
In sampling, we call this the \textbf{sampling error},
and it is defined as the difference between a true population parameter
and the observed mean due to chance selection of units.\sidenote{
	\url{https://economistjourney.blogspot.com/2018/06/what-is-sampling-noise.html}}
In randomized assignment, we call this the \textbf{randomization noise},
and define it as the difference between a true treatment effect
and the estimated effect due to placing units in groups.
The intuition for both measures is that from any group,
you can find some possible subsets that have higher-than-average values of some measure;
similarly, you can find some that have lower-than-average values.
Your random sample or treatment assignment will fall in one of these categories,
and we need to assess the likelihood and magnitude of this occurrence.\sidenote{
	\url{https://davegiles.blogspot.com/2019/04/what-is-permutation-test.html}}
\textit{Power calculation} and \textit{randomization inference} are the two key tools to doing so.

% why to do power calculations
\textbf{Power calculations} report the likelihood that your experimental design
\index{power calculations}
will be able to detect the treatment effects you are interested in
given these sources of noise.\sidenote{
	\url{https://dimewiki.worldbank.org/Sample\_Size\_and\_Power\_Calculations}}
This measure of \textbf{power} can be described in various different ways,
each of which has different practical uses.\sidenote{
	\url{https://www.stat.columbia.edu/~gelman/stuff\_for\_blog/chap20.pdf}}
The purpose of power calculations is to identify where the strengths and weaknesses
of your design are located, so you know the relative tradeoffs you will face
by changing your randomization scheme for the final design.

The \textbf{minimum detectable effect (MDE)}\sidenote{
	\url{https://dimewiki.worldbank.org/Minimum\_Detectable\_Effect}}
is the smallest true effect that a given research design can reliably detect.
This is useful as a check on whether a study is worthwhile.
If, in your field, a ``large'' effect is just a few percentage points
or a small fraction of a standard deviation,
then it is nonsensical to run a study whose MDE is much larger than that.
This is because, given the sample size and variation in the population,
the effect needs to be much larger to possibly be statistically detected,
so such a study would never be able to say anything about the effect size that is practically relevant.
Conversely, the \textbf{minimum sample size} pre-specifies expected effect sizes
and tells you how large a study's sample would need to be to detect that effect,
which can tell you what resources you would need
to implement a useful study.

% what is randomization inference
\textbf{Randomization inference}\sidenote{
	\url{https://dimewiki.worldbank.org/Randomization\_Inference}}
is used to analyze the likelihood
\index{randomization inference}
that the randomized assignment process, by chance,
would have created a false treatment effect as large as the one you observed.
Randomization inference is a generalization of placebo tests,
because it considers what the estimated results would have been
from a randomized assignment that did not in fact happen in reality.
Randomization inference is particularly important
in quasi-experimental designs and in small samples,
where the number of possible \textit{randomizations} is itself small.
Randomization inference can therefore be used proactively during experimental design,
to examine the potential spurious treatment effects your exact design is able to produce.
If there is significant heaping at particular result levels,
or if results seem to depend dramatically on the outcomes for a small number of individuals,
randomization inference will flag those issues before the experiment is fielded
and allow adjustments to the design to be made.
