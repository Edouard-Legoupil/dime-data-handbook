%------------------------------------------------

\begin{fullwidth}
Transforming raw data into a substantial contribution to scientific knowledge 
requires a mix of subject expertise, programming skills, 
and statistical and econometric knowledge. 
The process of data analysis is, therefore, 
a back-and-forth discussion between people 
with differing skill sets. 
The research assistant usually ends up being the pivot of this discussion. 
It is their job to translate the data received from the field into
economically meaningful indicators and to analyze them 
while making sure that code and outputs do not become too difficult to follow or get lost over time.

When it comes to code, though, analysis is the easy part, 
\textit{as long as you have organized your data well}. 
Of course, there is plenty of complexity behind it: 
the econometrics, the theory of change, the measurement methods, and so much more.
But none of those are the subject of this book. 
\textit{Instead, this chapter will focus on how to organize your data work so that coding the analysis becomes easy}.
Most of a Research Assistant's time is spent cleaning data and getting it into the right format. 
When the practices recommended here are adopted,
analyzing the data is as simple as using a command that is already implemented in a statistical software. 


\end{fullwidth}

%------------------------------------------------

\section{Data management}
The goal of data management is to organize the components of data work 
so it can traced back and revised without massive effort.
In our experience, there are four key elements to good data management: 
folder structure, task breakdown, master scripts, and version control. 
A good folder structure organizes files so that any material can be found when needed.
It reflects a task breakdown into steps with well-defined inputs, tasks, and outputs.
This breakdown is applied to code, data sets, and outputs.
A master script connects folder structure and code.
It is a one-file summary of your whole project.
Finally, version histories and backups enable the team 
to edit files without fear of losing information.
Smart use of version control also allows you to track 
how each edit affects other files in the project.

% Task breakdown
We divide the process of turning raw data into analysis data into three stages: 
data cleaning, variable construction, and data analysis. 
Though they are frequently implemented at the same time, 
we find that creating separate scripts and data sets prevents mistakes. 
It will be easier to understand this division as we discuss what each stage comprises. 
What you should know by now is that each of these stages has well-defined inputs and outputs. 
This makes it easier to track tasks across scripts, 
and avoids duplication of code that could lead to inconsistent results. 
For each stage, there should be a code folder and a corresponding data set. 
The names of codes, data sets and outputs for each stage should be consistent,
making clear how they relate to one another. 
So, for example, a script called \texttt{clean-section-1} would create
a data set called \texttt{cleaned-section-1}.

The division of a project in stages also helps the review workflow inside your team.
The code, data and outputs of each of these stages should go through at least one round of code review.
During the code review process, team members should read and run each other's codes.
Doing this at the end of each stage helps prevent the amount of work to be reviewed to become too overwhelming.
Code review is a common quality assurance practice among data scientists.
It helps to keep the level of the outputs high, and is also a great way to learn and improve your code.

% Folder structure
There are many schemes to organize research data. 
Our preferred scheme reflects the task breakdown just discussed.
\index{data organization}
DIME Analytics created the \texttt{iefolder}\sidenote{
	\url{https://dimewiki.worldbank.org/wiki/iefolder}}
package (part of \texttt{ietoolkit}\sidenote{
	\url{https://dimewiki.worldbank.org/wiki/ietoolkit}})
to standardize folder structures across teams and projects.
This means that PIs and RAs face very small costs when switching between projects, 
because they are organized in the same way.\sidenote{\url{https://dimewiki.worldbank.org/wiki/DataWork\_Folder}}
We created the command based on our experience with primary data,
but it can be used for different types of data.
Whatever you team may need in terms of organization, 
the principle of creating one standard remains.

At the first level of the structure created by \texttt{iefolder} are what we call survey round folders.\sidenote{\url{https://dimewiki.worldbank.org/wiki/DataWork\_Survey\_Round}}
You can think of a ``round'' as one source of data, 
that will be cleaned in the same script. 
Inside round folders, there are dedicated folders for 
raw (encrypted) data; de-identified data; cleaned data; and final (constructed) data. 
There is a folder for raw results, as well as for final outputs. 
The folders that hold code are organized in parallel to these, 
so that the progression through the whole project can be followed by anyone new to the team.  
Additionally, \texttt{iefolder} creates \textbf{master do-files}\sidenote{\url{https://dimewiki.worldbank.org/wiki/Master\_Do-files}} 
so all project code is reflected in a top-level script.

% Master scripts
Master scripts allow users to execute all the project code from a single file.
They briefly describes what each code, 
and maps the files they require and create. 
They also connects code and folder structure through globals or objects. 
In short, a master script is a human-readable map to the tasks, 
files and folder structure that comprise a project.  
Having a master script eliminates the need for complex instructions to replicate results. 
Reading the master do-file should be enough for anyone unfamiliar with the project
to understand what are the main tasks, which scripts execute them,
and where different files can be found in the project folder. 
That is, it should contain all the information needed to interact with a project's data work.

% Version control
Finally, everything that can be version-controlled should be. 
Version control allows you to effectively track code edits,
including the addition and deletion of files. 
This way you can delete code you no longer need, 
and still recover it easily if you ever need to get back previous work.
Both analysis results and data sets will change with the code.
You should have each of them stored with the code that created it.
If you are writing code in Git/GitHub,
you can output plain text files such as \texttt{.tex} tables
and metadata saved in \texttt{.txt} or \texttt{.csv} to that directory.
Binary files that compile the tables,
as well as the complete data sets, on the other hand,
should be stored in your team's shared folder. 
Whenever data cleaning or data construction codes are edited,
use the master script to run all the code for your project.
Git will highlight the changes that were in data sets and results that they entail. 

%------------------------------------------------

\section{Data cleaning}

% intro: what is data cleaning -------------------------------------------------
Data cleaning is the first stage of transforming the data you received from the field into data that you can analyze.\sidenote{\url{https://dimewiki.worldbank.org/wiki/Data\_Cleaning}}
The cleaning process involves (1) making the data set easily usable and understandable, 
and (2) documenting individual data points and patterns that may bias the analysis.
The underlying data structure does not change.
The cleaned data set should contain only the variables collected in the field.
No modifications to data points are made at this stage, except for corrections of mistaken entries.

Cleaning is probably the most time consuming of the stages discussed in this chapter.
This is the time when you obtain an extensive understanding of  the contents and structure of the data that was collected.
Explore your data set using tabulations, summaries, and descriptive plots.
You should use this time to understand the types of responses collected, both within each survey question and across respondents.
Knowing your data set well will make it possible to do analysis.


% Deidentification ------------------------------------------------------------------
The initial input for data cleaning is the raw data.
It should contain only materials that are received directly from the field.
They will invariably come in a host of file formats and nearly always contain personally-identifying information.\index{personally-identifying information}
These files should be retained in the raw data folder \textit{exactly as they were received}.
Be mindful of where this file is stored. 
Maintain a backup copy in a secure offsite location.
Every other file is created from the raw data, and therefore can be recreated.
The exception, of course, is the raw data itself, so it should never be edited 
directly.
The rare and only exception to this is if the raw data is encoded incorrectly
and some non-English letter is causing rows or columns to break incorrectly
when the data is imported. In this rare scenario you will have to fix this 
manually, and then both the broken and the fixed version of the raw data should 
be securely backed up.
Additionally, no one who is not listed in the IRB should be able to access its content, not even the company providing file-sharing services.
Check if your organization has guidelines on how to store data securely, as they may offer an institutional solution. 
If that is not the case, you will need to encrypt the data, especially before
sharing it, and make sure that only IRB-listed team members have the
encryption key.\sidenote{\url{https://dimewiki.worldbank.org/wiki/Encryption}}.

Secure storage of the raw data means access to it will be restricted even inside the research team.\sidenote{\url{https://dimewiki.worldbank.org/wiki/Data\_Security}}
Loading encrypted data frequently can be disruptive to the workflow.
To facilitate the handling of the data, remove any personally identifiable information from the data set.
This will create a de-identified data set, that can be saved in a non-encrypted folder. 
De-identification,\sidenote{\url{https://dimewiki.worldbank.org/wiki/De-identification}}
at this stage, means stripping the data set of direct identifiers such as names, phone numbers, addresses, and geolocations.\sidenote{\url{ https://www.povertyactionlab.org/sites/default/files/resources/J-PAL-guide-to-deidentifying-data.pdf}}
The resulting de-identified data will be the underlying source for all cleaned and constructed data.
Because identifying information is typically only used during data collection, 
to find and confirm the identity of interviewees, 
de-identification should not affect the usability of the data.
In fact, most identifying information can be converted into non-identified variables for analysis purposes
(e.g. GPS coordinates can be translated into distances). 
However, if sensitive information is strictly needed for analysis,
all the tasks described in this chapter must be performed in a secure environment.
What that means for a specific project will depend on IRB conditions,
but a few examples are company-managed machines, 
servers accessed through two-factor-authentication,
or even cold rooms.

% Unique ID and data entry corrections ---------------------------------------------
There are two main cases when the raw data will be modified during data cleaning.
The first one is when there are duplicated entries in the data.
Ensuring that observations are uniquely and fully identified\sidenote{\url{https://dimewiki.worldbank.org/wiki/ID\_Variable\_Properties}}
is possibly the most important step in data cleaning.
Modern survey tools create unique observation identifiers.
That, however, is not the same as having a unique ID variable for each individual in the sample.
You want to make sure the data set has a unique ID variable
that can be cross-referenced with other records, such as the Master Data Set\sidenote{\url{https://dimewiki.worldbank.org/wiki/Master\_Data\_Set}}
and other rounds of data collection.
\texttt{ieduplicates} and \texttt{iecompdup}, 
two Stata commands included in the \texttt{iefieldkit} 
package\index{iefieldkit},\sidenote{\url{https://dimewiki.worldbank.org/wiki/iefieldkit}}
create an automated workflow to identify, correct and document
occurrences of duplicate entries. 

Looking for duplicated entries is usually part of data quality monitoring,
as is the only other reason to change the raw data during cleaning:
correcting mistakes in data entry.
During data quality monitoring, you will inevitably encounter data entry mistakes,
such as typos and inconsistent values.
These mistakes should be fixed in the cleaned data set,
and you should keep a careful record of how they were identified,
and how the correct value was obtained.

% Data description ------------------------------------------------------------------
On average, making corrections to primary data is more time-consuming than when using secondary data.
But you should always check for possible issues in any data you are about to use.
The last step of data cleaning, however, will most likely still be necessary.
It consists of annotating the data, so that its users have all the information needed to interact with it.
This is a key step to making the data easy to use, but it can be quite repetitive.
The \texttt{iecodebook} command suite, also part of \texttt{iefieldkit},
is designed to make some of the most tedious components of this process,
such as renaming, relabeling, and value labeling, much easier.\sidenote{\url{https://dimewiki.worldbank.org/wiki/iecodebook}}
\index{iecodebook}
We have a few recommendations on how to use this command for data cleaning.
First, we suggest keeping the same variable names in the cleaned data set as in the survey instrument, so it's straightforward to link data points for a variable to the question that originated them.
Second, don't skip the labeling.
Applying labels makes it easier to understand what the data is showing while exploring the data. 
This minimizes the risk of small errors making their way through into the analysis stage.
Variable and value labels should be accurate and concise.\sidenote{\url{https://dimewiki.worldbank.org/wiki/Data\_Cleaning\#Applying\_Labels}}
Third, recodes should be used to turn codes for ``Don't know'', ``Refused to answer'', and
other non-responses into extended missing values.\sidenote{\url{https://dimewiki.worldbank.org/wiki/Data\_Cleaning\#Survey\_Codes\_and\_Missing\_Values}}
String variables need to be encoded, and open-ended responses, categorized or dropped\sidenote{\url{https://dimewiki.worldbank.org/wiki/Data\_Cleaning\#Strings}}
(unless you are using qualitative or classification analyses, which are less common).
Finally, any additional information collected only for quality monitoring purposes,
such as notes and duration fields, can also be dropped.

% Outputs -----------------------------------------------------------------

% Data set
The most important output of data cleaning is the cleaned data set. 
It should contain the same information as the raw data set,
with no changes to data points.
It should also be easily traced back to the survey instrument,
and be accompanied by a dictionary or codebook.
Typically, one cleaned data set will be created for each data source,
i.e. per survey instrument.
Each row in the cleaned data set represents one survey entry or unit of observation.\sidenote{\cite{tidy-data}}
If the raw data set is very large, or the survey instrument is very complex,
you may want to break the data cleaning into sub-steps, 
and create intermediate cleaned data sets
(for example, one per survey module).
Breaking cleaned data sets into the smallest unit of observation inside a roster
make the cleaning faster and the data easier to handle during construction.
But having a single cleaned data set will help you with sharing and publishing the data.
To make sure this file doesn't get too big to be handled,
use commands such as \texttt{compress} in Stata to make sure the data
is always stored in the most efficient format.

% Documentation
Throughout the data cleaning process, you will need inputs from the field, 
including enumerator manuals, survey instruments, 
supervisor notes, and data quality monitoring reports.
These materials are part of what we call data documentation
\sidenote{\url{https://dimewiki.worldbank.org/wiki/Data\_Documentation}}
\index{Documentation},
and should be stored in the corresponding folder, 
as you will probably need them during analysis and publication.
Include in the \texttt{Documentation} folder records of any
corrections made to the data, including to duplicated entries,
as well as communications from the field where theses issues are reported.
Make sure to also have a record of potentially problematic patterns you noticed
while exploring the data, such as outliers and variables with many missing values.
Be very careful not to include sensitive information in 
documentation that is not securely stored, 
or that you intend to release as part of a replication package or data publication.

\section{Indicator construction}

% What is construction -------------------------------------
Data construction is the process of processing the data points as provided in the raw data to make them suitable for analysis.
It is at this stage that the raw data is transformed into analysis data.
This is done by creating derived variables
(binaries, indices, and interactions, to name a few).
To understand why construction is necessary,
let's take the example of a household survey's consumption module.
It will result in separate variables indicating the 
amount of each item in the bundle that was consumed.
There may be variables indicating the cost of these items.
You cannot run a meaningful regression on these variables. 
You need to manipulate them into something that has \textit{economic} meaning. 
During this process, the data points will typically be reshaped and aggregated 
so that level of the data set goes from the unit of observation in the survey to the unit of analysis.\sidenote{\url{https://dimewiki.worldbank.org/wiki/Unit\_of\_Observation}} 
To use the same example, the data on quantity consumed was collect for each item, and needs to be aggregated to the household level before analysis.

% Why it is a separate process -------------------------------

% From cleaning
Construction is done separately from data cleaning for two reasons. 
The first one is to clear differentiation between the data originally collected and the result of data processing decisions.
The second is to ensure that variable definition is consistent across data sources. 
Unlike cleaning, construction can create many outputs from many inputs. 
Let's take the example of a project that has a baseline and an endline survey. 
Unless the two instruments are exactly the same, which is preferable but often not the case,  the data cleaning for them will require different steps, and therefore will be done separately. 
However, you still want the constructed variables to be calculated in the same way, so they are comparable.
So you want to construct indicators for both rounds in the same code, after merging them.

% From analysis
Data construction is never a finished process.
It comes ``before'' data analysis only in a limited sense: the construction code must be run before the analysis code.
Typically, however, construction and analysis code are written concurrently.
As you write the analysis, different constructed variables will become necessary, as well as subsets and other alterations to the data.
Still, constructing variables in a separate script from the analysis will help you ensure consistency across different outputs. 
If every script that creates a table starts by loading a data set, subsetting it and manipulating variables, any edits to construction need to be replicated in all scripts. 
This increases the chances that at least one of them will have a different sample or variable definition.


% What to do during construction -----------------------------------------
Keep in mind that details matter when constructing variables, and overlooking them may affect your results. 
It is important to check and double-check the value-assignments of questions and their scales before constructing new variables using them.
Are they in percentages or proportions? 
Are all variables you are combining into an index or average using the same scale? 
Are yes or no questions coded as 0 and 1, or 1 and 2?
This is when you will use the knowledge of the data you acquired and the documentation you created during the cleaning step the most.
It is often useful to start looking at comparisons and other documentation 
outside the code editor.

Adding comments to the code explaining what you are doing and why is crucial here.
There are always ways for things to go wrong that you never anticipated, but two issues to pay extra attention to are missing values and dropped observations. 
If you are subsetting a data set, you should drop observations explicitly, indicating why you are doing that and how the data set changed.
Merging, reshaping and aggregating data sets can change both the total number of observations and the number of observations with missing values.
Make sure to read about how each command treats missing observations and, whenever possible, add automated checks in the script that throw an error message if the result is changing.

At this point, you will also need to address some of the issues in the data that you identified during data cleaning. 
The most common of them is the presence of outliers.
How to treat outliers is a research question, but make sure to note what we the decision made by the research team, and how you came to it. 
Results can be sensitive to the treatment of outliers, so keeping the original variable in the data set will allow you to test how much it affects the estimates.
More generally, create derived measures in new variables instead of overwriting the original information.
Because data construction involves translating concrete data points to more abstract measurements, it is important to document exactly how each variable is derived or calculated.

% Outputs -----------------------------------------------------------------

% Data set
The outputs of construction are the data sets that will be used for analysis.
The level of observation of a constructed data set is the unit analysis. 
Each data set is purpose-built to answer an analysis question.
Since different pieces of analysis may require different samples,
or even different units of observation,
you may have one or multiple constructed data sets, 
depending on how your analysis is structured.
So don't worry if you cannot create a single, ``canonical'' analysis data set.
It is common to have many purpose-built analysis datasets.
Think of an agricultural intervention that was randomized across villages and only affected certain plots within each village. 
The research team may want to run household-level regressions on income, test for plot-level productivity gains, and check if village characteristics are balanced.
Having three separate datasets for each of these three pieces of analysis will result in much cleaner do files than if they all started from the same file. 
 
One thing all constructed data sets should have in common, though, are functionally-named variables.
Constructed variables are called ``constructed'' because they were not present in the survey to start with,
so making their names consistent with the survey form is not as crucial.
Of course, whenever possible, having variables names that are both intuitive and can be linked to the survey is ideal.
However, functionality should be prioritized here.
Remember to consider keeping related variables together and adding notes to each as necessary.

% Documentation
It is wise to start an explanatory guide as soon as you start making changes to the data.
Carefully record how specific variables have been combined, recoded, and scaled. 
This can be part of a wider discussion with your team about creating protocols for variable definition.
That will guarantee that indicators are defined consistently across projects.
Documentation is an output of construction as relevant as the code and the data.
Someone unfamiliar with the project should be able to understand the contents of the analysis data sets, the steps taken to create them, and the decision-making process through your documentation.
The construction documentation will complement the reports and notes created during data cleaning.
Together, they will form a detailed account of the data processing.



%------------------------------------------------

\section{Writing data analysis code}

% Intro --------------------------------------------------------------
Data analysis is the stage when research outputs are created. 
\index{data analysis}
Many introductions to common code skills and analytical frameworks exist, such as
\textit{R for Data Science};\sidenote{\url{https://r4ds.had.co.nz/}}
\textit{A Practical Introduction to Stata};\sidenote{\url{https://scholar.harvard.edu/files/mcgovern/files/practical\_introduction\_to\_stata.pdf}}
\textit{Mostly Harmless Econometrics};\sidenote{\url{https://www.researchgate.net/publication/51992844\_Mostly\_Harmless\_Econometrics\_An\_Empiricist's\_Companion}} 
and \textit{Causal Inference: The Mixtape}.\sidenote{\url{http://scunning.com/mixtape.html}}
This section will not include instructions on how to conduct specific analyses.
That is a research question, and requires expertise beyond the scope of this book.
Instead, we will outline the structure of writing analysis code,
assuming you have completed the process of data cleaning and construction.

% Exploratory and final data analysis -----------------------------------------
The analysis stage usually starts with a process we call exploratory data analysis.
This is when you are trying different things and looking for patterns in your data. 
It progresses into final analysis when your team starts to decide what are the main results, those that will make it into the research output.
The way you deal with code and outputs for exploratory and final analysis is different, and this section will discuss how.
% Organizing scripts ---------------------------------------------------------
During exploratory data analysis, you will be tempted to write lots of analysis into one big, impressive, start-to-finish script. 
It subtly encourages poor practices such as not clearing the workspace and not reloading the constructed data set before each analysis task. 
It's important to take the time to organize scripts in a clean manner and to avoid mistakes.

A well-organized analysis script starts with a completely fresh workspace and explicitly loads data before analyzing it.
This encourages data manipulation to be done earlier in the workflow (that is, during construction).
It also and prevents you from accidentally writing pieces of analysis code that depend on one another and requires manual instructions for all required code snippets be run in the right order.
Each script should run completely independently of all other code.
You can go as far as coding every output in a separate script.
There is nothing wrong with code files being short and simple -- as long as they directly correspond to specific pieces of analysis.

Analysis files should be as simple as possible, so whoever is reading it can focus on the econometrics.
All research decisions should be made very explicit in the code.
This includes clustering, sampling, and control variables, to name a few. 
If you have multiple analysis data sets, each of them should have a descriptive name about its sample and unit of observation.
As your team comes to a decision about model specification, you can create globals or objects in the master script to use across scripts.
This is a good way to make sure specifications are consistent throughout the analysis. 
Using pre-specified globals or objects also makes your code more dynamic, so it is easy to update specifications and results without changing every script.
It is completely acceptable to have folders for each task, and compartmentalize each analysis as much as needed.
It is always better to have more code files open than to keep scrolling inside a given file.

To accomplish this, you will need to make sure that you have an effective data management system, including naming, file organization, and version control.
Just like you did with each of the analysis datasets, name each of the individual analysis files descriptively.
Code files such as \path{spatial-diff-in-diff.do}, \path{matching-villages.R}, and \path{summary-statistics.py} 
are clear indicators of what each file is doing, and allow you to find code quickly.
If you intend to numerically order the code as they appear in a paper or report, 
leave this to near publication time.

% Self-promotion ------------------------------------------------
Our team has created a few products to automate common outputs and save you 
precious research time.
The \texttt{ietoolkit} package includes two commands to export nicely formatted tables.
\texttt{iebaltab}\sidenote{\url{https://dimewiki.worldbank.org/wiki/Iebaltab}} creates and exports balance tables to excel or {\LaTeX}. 
\texttt{ieddtab}\sidenote{\url{https://dimewiki.worldbank.org/wiki/Ieddtab}} does the same for difference-in-differences regressions.
The \textbf{Stata Visual Library}\sidenote{	\url{https://worldbank.github.io/Stata-IE-Visual-Library/}}
has examples of graphs created in Stata and curated by us.\sidenote{A similar resource for R is \textit{The R Graph Gallery}. \\\url{https://www.r-graph-gallery.com/}}
\textbf{Data visualization}\sidenote{\url{https://dimewiki.worldbank.org/wiki/Data\_visualization}} \index{data visualization}
is increasingly popular, but a great deal lacks in quality.\cite{healy2018data,wilke2019fundamentals}
We attribute some of this to the difficulty of writing code to create them.
Making a visually compelling graph would already be hard enough if you didn't have to go through many rounds of googling to understand a command.
The trickiest part of using plot commands is to get the data in the right format.
This is why the \textbf{Stata Visual Library} includes example data sets to use 
with each do-file.

Whole books have been written on how to create good data visualizations,
so we will not attempt to give you advice on it.
Rather, here are a few resources we have found useful.
The Tapestry conference focuses on ``storytelling with data''.\sidenote{
	\url{https://www.youtube.com/playlist?list=PLb0GkPPcZCVE9EAm9qhlg5eXMgLrrfMRq}}
\textit{Fundamentals of Data Visualization} provides extensive details on practical application;\sidenote{
	\url{https://serialmentor.com/dataviz}}
as does \textit{Data Visualization: A Practical Introduction}.\sidenote{
	\url{http://socvis.co}}
Graphics tools like Stata are highly customizable.
There is a fair amount of learning curve associated with extremely-fine-grained adjustment,
but it is well worth reviewing the graphics manual\sidenote{\url{https://www.stata.com/manuals/g.pdf}}
For an easier way around it, Gray Kimbrough's \textit{Uncluttered Stata Graphs} code is an excellent default replacement for Stata graphics that is easy to install.
\sidenote{\url{https://graykimbrough.github.io/uncluttered-stata-graphs/}}
If you are a R user, the \textit{R Graphics Cookbook}\sidenote{\url{https://r-graphics.org/}} 
is a great resource for the most popular visualization package \texttt{ggplot}\sidenote{\url{https://ggplot2.tidyverse.org/}}. 
But there are a variety of other visualization packages, 
such as \texttt{highcharter}\sidenote{\url{http://jkunst.com/highcharter/}}, 
\texttt{r2d3}\sidenote{\url{https://rstudio.github.io/r2d3/}}, 
\texttt{leaflet}\sidenote{\url{https://rstudio.github.io/leaflet/}}, 
and \texttt{plotly}\sidenote{\url{https://plot.ly/r/}}, to name a few.
We have no intention of creating an exhaustive list, and this one is certainly missing very good references.
But at least it is a place to start.

\section{Exporting analysis outputs}

It's ok to not export each and every table and graph created during exploratory analysis. 
Final analysis scripts, on the other hand, should export final outputs, which are ready to be included to a paper or report.
No manual edits, including formatting, should be necessary after exporting final outputs -- 
those that require copying and pasting edited outputs, in particular, are absolutely not advisable. 
Manual edits are difficult to replicate, and you will inevitably need to make changes to the outputs. 
Automating them will save you time by the end of the process. 
However, don't spend too much time formatting tables and graphs until you are ready to publish.\sidenote{For a more detailed discussion on this, including different ways to export tables from Stata, see \url{https://github.com/bbdaniels/stata-tables}}
Polishing final outputs can be a time-consuming process, 
and you want to it as few times as possible.

We cannot stress this enough: don't ever set a workflow that requires copying and pasting results.
Copying results from excel to word is error-prone and inefficient.
Copying results from a software console is risk-prone, even more inefficient, and unnecessary.
There are numerous commands to export outputs from both R and Stata to a myriad of formats.\sidenote{Some examples are \href{ http://repec.sowi.unibe.ch/stata/estout/}{\texttt{estout}}, \href{https://www.princeton.edu/~otorres/Outreg2.pdf}{\texttt{outreg2}}, 
and \href{https://www.benjaminbdaniels.com/stata-code/outwrite/}{\texttt{outwrite}} in Stata, 
and \href{https://cran.r-project.org/web/packages/stargazer/vignettes/stargazer.pdf}{\texttt{stargazer}}
and \href{https://ggplot2.tidyverse.org/reference/ggsave.html}{\texttt{ggsave}} in R.}
Save outputs in accessible and, whenever possible, lightweight formats.
Accessible means that it's easy for other people to open them.
In Stata, that would mean always using \texttt{graph export} to save images as \texttt{.jpg}, \texttt{.png}, \texttt{.pdf}, etc., 
instead of \texttt{graph save}, which creates a \texttt{.gph} file that can only be opened through a Stata installation.
Some publications require ``lossless'' TIFF of EPS files, which are created by specifying the desired extension.
Whichever format you decide to use, remember to always specify the file extension explicitly.
For tables there are less options and more consideration to be made.
Exporting table to \texttt{.tex} should be preferred. 
Excel \texttt{.xlsx} and \texttt{.csv} are also commonly used, 
but require the extra step of copying the tables into the final output.
The amount of work needed in a copy-paste workflow increases rapidly with the number of tables and figures included in a research output, 
and do the chances of having the wrong version a result in your paper or report.


% Formatting
If you need to create a table with a very particular format, that is not automated by any command you know, consider writing the it manually 
(Stata's \texttt{filewrite}, for example, allows you to do that).
This will allow you to write a cleaner script that focuses on the econometrics, and not on complicated commands to create and append intermediate matrices.
To avoid cluttering your scripts with formatting and ensure that formatting is consistent across outputs,
define formatting options in an R object or a Stata global and call them when needed.
% Output content
Keep in mind that final outputs should be self-standing.
This means it should be easy to read and understand them with only the information they contain.
Make sure labels and notes cover all relevant information, such as sample, unit of observation, unit of measurement and variable definition.\sidenote{\url{https://dimewiki.worldbank.org/wiki/Checklist:\_Reviewing\_Graphs} \\ \url{https://dimewiki.worldbank.org/wiki/Checklist:\_Submit\_Table}}



%------------------------------------------------
