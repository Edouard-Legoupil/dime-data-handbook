%------------------------------------------------

\begin{fullwidth}

	Policy decisions are made every day using the results of development briefs and studies,
	and these have wide-reaching effects on the lives of millions.
	As the range of policy questions asked by researchers grows,
	so too does the (rightful) scrutiny under which methods and results are placed.
	It is therefore useful to think of research as a public service
  that requires researchers as a group to be accountable to the general public.
	This means acting to collectively protect the credibility of development research
	by following modern practices for research transparency and reproducibility.

  Across the social sciences, the open science movement has been fueled
  by discoveries of low-quality research practices,
	data and code that are inaccessible to the public,
  analytical errors in major research papers,
	and in some cases even outright fraud.
  While the development research community has not yet
  experienced any major scandals,
  it has become clear that there are necessary incremental improvements
	in the way that code and data are handled as part of research.
	Neither reproducibility nor transparency is an all-or-nothing objective:
	the most key is to report the transparency and privacy measures you have taken
  and always strive to do the best that you can with current technology.
	In this chapter, we outline a set of practices that help to ensure
	research consumers can be confident in the conclusions reached.

\end{fullwidth}

%------------------------------------------------

\section{Protecting confidence in development research}

The empirical revolution in development research\cite{angrist2017economic}
has therefore led to increased public scrutiny of the reliability of research.\cite{rogers_2017}\index{transparency}\index{credibility}\index{reproducibility}
Three major components make up this scrutiny: \textbf{reproducibility}\cite{duvendack2017meant}, \textbf{transparency},\cite{christensen2018transparency} and \textbf{credibility}.\cite{ioannidis2017power}
Development researchers should take these concerns seriously.
Many development research projects are purpose-built to address specific questions,
and often use unique data or small samples.
As a result, it is often the case that the data
researchers use for such studies has never been reviewed by anyone else,
so it is hard for others to verify that it was
collected, handled, and analyzed appropriately.

Reproducible and transparent methods are key to maintaining credibility
and avoiding serious errors.
This is particularly true for research that relies on original or novel data sources,
from innovative big data sources to surveys.
The field is slowly moving in the direction of requiring greater transparency.
Major publishers and funders, most notably the American Economic Association,
have taken steps to require that code and data
are accurately reported, cited, and preserved as outputs in themselves.\sidenote{
	\url{https://www.aeaweb.org/journals/policies/data-code}}


\subsection{Research reproducibility}

Can another researcher reuse the same code on the same data
and get the exact same results as in your published paper?\sidenote{
  \url{https://blogs.worldbank.org/impactevaluations/what-development-economists-talk-about-when-they-talk-about-reproducibility}}
This is a standard known as \textbf{computational reproducibility},
and it is an increasingly common requirement for publication.\sidenote{
\url{https://www.nap.edu/resource/25303/R&R.pdf}})
It is best practice to verify computational reproducibility before submitting a paper before publication.
This should be done by someone who is not on your research team, on a different computer,
using exactly the package of code and data files you plan to submit with your paper.
Code that is well-organized into a master script, and written to be easily run by others,
makes this task simpler.
The next chapter discusses organization of data work in detail.

For research to be reproducible,
all code files for data cleaning, construction and analysis
should be public, unless they contain identifying information.
Nobody should have to guess what exactly comprises a given index,
or what controls are included in your main regression,
or whether or not you clustered standard errors correctly.
That is, as a purely technical matter, nobody should have to ``just trust you'',
nor should they have to bother you to find out what happens
if any or all of these things were to be done slightly differently.\cite{simmons2011false,simonsohn2015specification,wicherts2016degrees}
Letting people play around with your data and code
is a great way to have new questions asked and answered
based on the valuable work you have already done.\sidenote{
	\url{https://blogs.worldbank.org/opendata/making-analytics-reusable}}

Making your research reproducible is also a public good.\sidenote{
	\url{https://dimewiki.worldbank.org/Reproducible_Research}}
It enables other researchers to re-use your code and processes
to do their own work more easily and effectively in the future.
This may mean applying your techniques to their data
or implementing a similar structure in a different context.
As a pure public good, this is nearly costless.
The useful tools and standards you create will have high value to others.
If you are personally or professionally motivated by citations,
producing these kinds of resources can lead to that as well.
Therefore, your code should be written neatly with clear instructions and published openly.
It should be easy to read and understand in terms of structure, style, and syntax.
Finally, the corresponding dataset should be openly accessible
unless for legal or ethical reasons it cannot be.\sidenote{
	\url{https://dimewiki.worldbank.org/Publishing_Data}}

\subsection{Research transparency}

Transparent research will expose not only the code,
but all research processes involved in developing the analytical approach.\sidenote{
	\url{https://www.princeton.edu/~mjs3/open_and_reproducible_opr_2017.pdf}}
This means that readers are able to judge for themselves if the research was done well
and the decision-making process was sound.
If the research is well-structured, and all of the relevant documentation\sidenote{
	\url{https://dimewiki.worldbank.org/Data_Documentation}}
is shared, this makes it easy for the reader to understand the analysis later.
Expecting process transparency is also an incentive for researchers to make better decisions,
be skeptical and thorough about their assumptions,
and, as we hope to convince you, make the process easier for themselves,
because it requires methodical organization that is labor-saving over the complete course of a project.

Tools like \textbf{pre-registration}\sidenote{
	\url{https://dimewiki.worldbank.org/Pre-Registration}},
\textbf{pre-analysis plans}\sidenote{
	\url{https://dimewiki.worldbank.org/Pre-Analysis_Plan}},
and \textbf{registered reports}\sidenote{
	\url{https://blogs.worldbank.org/impactevaluations/registered-reports-piloting-pre-results-review-process-journal-development-economics}}
can help with this process where they are available.\index{pre-registration}\index{pre-analysis plans}\index{Registered Reports}
By pre-specifying a large portion of the research design,\sidenote{
	\url{https://www.bitss.org/2019/04/18/better-pre-analysis-plans-through-design-declaration-and-diagnosis}}
a great deal of analytical planning has already been completed,
and at least some research questions are pre-committed for publication regardless of the outcome.
This is meant to combat the ``file-drawer problem'',\cite{simonsohn2014p}
and ensure that researchers are transparent in the additional sense that
all the results obtained from registered studies are actually published.
In no way should this be viewed as binding the hands of the researcher.\cite{olken2015promises}
Anything outside the original plan is just as interesting and valuable
as it would have been if the the plan was never published;
but having pre-committed to any particular inquiry makes its results
immune to a wide range of criticisms of specification searching or multiple testing.

Documenting a project in detail greatly increases transparency.
Many disciplines have a tradition of keeping a ``lab notebook'',
and adapting and expanding this process to create a
lab-style workflow in the development field is a
critical step towards more transparent practices.
This means explicitly noting decisions as they are made,
and explaining the process behind the decision-making.
Documentation on data processing and additional hypotheses tested
will be expected in the supplemental materials to any publication.
Careful documentation will also save the research team a lot of time during a project,
as it prevents you from having the same discussion twice (or more!),
since you have a record of why something was done in a particular way.
There are a number of available tools
that will contribute to producing documentation,
\index{project documentation}
but project documentation should always be an active and ongoing process,
not a one-time requirement or retrospective task.
New decisions are always being made as the plan begins contact with reality,
and there is nothing wrong with sensible adaptation so long as it is recorded and disclosed.

There are various software solutions for building documentation over time.
The \textbf{Open Science Framework}\sidenote{\url{https://osf.io}} provides one such solution,\index{Open Science Framework}
with integrated file storage, version histories, and collaborative wiki pages.
\textbf{GitHub}\sidenote{\url{https://github.com}} provides a transparent documentation system\sidenote{
	\url{https://dimewiki.worldbank.org/Getting_started_with_GitHub}},\index{task management}\index{GitHub}
in addition to version histories and wiki pages.
Such services offer multiple different ways
to record the decision process leading to changes and additions,
track and register discussions, and manage tasks.
These are flexible tools that can be adapted to different team and project dynamics.
Services that log your research process can show things like modifications made in response to referee comments,
by having tagged version histories at each major revision.
They also allow you to use issue trackers
to document the research paths and questions you may have tried to answer
as a resource to others who have similar questions.
Each project has specific requirements for data, code, and documentation management,
and the exact transparency tools to use will depend on the team's needs,
but they should be agreed upon prior to project launch.
This way, you can start building a project's documentation as soon as you start making decisions.
Email, however, is \textit{not} a note-taking service, because communications are rarely well-ordered,
can be easily deleted, and are not available for future team members.

\subsection{Research credibility}

The credibility of research is traditionally a function of design choices.\cite{angrist2010credibility,ioannidis2005most}
Is the research design sufficiently powered through its sampling and randomization?
Were the key research outcomes pre-specified or chosen ex-post?
How sensitive are the results to changes in specifications or definitions?
Pre-analysis plans can be used to assuage these concerns for experimental evaluations
\index{pre-analysis plan}
by fully specifying some set of analysis intended to be conducted.
Regardless of whether or not a formal pre-analysis plan is utilized,
all experimental and observational studies should be pre-registered
simply to create a record of the fact that the study was undertaken.\sidenote{\url{https://datacolada.org/12}}
This is increasingly required by publishers and can be done very quickly
using the \textbf{AEA} database,\sidenote{\url{https://www.socialscienceregistry.org}}
the \textbf{3ie} database,\sidenote{\url{https://ridie.3ieimpact.org}}
the \textbf{eGAP} database,\sidenote{\url{https://egap.org/content/registration}}
or the \textbf{OSF} registry,\sidenote{\url{https://osf.io/registries}} as appropriate.
\index{pre-registration}

Common research standards from journals and funders feature both ex ante
(or ``regulation'') and ex post (or ``verification'') policies.\cite{stodden2013toward}
Ex ante policies require that authors bear the burden
of ensuring they provide some set of materials before publication
and their quality meet some minimum standard.
Ex post policies require that authors make certain materials available to the public,
but their quality is not a direct condition for publication.
Still others have suggested ``guidance'' policies that would offer checklists
for which practices to adopt, such as reporting on whether and how
various practices were implemented.\cite{nosek2015promoting}

With the ongoing rise of empirical research and increased public scrutiny of scientific evidence,
simply making analysis code and data available
is no longer sufficient on its own to guarantee that findings will hold their credibility.
Even if your methods are highly precise,
your evidence is only as good as your data --
and there are plenty of mistakes that can be made between
establishing a design and generating final results that would compromise its conclusions.
That is why transparency is key for research credibility.
It allows other researchers, and research consumers,
to verify the steps to a conclusion by themselves,
and decide whether their standards for accepting a finding as evidence are met.
Therefore we encourage you to work, gradually, towards improving
the documentation and release of your research materials,
and finding the tools and workflows that best match your project and team.
Every investment you make in documentation and transparency up front
protects your project down the line, particularly as these standards continue to tighten.
Since projects tend to span over many years,
the records you will need to have available for publication are
only bound to increase by the time you do so.
