\begin{fullwidth}
Welcome to \textit{Development Research in Practice: The DIME Analytics Data Handbook}.
This book is intended to teach all users of development data
how to handle data effectively, efficiently, and ethically.
It takes lessons, tools, and processes that emerged from the DIME portfolio
and compiles them into a single narrative about doing data work
that we hope will provide a foundation for these professional skills.

DIME generates high-quality and operationally relevant data and research
to transform development policy, help reduce extreme poverty, and secure shared prosperity.
It develops customized data and evidence ecosystems to produce actionable information
and recommend specific policy pathways to maximize impact.
DIME conducts research in 60 countries with 200 agencies, leveraging a
US\$180 million research budget to shape the design and implementation of
US\$18 billion in development finance.
DIME also provides advisory services to 30 multilateral and bilateral development agencies.
The DIME team includes four primary topic pillars,
has included dozens of research economists,
and, over the years, has employed hundreds of full-time research assistants, field coordinators, and staff.
The team has conducted over 325 impact evaluations.
This book exists to take advantage of that concentration and scale of research,
to synthesize many resources for data collection and research,
and to make DIME tools available to the larger community of development researchers.

As part of its broader mission, DIME invests in public goods (such as this book)
to improve the quality and reproducibility of development research around the world.
One key early innovation at DIME was the creation of DIME Analytics ,
the team responsible for writing and maintaining this book.
This team has five full-time staff and several part-time members,
and are collectively responsible for ensuring the quality of research practices across DIME.
This is done through an intensive, collaborative innovation cycle:
DIME Analytics onboards and supports research assistants and field coordinators,
provides standard tools and workflows to all teams,
delivers intensive support when new tasks or challenges arise,
and then develops and integrates lessons from those engagements to bring to the full team.
In that way the Analytics team is DIME's method of ``institutionalizing''
a set of tools and practices, developed and refined over time,
that give the unit a common base of knowledge and practice.
This book is targeted to everyone who interacts with development data:
graduate students, research assistants, policymakers, and empirical researchers.
It aims to be a highly practical resource so the reader can
immediately begin to collaborate effectively on large, long-term research projects
that already use the methods and tools outlined in this book.

\end{fullwidth}

%------------------------------------------------

\section{Doing credible research at scale}

An empirical revolution has changed the face of development economics research rapidly over the last decade.
Today, working with raw data --
whether collected through surveys,
shared by partner organizations
or acquired from ``big'' data sources
like sensors, satellites, or call data records --
is a key skill for researchers and their staff.
At the same time, the scope and scale of empirical research projects is expanding:
more people are working on the same data over longer timeframes.
As the ambition of development researchers grows, so too has the complexity of the data
on which they rely to make policy-relevant research conclusions.
Yet there are few guides to the conventions, standards, and best practices
that are fast becoming a necessity for empirical research.

This book aims to fill that gap.
It covers data workflows at all stages of the research process,
from design to data acquisition and analysis.
Its content is not sector-specific;
it will not teach you econometrics,
or how to design an impact evaluation.
There are many excellent existing resources on those topics.
Instead, this book will teach you how to think about all aspects of your research from a data perspective,
how to structure research projects to maximize data quality,
and how to institute transparent and reproducible workflows.
The central premise of this book is that data work is a ``social process'',
in which many people on a team need to have the same ideas
about what is to be done, and when and where and by whom,
so that they can collaborate effectively on large, long-term research projects.

\textit{Development Research in Practice} compiles the ideas, best practices and software tools
that the DIME Analytics team
has developed while supporting DIME's global impact evaluation portfolio.
Each chapter in this book focuses on one task, providing a primarily narrative account of:
what you will be doing; where in the workflow this task falls;
when it should be done; and how to implement it according to best practices.
We will use broad terminology throughout this book to refer to some typical research team members:
\textbf{principal investigators (PIs)} who are responsible for
the overall design and stewardship of the study;
\textbf{field coordinators (FCs)} who are responsible for
the implementation of the study on the ground;
and \textbf{research assistants (RAs)} who are responsible for
handling data processing and analytical tasks.

We will not always give a lot of highly specific implementation details in this text,
but will often point you to where they can be found on the \textbf{DIME Wiki}.\sidenote{Like this:
\url{https://dimewiki.worldbank.org/Primary_Data_Collection}}
The DIME Wiki is one of DIME Analytics' flagship products,
a free online collection of our resources and best practices.\sidenote{
\url{https://dimewiki.worldbank.org}}
This book complements the DIME Wiki by providing a structured narrative
of the data workflow for a typical research project.
The Wiki, by contrast, provides unstructured but detailed information
on how to complete each task, and links to further practical resources.

\section{Adopting reproducible analytical practices through code}

Modern quantitative research relies heavily on statistical software tools.
Many development researchers come from economics and statistics backgrounds
and often understand code to be a means to an end rather than an output itself.
We believe that this must change somewhat:
in particular, we think that development practitioners
must think about their code and programming workflows
just as methodologically as they think about their research workflows,
and think of code and data as research outputs, just as manuscripts and briefs are.
To this end, DIME now maintains strict portfolio-wide standards
about how analytical code should be maintained and made accessible
before, during, and after release or publication.
This is because we see the code as the ``recipe'' for the analysis:
it tells others exactly what was done,
how they can do it again in the future,
and provides a roadmap and knowledge base for further original work.
DIME Analytics is responsible for maintaining and enforcing these standards,
and they are intended to cooperatively improve the quality of the work done by
every member of the DIME team, never to police or punish them.

Similarly, we assume throughout all of this book
that you are going to do nearly all of your data work though code.
It is often \textit{possible} to perform nearly all the relevant tasks
through the interactive user interface in a statistical software,
or even through non-specialized software such as Excel.
However, we strongly advise against it.
The reason for this is the set of
transparency, reproducibility, and credibility principles
that we discuss in Chapter 1.
Performing every task through written code
creates a record of every task you performed.
It also prevents direct interaction
with the data files that could lead to non-reproducible processes.
Finally, we have invested a lot of time in developing code as a learning tool:
the examples we have written and the commands we provide
are designed to provide a framework for common practice
across the entire DIME team, so that everyone is able to
read, review, and provide feedback on the work of others
starting from the same basic ideas about how various tasks are done.

Most specific code tools have a learning and adaptation process,
meaning you will become most comfortable with each tool
only by using it in real-world work.
To support your process of learning reproducible tools and workflows,
we reference free and open-source tools wherever possible,
and point to more detailed instructions when relevant.
Stata, as a proprietary software, is the notable exception here
due to its persistent popularity in development economics and econometrics.\sidenote{
  \url{https://aeadataeditor.github.io/presentation-20191211/\#9}}
This book also includes, as an appendix,
the \textbf{DIME Analytics Stata Style Guide}
that we use in our work, which provides
standards for coding in Stata so that code styles
can be harmonized across teams for easier understanding and reuse of code.
Stata has relatively few resources of this type available,
and the one that we have created and shared here
we hope will be an asset to all its users.


\section{Ensuring high-quality data through good coding}

Throughout this book, we refer to the importance of good coding practices.
These are the foundation of reproducible and credible data work,
and a core part of the new data science of development research.
Code today is no longer a means to an end (such as a research paper),
rather it is part of the output itself: a means for communicating how something was done,
in a world where the credibility and transparency of data collection, cleaning, and analysis is increasingly important.
As this is fundamental to the remainder of the book's content,
we provide here a brief introduction using code for \textbf{process standardization} in data collection.

Process standardization means that there is
little ambiguity about how something ought to be done,
and therefore the tools to do it can be set in advance.
Standard processes for code help other people to read your code.\sidenote{
\url{https://dimewiki.worldbank.org/Stata_Coding_Practices}}
Code should be well-documented, contain extensive comments, and be readable in the sense that others can:
(1) quickly understand what a portion of code is supposed to be doing;
(2) evaluate whether or not it does that thing correctly; and
(3) modify it efficiently either to test alternative hypotheses
or to adapt into their own work.\sidenote{\url{https://kbroman.org/Tools4RR/assets/lectures/07_clearcode.pdf}}

In the context of data collection, there are many places where good code and code annotation
are important to accurately convey and implement the intended research design in reality.
For example, statistical software is now an essential part of
design components such as sampling, randomization, and power analysis.
Being able to accurately and reproducibily implement these tasks
is essential to the success and credibility of any modern randomized experiment.
Simultaneously, researchers need to maintain records of the structure of their data,
which involves managing and collating different types of information,
often at different levels of analysis and different stages in time.
Keeping a clear, human-readable record of these code and data structures is critical.
Finally, as data is collected, code tools must be used to validate its quality,
organize and manage the linkages between datasets,
resolve and correct errors that occur in the process of data acquisition,
and, ultimately, create and analyze the measures that
are the motivation for the research study.

A breakdown in any part of this data pipeline
means that the data that are acquired become unreliable.
If that happens, the results cannot be faithfully interpreted
as being an accurate picture of the intended research design.
Because we almost never have ``laboratory'' settings in this type of research,
such a failure has a very high cost:
we will have wasted the investments that were made into knowledge generation,
with little ability to reproduce or recreate the situation
where we intended to operate the research project.
Hence accurate and reproducible data management is a core component
of the credibility of development research.

\section{Outline of this book}

This book covers each stage of an empirical research project, from design to publication.
We start with ethical principles to guide empirical research,
focusing on research reproducibility, transparency, and credibility.
In Chapter 1, we outline a set of practices that help to ensure that
research consumers can be confident in the conclusions reached,
and research work can be assumed and verified to be reliable.
Chapter 2 will teach you to structure your data work for collaborative research,
while ensuring the privacy and security of research participants.
It discusses the importance of planning the tools that will be used;
lays the groundwork to structure the research project at its outset --
long before any data is acquired --
and provides suggestions for collaborative workflows and tools.
In Chapter 3, we turn to establishing a measurement framework,
focusing specifically on how to translate research design to a data work plan
and how to implement both simple and complex randomized designs in a reproducible manner.

Chapter 4 covers data acquisition. We start with
the legal and institutional frameworks for data ownership and licensing,
dive in depth on collecting high-quality survey data,
and finally discuss secure data handling during transfer, sharing, and storage.
Chapter 5 teaches workflows for data processing.
It details how to construct ``tidy'' data at the appropriate units of analysis,
how to ensure uniquely identified datasets, and
how to routinely incorporate data quality checks into the workflow.
It also provides guidance on de-identification and cleaning of personally-identified data,
focusing on how to understand and structure data
so that it is ready for indicator construction and analytical work.
Chapter 6 discusses data analysis.
It begins with data construction, or the creation of new variables
from the raw data acquired or collected in the field.
It also introduces core principles for writing analytical code
and creating, exporting, and storing research outputs such as figures and tables reproducibily with dynamic documents.
In Chapter 7, we turn to publication.
This chapter discusses
how to effectively collaborate on technical writing,
how and why to publish data,
and guidelines for preparing functional and informative reproducibility packages.

While adopting the workflows and mindsets described in this book requires an up-front cost,
it will save you (and your collaborators) a lot of time and hassle very quickly.
In part this is because you will learn how to implement essential practices directly;
in part because you will find tools for the more advanced practices;
and most importantly because you will acquire the mindset of doing research with a high-quality data focus.
We hope you will find this book helpful for accomplishing all of the above,
and that mastery of data helps you make an impact.
We hope that by the end of the book,
you will have learned how to handle data more efficiently, effectively and ethically
at all stages of the research process.

\mainmatter
