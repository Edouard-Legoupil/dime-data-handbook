%------------------------------------------------

\begin{fullwidth}
Research design is the process of designing field work
that will answer a specific research question.
You don't need to be an expert in this,
and there are lots of good resources out there
that focus on designing interventions and evaluations.
This section will present a very brief overview
of the most common methods that are used,
so that you can have an understanding of
how to construct appropriate counterfactuals
when you write analysis code later.

You can break most research questions into two main types.
There are cross-sectional or observational analyses,
which seek only to describe something for the first time,
such as the structure or variation of a population.
Then there are causal analyses,
which typically rely on establishing exogenous variation
to draw a conclusion about the impact of some event
on specific outcomes of interest.
We'll focus on the latter, since these are primarily
a standardized set of approaches, with publications
and code tools broadly available to support your work.
\end{fullwidth}

%------------------------------------------------

\section{Counterfactuals and treatment effects}

In causal analysis, a typical goal is to obtain an estimate
of a \textbf{treatment effect}, or the change in outcome
caused by a change in exposure to some intervention or circumstance.
However, in the potential outcomes framework,
we can never observe this directly:
we never see the same person in both their treated and untreated state.
Instead, we make inferences from samples:
we try to devise a comparison group that evidence suggests
would be identical to the treated group had they not been treated.

This control group serves as a counterfactual to the treatment group,
and we compare the distributions of outcomes within each
to make a computation of ``how different'' the groups are from each other.
\marginnote{\textit{Causal Inference: The Mixtape}
provides a detailed introduction to and history of
each of these methods, so we will only introduce you to
them very abstractly in this chapter. \url{http://scunning.com/cunningham_mixtape.pdf}}
Each of the methods described in this chapter
relies on some variant of this basic strategy.
In counterfactual causal analysis,
the role of the econometric model and estimating equations
is not to create a predictive or realistic model
of how the outcome of interest is actually generated --
typically we do not care about measures of fit or predictive accuracy
like R-squareds or root mean square errors.
What matters is whether the estimating equation
correctly describes the experimental design being used,
so that the correct estimate of the difference
between the treatment and control groups is captured.

This means capturing design choices
such as stratification and clustering and
ensuring that time trends are handled sensibly.
It means thinking carefully about how to transform and scale your data,
using fixed effects to extract ``within-group'' comparisons as needed,
and choosing estimators appropriate to your design.
As the saying goes, all models are wrong, but some are useful.
The models you will construct and estimate are intended to do two things:
to express the intention of your research design,
and to help you group the potentially endless concepts of field reality
into intellectually tractable categories.
In other words, these models tell the story of your research.

%------------------------------------------------

\section{Experimental reseach designs}

Experimental research designs explicitly allow the research team
to change the condition of the populations being studies.
This can be in the form of NGO programs, government regulations,
information campaigns, and many more types of interventions.
The ``classic'' method is the randomized control trial:
treatment groups are drawn from the same underlying population
so that the strong condition of statistical equality
in the absense of the experiment can be assumed.
Three methods are discussed here:
cross-sectional randomization (``endline-only'' studies),
differences-in-differences (``panel data'' studies),
and regression discontinuity (``cutoff'' studies).

\subsection{Cross-sectional RCTs}

Cross-sectional RCTs are the simplest possible study design:
a program is implemented, surveys are conducted, and data is analyzed.
The randomization process, like all RCTs,
draws the treatment groups from the same underlying population,
so that it is assumed the groups would be identical
before the intervention, and would have been identical at measurement
except for the effects of the intervention.
Cross-sectional data is simple because there is no need
for research teams to engage in tracking over time,
or analyzing attrition and follow-up.
Cross-sectional designs can have a time dimension;
they are then called ``repeated cross-sections'',
but do not imply a panel structure for individual observations.

Typically, the cross-sectional model is developed
only with controls for the research design.
Balance checks can be utilized, but an effective experiment
can use stratification aggressively to ensure balance
before data is collected.\cite{athey2017econometrics}
Since the experiment is fully randomized,
controls for imbalance that are not part of the design
only increase the precision of estimates,
but mechanically induce some bias away from the true effect.

\subsection{Differences-in-differences}

Differences-in-differences deals with imbalance differently:
it uses a panel data structure to additionally use each
unit in the pre-treatment phase as a control for itself post-treatment.
Therefore, rather than relying on balance design for credibility,
DD-type designs therefore only need to test whether \textit{changes}
in outcomes over time were different in the treatment group than the control group.
Therefore, the identifying assumption is \textbf{parallel trends},
the idea that the change in these groups would have been identical
in the absense of the treatment.

Diff-in-diff experiments therefore require substantially more effort
in the field work portion, so that the panel is well-constructed.
Since baseline and endline data collection may be far apart,
it is important to create careful records during the first round
so that follow-ups can be conducted with the same subjects,
and attrition across rounds can be properly taken into account.
Depending on the distribution of results,
estimates may become completely uninformative
with relatively little loss to follow-up.\sidenote{\url{http://blogs.worldbank.org/impactevaluations/dealing-attrition-field-experiments}}

The diff-in-diff model is a ``four-way'' comparison:
the experimental design wants to compare treatment to control,
after taking out the pre-levels for both.
Therefore the model includes a time period indicator,
a treatment group indicator (the pre-treatment control is the base level),
and it is the \textit{interaction} of treatment and time indicators
that we interpret as the differential effect of the treatment assignment.

\subsection{Regression discontinuity}

Regression discontinuity designs differ from RCTs
in that the treatment group is not randomly assigned.
(In fact, many RDs are not experimental at all.)
Instead, there is a \textbf{running variable}
which gives eligible people access to some program,
and a strict cutoff for who is included.
This is ususally justified by budget limitations.
The running variable should not be the outcome of interest,
and while it can be time, that may require additional modeling assumptions.
Those who qualify are given the intervention and those who don't are not,
and this process substitutes for explicit randomization.

For example, imagine that there is a strict income cutoff created
for a program that subsidizes some educational resources.
The intuition is that the people who are ``barely eligible''
should not in reality be very different from those who are ``barely ineligible'',
and that resulting differences between them at measurement
are therefore due to the intervention or program.
For the modeling component, the \textbf{bandwidth},
or the size of the window around the cutoff to use,
has to be decided and tested against various options for robustness.
The rest of the model depends largely on the design and execution of the experiment.

%------------------------------------------------

\section{Quasi-experimental reseach designs}
