%------------------------------------------------

\begin{fullwidth}
Data analysis is hard. Making sense of a dataset in such a way
that makes a substantial contribution to scientific knowledge
requires a mix of subject expertise, programming skills,
and statistical and econometric knowledge.
The process of data analysis is therefore typically
a back-and-forth discussion between the various people
who have differing experiences, perspectives, and research interests.
The research assistant usually ends up being the fulcrum
for this discussion, and has to transfer and translate
results among people with a wide range of technical capabilities
while making sure that code and outputs do not become
tangled and lost over time (typically months or years).

Organization is the key to this task.
The structure of files needs to be well-organized,
so that any material can be found when it is needed.
Data structures need to be organized,
so that the various steps in creating datasets
can always be traced and revised without massive effort.
The structure of version histories and backups need to be organized,
so that different workstreams can exist simultaneously
and experimental analyses can be tried without a complex workflow.
Finally, the outputs need to be organized,
so that it is clear what results go with what analyses,
and that each individual output is a readable element in its own right.
This chapter will teach you how to stay organized,
so that you and the team can focus on getting the work right
rather than trying to understand what you did in the past.
\end{fullwidth}

%------------------------------------------------
