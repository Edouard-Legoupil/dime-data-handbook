%------------------------------------------------

\begin{fullwidth}
Preparation for data work begins long before you collect any data.
In order to be prepared for fieldwork, you need to know what you are getting into.
This means knowing which data sets you need,
how those data sets will stay organized and linked,
and what identifying information you will collect
for the different types and levels of data you'll observe.
Identifying these details creates a \textbf{data map} for your project,
giving you and your team a sense of how information resources should be organized.
It's okay to update this map once the project is underway --
the point is that everyone knows what the plan is.

Then, you must identify and prepare your tools and workflow.
All of the tools we provide you here are designed to prepare you for collaboration and replication,
so that you can confidently manage tools and tasks on your computer.
We will try to provide free, open-source, and platform-agnostic tools wherever possible,
and provide more detailed instructions for those with which we are familiar.
However, most have a learning and adaptation process,
meaning you will become most comfortable with each tool
only by using it in real-world work.
Get to know them well early on,
so that you do not spend a lot of time later figuring out basic functions.
\end{fullwidth}


%------------------------------------------------

\section{Preparing your digital workspace}

Being comfortable using your computer and having the tools you need in reach is key.
This section provides a brief introduction to key concepts and toolkits
that can help you take on the work you will be primarily responsible for.
Some of these skills may seem elementary,
but thinking about simple things from a workflow perspective
can help you make marginal improvements every day you work.

\subsection{Setting up your computer}

First things first: turn on your computer.
\marginnote{One reasonable setup is having your primary disk,
a local hard drive managed with a tool like Time Machine,
and a remote copy managed by a tool like Backblaze.
Dropbox files count only as local copies and never backups,
because others can alter it.}
Make sure you have fully updated the operating system,
that it is in good working order,
and that you have a \textbf{password-protected} login.
Then, make sure your computer is backed up.
Follow the \textbf{3-2-1 rule}:
(3) copies of everything;
(2) different physical media;
(1) offsite storage.\sidenote{\url{https://www.backblaze.com/blog/the-3-2-1-backup-strategy/}}

There are a few things that can make your life much easier,
although some have a little expense associated with them.\marginnote{This
section is the only time we'll suggest you spend money,
and it is totally okay if you or your organization cannot.}
Mainly, make sure you have a \textit{good} computer.
Get at least 16GB of RAM and a 500GB or 1TB hard drive.
(Processor speeds matter less these days.)
Get a monitor with high-definition resolution.
\marginnote{Free alternatives to these tools include LibreOffice, Bitwarden, and Duplicati,
although Dropbox is harder to replace effectively.}
Your life will be easier with paid copies of software like Microsoft Office 365
and critical services like \textbf{Dropbox},\sidenote{\url{https://www.dropbox.com}} \textbf{Backblaze},\sidenote{\url{https://www.backblaze.com}} and \textbf{LastPass}.\sidenote{\url{https://www.lastpass.com}}
Get a decent email client (like \textbf{Spark}\sidenote{\url{https://sparkmailapp.com}} or \textbf{Outlook}),
\index{software}
a calendar that you like, the communication and note-taking tools you need,
a good music streaming service, and solid headphones with a microphone.
None of these are essential to the work,
but they will make you a lot more comfortable doing it,
and being comfortable at your computer helps you stay happy and healthy.

When using a computer for research,
you should keep in mind a structure of work
known as \textbf{scientific computing}.\cite{wilson2014best,wilson2017good}
\index{scientific computing}
Scientific computing is a set of practices developed to help you
ensure that the computer is being used to improve your efficiency,
so that you can focus on the real-world problems instead of technical ones.\sidenote{
\url{https://www.dropbox.com/s/wqefknwfb91kop8/Coding_For_Econs_20190221.pdf?raw=1}}
This means getting to know your computer a little better than most people do,
and thinking critically about tasks like file structures,
code and \textbf{process reusability},\sidenote{
\url{http://blogs.worldbank.org/opendata/making-analytics-reusable}}
and software choice. Most importantly,
it means detecting early warning signs of \textbf{process bloat}.
As a general rule, if the work required to maintain a process
grows as fast (or faster) than the number of objects controlled by that process,
you need to stop work immediately and rethink processes.
You should work to design processes that are
close to infinitely scalable by the number of objects being handled --
whether they be field samples, data files, surveys, or other real or digital objects.

The first thing you need to have for scientific computing
is a sense of where you are on your filesystem.
\marginnote{You should \textit{always} use forward slashes (\texttt{/}) in filepaths,
just like an internet address, and no matter how your computer writes them,
because the other type will cause your work to break many systems.}
Find your \textbf{home folder}. On MacOS, this will be a folder with your username.
On Windows, this will be something like ``This PC''. (It is never your desktop.)
Nearly everything we talk about will assume you are starting from here.
Ensure you know how to get the \textbf{absolute file path} for any given file.
On MacOS this will be something like \path{/users/username/github/project/...},
and on Windows, \path{C:/users/username/github/project/...}.
We will write filepaths such as \path{/Dropbox/project-title/DataWork/EncryptedData/}
using forward slashes, and mostly use only A-Z, dash, and underscore.
You can use spaces in names of non-technical files, but not technical ones.\sidenote{
\url{http://www2.stat.duke.edu/~rcs46/lectures_2015/01-markdown-git/slides/naming-slides/naming-slides.pdf}}
Making the structure of your files part of your workflow is really important,
as is naming them correctly so you know what is where.

\subsection{Preparing for collaboration and replication}

Next, you will need a personal working environment and a team working environment.
If you are working in R, \textbf{RStudio} is great.\sidenote{\url{https://www.rstudio.com}}
For advance users of Stata, we recommend the text and code editor \textbf{Atom},\sidenote{\url{https://atom.io}}
which can open an entire targeted directory by \textit{adding a project folder} which
loads the whole tree view in the sidebar, and gives you access to directory management actions.
You can start to manage your projects as a whole --
Atom is capable of sending code to Stata,\sidenote{\url{https://atom.io/packages/stata-exec}}
writing and building \LaTeX,\sidenote{\url{https://atom.io/packages/latex}}
and connecting directly with others to team code.\sidenote{\url{https://atom.io/packages/teletype}}
It is highly customizable, and since it is your personal environment,
there are lots of stylistic and functional options in extension packages
that you can use to make your work easier and more enjoyable.

You will also need collaboration software.\sidenote{\url{https://dimewiki.worldbank.org/wiki/Collaboration_Tools}}
The two most common softwares in use are Dropbox and GitHub.\sidenote{
\url{https://michaelstepner.com/blog/git-vs-dropbox/}}
Dropbox and GitHub fall at the same level of the organizational hierarcy:
they should both have a top-level folder inside your home folder,
and neither one should ever be placed inside the other.
Many organizations have an institutional subscription to one or both services;
although some organizations have banned Dropbox outright due to its security issues.
Dropbox (or equivalent such as OneDrive) will be preferred by most non-technical staff,
including project PIs and field staff.
It is preferable for sharing documents that are non-technical (like \texttt{.docx}),
and data that is non-sensitive (no personal information).
GitHub will be preferred for code,
and therefore most of what you are doing can be managed with Git.
Code is written in its native language,
and increasingly, outputs like reports\sidenote{\url{https://www.latex-project.org}}
and presentations\sidenote{\url{https://www.overleaf.com/learn/latex/Beamer}}
are being written in languages like \LaTeX.

For non-technical materials,
it is common to have the field coordinator organize
the folder structure using Dropbox,
since they will need to handle a great deal of
project-specific information on a regular basis.
\marginnote{Increasingly, we recommend the entire data work folder
to be created and stored separately in GitHub.
Nearly all code and outputs (except datasets) are better managed this way.}
Then, there should be a top-level folder which the data handler controls.
This division of labor makes sure everyone knows
which parts of file organization they are responsible for.
Agree on a specific folder structure: we provide the
\texttt{iefolder}\sidenote{\url{https://dimewiki.worldbank.org/wiki/iefolder}}
structure as a baseline guide for teams.
\texttt{iefolder} will set up and manage the entire folder structure
you will need for the entire data portion of a project, from start to finish.
Our training materials for how to use the \texttt{iefolder} structure
are the first introduction DIME staff get to team work.
For code files, agree on a \textbf{version control}\cite{blischak2016quick} workflow such as
Git Flow,\sidenote{\url{https://nvie.com/posts/a-successful-git-branching-model/}}
and a naming structure for files and folders.

To help you stay organized, we have created a Stata command
called \texttt{iefolder}, as a part of our \texttt{ietoolkit} suite.\sidenote{
\url{https://dimewiki.worldbank.org/wiki/iefolder}}
This command sets up a standardized folder structure for you that we call the \textbf{DataWork} folder.\sidenote{
\url{https://dimewiki.worldbank.org/wiki/DataWork_Folder}}
Since the DataWork folder, \texttt{iefolder} also creates master do-files
that are linked to all main folders in the folder structure.\sidenote{
\url{https://dimewiki.worldbank.org/wiki/Master_Do-files}}
These master do-files are updated whenever more rounds, units of observations,
and subfolders are added to the project folder using this command.
It is important to use \texttt{iefolder} at the beginning of a research project
in order to employ the DataWork folder throughout the entire project.

You should therefore feel comfortable having both a project folder and a code folder.
Their structures can be managed in parallel by using \texttt{iefolder} twice.
The project folder can be maintained in a synced location like Dropbox,
and the code folder can be maintained in a version-controlled location like GitHub.
While both are used for sharing and collaborating,
there is a sharp difference between the functionality of sync and version control.
Namely, sync forces everyone to have the same version of every file at all times
and does not support simultaneous editing well; version control does the opposite.
Keeping code in a version-controlled folder will allow you
to maintain better control of its history and functionality,
and because of the specificity with which code depends on file structure,
you will be able to enforce better practices there than in the project folder.

Finally, set up task management. GitHub issues are a great tool for this,
and Dropbox Paper also provides a good interface with notifications.
Neither of these tools require much technical knowledge;
they merely require an agreement and workflow design
so that the people assigning the tasks
are sure to set them up in the system.
GitHub is useful because tasks can clearly be tied to file versions;
therefore it is useful for managing technical tasks.
Dropbox Paper is useful because tasks can be easily linked to other documents;
therefore it is useful for managing non-technical tasks.
Our team uses both.
