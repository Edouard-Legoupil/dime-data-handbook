%------------------------------------------------

\begin{fullwidth}
Preparation for data work begins long before you collect any data.
In order to be prepared for fieldwork, you will need to know what you are getting into.
This means having a plan for the data sets you will need,
a plan for how those data sets will stay organized and linked,
and what identifying information each type of observation you'll collect.
This is called a \textbf{data map} for your project,
and it gives you and your team a sense of how information resources should be organized.
It's okay to update this once the project is underway --
the point is that everyone knows what the plan is.

At the same time, you also need to be technically prepared.
This means having a strong technical sense of your tools and workflow,
so that you do not spend a lot of time trying to figure out basic functions.
All of the tools we provide you here are designed to prepare you for collaboration and replication,
so that you can confidently manage tools and tasks on your computer.
We will try to provide free, open-source, and platform-agnostic tools wherever possible,
and provide more detailed instructions for those we are familiar.
However, most have a learning and adaptation process,
meaning you will become most comfortable with each tool
only by using it in real-world work.
\end{fullwidth}


%------------------------------------------------

\section{Preparing your digital workspace}

Being comfortable using your computer and having the tools you need in reach is key.
This section provides a brief introduction to key concepts and toolkits
that can help you take on the work that you will be primarily responsible for.
Some of these items skills may seem elementary,
but thinking about simple things from a workflow perspective
can help you make marginal improvements every day you work.

\subsection{Setting up your computer}

First things first: turn on your computer.
\marginnote{One reasonable setup is having your primary disk,
a local hard drive managed with a tool like Time Machine,
and a remote copy managed by a tool like Backblaze.
Dropbox files count only as local copies and never backups,
because others can alter it.}
Make sure you have fully updated the operating system,
that it is in good working order,
and that you have a password-protected login.
Then, make sure your computer is backed up.
Follow the 3-2-1 rule:
(3) copies of everything;
(2) different physical media;
(1) offsite storage.

There are a few things that can make your life much easier,
although some have a little expense associated with them.
\marginnote{This is the only time we'll suggest you spend money,
and it is totally okay if you or your organization cannot.
There are free options for almost everything except functional hardware.}
Mainly, make sure you have a \textit{good} computer.
Get at least 16GB of RAM and a 500MB or 1TB hard drive.
(Processor speeds matter less these days.)
Get a monitor with high-definition resolution.
Purchase authorized copies of software like Microsoft Office 365,
and pay for the individual tier of critical services like Dropbox,\sidenote{\url{https://www.dropbox.com}} Backblaze,\sidenote{\url{https://www.backblaze.com}} and LastPass.\sidenote{\url{https://www.lastpass.com}}
Get a decent email client (like Spark\sidenote{\url{https://sparkmailapp.com}} or Outlook),
a calendar that you like, the communication and note-taking tools you need,
a good music streaming service, and solid headphones with a mic.
None of these are essential to the work,
but they will make you a lot more comfortable doing it,
and being comfortable at your computer helps you stay happy and healthy.

The first thing you need to have when working on your computer is a sense of where you are.
Find your \textbf{home folder}. On MacOS, this will be a folder with your username.
On Windows, this will be something like ``This PC''. (It is never your desktop.)
Nearly everything we talk about will assume you are starting from here.
Ensure you know how to get the \textbf{absolute file path} for any given file.
On MacOS this will be something like \path{/users/username/dropbox/project/...},
and on Windows, \path{C:/users/username/github/project/...}.
\marginnote{You should \textit{always} use forward slashes (\texttt{/}) in filepaths,
just like an internet address, and no matter how your computer writes them,
because the other type will cause your work to break many systems.}
We will write filepaths such as \path{/Dropbox/project-title/DataWork/EncryptedData/}
using forward slashes, and mostly use only A-Z, dash, and underscore.
You can use spaces in names of non-technical files, but not technical ones.
Making the structure of your files part of your workflow is really important,
as is naming them correctly so you know what is where.\sidenote{\url{http://www2.stat.duke.edu/~rcs46/lectures_2015/01-markdown-git/slides/naming-slides/naming-slides.pdf}}

\subsection{Preparing for collaboration and replication}

Next, you will need a personal working environment and a team working environment.
If you are working in R, RStudio is great.\sidenote{\url{https://www.rstudio.com}}
For Stata, we recommend the text editor Atom,\sidenote{\url{https://atom.io}}
which can open an entire targeted directory by writing \path{atom /path/to/directory/}
in the command line (Terminal or PowerShell), after copying it from the browser.
Opening the entire directory loads the whole tree view in the sidebar,
and gives you access to directory management actions.
You can start to manage your projects as a whole --
Atom is capable of sending code to Stata,\sidenote{\url{https://atom.io/packages/stata-exec}}
writing and building \LaTeX,\sidenote{\url{https://atom.io/packages/latex}}
and connecting directly with others to team code.\sidenote{\url{https://atom.io/packages/teletype}}
It is highly customizable, and since it is your personal environment,
there are lots of stylistic and functional options in extension packages
that you can use to make your work easier and more enjoyable.

You will also need collaboration software.
The two most common softwares in use are Dropbox and GitHub.\sidenote{\url{https://michaelstepner.com/blog/git-vs-dropbox/}}
Many organizations have an institutional subscription to one or both.
Dropbox will be preferred by most non-technical staff,
including project PIs and field staff.
It is preferable for sharing documents that are non-technical (like .docx),
and data that is non-sensitive (no personal information).
GitHub will be preferred for code,
and therefore most of what you are doing can be managed with Git.
Code is written in its native language,
and increasingly, outputs like reports\sidenote{\url{https://www.latex-project.org}}
and presentations\sidenote{\url{https://www.overleaf.com/learn/latex/Beamer}}
are being written in languages like \LaTeX.

%------------------------------------------------

\section{Collaborative workflow in practice}

Before you begin any work, you will need to settle in to working with others.
\marginnote{\texttt{ietoolkit} and \texttt{iefieldkit} provide Stata tools
for the workflows we recommend. They can save you a lot of time and hassle,
in data collection, data cleaning, and data analysis. Both are available on SSC.}
DIME Analytics has put a lot of time into making this process painless,
in a couple of key ways.
First, we provide guidelines to standardize work across teams;
second, we provide tools to make these standards efficiency-enhancing;
third, we provide training so people understand these tools and guidelines.
Your team doesn't have to work exactly as we describe here,
but know that a lot of work has gone into getting this all figured out.

This book and the DIME Wiki are a guide and a manual to our workflow.
This book will give you a narrative account of the things you'll need to do,
and some basic tools to do them.
The DIME Wiki will stay up to date with more extensive details and resources,
and will be very useful once you know what to search for.
However, these resources cannot predict everything that will happen on your project.
Most importantly, talk to others in your team.
Figure out what datasets and code inputs are going to be created,
and how they need to be handled.

The first task is file management.\sidenote{\url{https://dimewiki.worldbank.org/wiki/Data_Management}}
Figure out what is going to be stored where.
For non-technical materials,
it is common to have the field coordinator organize
the folder structure using Dropbox,
since they will need to handle a great deal of
project-specific information on a regular basis.
\marginnote{Increasingly, we recommend the entire data work folder
to be created and stored separately in GitHub.
Nearly all code and outputs (except datasets) are better managed this way.}
Then, there should be a top-level folder which the data handler controls.
This division of labor makes sure everyone knows
which parts of file organization they are responsible for.
Agree on a specific folder structure: we provide the
\texttt{iefolder}\sidenote{\url{https://dimewiki.worldbank.org/wiki/iefolder}}
structure as a baseline guide for teams.
\texttt{iefolder} will set up and manage the entire folder structure
you will need for the entire data portion of a project, from start to finish.
Our training materials for how to use the \texttt{iefolder} structure
are the first introduction DIME staff get to team work.

For data files, agree on a \textbf{version control}\cite{blischak2016quick} workflow such as
Git Flow,\sidenote{\url{https://nvie.com/posts/a-successful-git-branching-model/}}
and a naming structure for files and folders.
