%------------------------------------------------

\begin{fullwidth}
Development research does not just \textit{involve} real people -- it also \textit{affects} real people.
Policy decisions are made every day using the results of briefs and studies,
and these can have wide-reaching consequences on the lives of millions.
As the range and importance of the policy-relevant questions
asked by development researchers grow,
so too does the (rightful) scrutiny under which methods and results are placed.
This scrutiny involves two major components: data handling and analytical quality.
Performing at a high standard in both means that research participants
are appropriately protected,
and that consumers of research can have confidence in its conclusions.

Ethical standards ensure that research products are able to be judged on their performance in both.
Without transparent measures of credibility, reputation is a primary signal, and two failures may occur:
low-quality studies from reputable sources may be used as evidence when in fact they don't warrant it,
and high-quality studies from sources without an international reputation may be ignored.
Both these outcomes reduce the quality of evidence overall.
Even more importantly, they usually shift power in development research
away from the people and places directly involved in and affected by it.
Simple transparency standards mean that high-quality research is more common, and more impactful.
This section provides some basic guidelines and resources
for collecting, handling, and using field data ethically and responsibly to publish research findings.
\end{fullwidth}

%------------------------------------------------

\section{Protecting confidence in development research}

The empirical revolution in development research
\index{transparency}\index{credibility}\index{reproducibility}
has led to increased public scrutiny of the reliability of research.\cite{rogers_2017}
Three major components make up this scrutiny: \textbf{credibility},\cite{ioannidis2017power} \textbf{transparency},\cite{christensen2018transparency} and \textbf{replicability}.\cite{duvendack2017meant}
Development researchers take these concerns seriously.
Many development research projects are purpose-built to cover specific questions,
and may utilize unique data or small samples.
This approach opens the door to working with the development community
to answer both specific programmatic questions and general research inquiries.
However, the data researchers utilize have never been reviewed by anyone else,
so it is hard for others to verify that it was collected, handled, and analyzed appropriately.
Maintaining confidence in research via the components of credibility, transparency, and replicability
is the most important way that researchers can avoid serious error,
and therefore these are not by-products but core components of research output.

\subsection{Research credibility}

The credibility of research is primarily a function of design choices.\cite{angrist2010credibility,ioannidis2005most}
Is the research design sufficiently powered through its sampling and randomization?
Were the key research outcomes pre-specified or chosen ex-post?
How sensitive are the results to changes in specifications or definitions?
Tools such as \textbf{pre-analysis plans}\sidenote{\url{https://dimewiki.worldbank.org/wiki/Pre-Analysis_Plan}}
are important to assuage these concerns for experimental evaluations,
\index{pre-analysis plan}
but they may feel like ``golden handcuffs'' for other types of research.\cite{olken2015promises}
Regardless of whether or not a formal pre-analysis plan is utilized,
all experimental and observational studies should be \textbf{pre-registered}\sidenote{
\url{https://dimewiki.worldbank.org/wiki/Pre-Registration}}
using the \textbf{AEA} database,\sidenote{\url{https://www.socialscienceregistry.org/}}
the \textbf{3ie} database,\sidenote{\url{http://ridie.3ieimpact.org/}}
or the \textbf{OSF} registry\sidenote{\url{https://osf.io/registries}} as appropriate.\sidenote{\url{http://datacolada.org/12}}
\index{pre-registration}

Highly specific documentation is always helpful to establish credibility,\sidenote{\url{https://dimewiki.worldbank.org/wiki/Data_Documentation}}
and will be expected in the supplemental materials to any publication.
\marginnote{Email is not a note-taking service.
Indexed note managers that include sync and search include
Dropbox Paper, nvAlt, and Microsoft OneNote.
\\ \noindent \url{https://paper.dropbox.com} \\ \noindent \url{https://brettterpstra.com/projects/nvalt/}}
The registries mentioned here will contribute to producing documentation,
\index{project documentation}
but project documentation should always be an active and ongoing process,
not a one-time requirement or retrospective task.
\marginnote{New decisions are always being made as the plan begins contact with reality,
and there is nothing wrong with sensible adaptation so long as it is recorded and disclosed.}
There are various software solutions for building these documentations over time.
The \textbf{Open Science Framework}\sidenote{\url{https://osf.io/}} provides one such solution,
with integrated file storage, version histories, and collaborative wiki pages.
\textbf{GitHub}\sidenote{\url{https://github.com}} provides a transparent task management
platform\sidenote{\url{https://dimewiki.worldbank.org/wiki/Getting_started_with_GitHub}}
\index{task management}\index{GitHub}in addition to version histories and wiki pages, but is not well suited for data file storage.
The exact shape of this process should be agreed on by team members prior to project launch.

\subsection{Research transparency}

Transparent research will expose all of the processes involved in establishing credibility to the public.\sidenote{\url{http://www.princeton.edu/~mjs3/open_and_reproducible_opr_2017.pdf}}
This means that not only will readers be able to verify for themselves that the research was done well,
but they will be able to test assumptions and results that the researchers may not even have considered.
Well-structured research should make this as easy as possible for the reader to do.
\marginnote{We also hope to convince you that transparent research is easier for the researcher,
because it requires you to be organized in a labor-saving way.}
Etiquette is important here: nobody should go around trying to tear down the findings of others,
and neither should researchers guard or obfuscate their materials from public view.

\textbf{Registered Reports}\sidenote{\url{https://blogs.worldbank.org/impactevaluations/registered-reports-piloting-pre-results-review-process-journal-development-economics}} can help with this process where they are available.
By setting up a large portion of the research design in advance,\sidenote{\url{https://www.bitss.org/2019/04/18/better-pre-analysis-plans-through-design-declaration-and-diagnosis/}}
a great deal of work has already been completed,
and at least some research questions are pre-committed for publication regardless of the outcome.
This is meant to combat the ``file-drawer problem'',\cite{simonsohn2014p}
and ensure that researchers are transparent in the additional sense that
all the results obtained from registered studies are actually published.

\subsection{Research replicability}

Replicable research, first and foremost,
means that the actual analytical processes you usedare executable by others.\cite{dafoe2014science}
\marginnote{We use ``replicable'' and ``reproducible'' somewhat interchangeably,
referring only to the code processes themselves in a specific study;
in other contexts they may have more specific meanings.
\newline \noindent \url{http://datacolada.org/76}}
All your code files involving data construction and analysis
should be public -- nobody should have to guess what exactly comprises a given index,
or what controls are included in your main regression,
or whether or not you clustered standard errors correctly.
That is, as a purely technical matter, nobody should have to ``just trust you'',
nor should they have to bother you to find out what happens
if any or all of these things were to be done slightly differently.\cite{simmons2011false,wicherts2016degrees}
Letting people play around with your data and code is a great way to have new questions asked and answered
based on the valuable work you have already done.
Services like GitHub that expose your code \textit{history}
are also valuable resources. They can show things like modifications
made in response to referee comments; for another, they can show
the research paths and questions you may have tried to answer
(but excluded from this publication)
as a resource to others who have similar questions of their own data.

Secondly, reproducible research\sidenote{\url{https://dimewiki.worldbank.org/wiki/Reproducible_Research}}
enables other researchers to re-utilize your code and processes
to do their own work more easily in the future.
This may mean applying your techniques to their data
or implementing a similar structure in a different context.
As a pure public good, this is nearly costless.
The useful tools and standards you create will have high value to others.
If you are personally or professionally motivated by citations,
producing these kinds of resources will almost certainly lead to that as well.
Therefore, your code should be written neatly and published openly.
It should be easy to read and understand in terms of stucture, style, and syntax.
Finally, the corresponding dataset should be openly accessible
unless for legal or ethical reasons it cannot.\sidenote{\url{https://dimewiki.worldbank.org/wiki/Publishing_Data}}

%------------------------------------------------

\section{Ensuring privacy and security in research}

Anytime you are collecting \textbf{primary data} in a development research project,
\index{primary data}
you are almost certainly handling data that include \textbf{personally-identifying information (PII)}.
\index{personally-identifying information}
Most of the field research done in development involves human subjects -- real people.\sidenote{
\url{https://dimewiki.worldbank.org/wiki/Human_Subjects_Approval}}
\index{human subjects}
As a researcher, you are asking people to trust you with personal information about themselves:
where they live, how rich they are, whether they have committed or been victims of crimes,
their names, their national identity numbers, and all sorts of other data.
PII data carries strict expectations about data storage and handling,
and it is the responsibility of the research team to satisfy these expectations.\sidenote{
\url{https://dimewiki.worldbank.org/wiki/Research_Ethics}}
Your funder or employer will most likely require you to hold a certification from a source
such as Protecting Human Research Participants\sidenote{
\url{https://humansubjects.nih.gov/sites/hs/phrp/PHRP_Archived_Course_Materials.pdf}}
or the CITI Program\sidenote{
\url{https://about.citiprogram.org/en/series/human-subjects-research-hsr/}}.

PII data contains information that can, without any transformation, be used to identify
individual people, households, villages, or firms that were included in \textbf{data collection}.
\index{data collection}
This includes names, addresses, and geolocations, and extends to personal information
\index{geodata}
such as email addresses, phone numbers, and financial information.
\index{de-identification}
In some contexts this list may be more extensive --
for example, if you are working in a small environment,
someone's age and gender may be sufficient to identify them
even though these would not be considered PII in a larger context.
Therefore you will have to use careful judgment in each case
to decide which pieces of information fall into this category.\sidenote{
\url{https://sdcpractice.readthedocs.io/en/latest/}}

\subsection{Handling personally-identifying information}

Raw data which contains PII \textit{must} be \textbf{encrypted}\sidenote{\url{https://dimewiki.worldbank.org/wiki/encryption}}
\index{encryption}
during collection, storage, and transfer.
\index{data transfer}\index{data storage}
Most modern data collection software makes the first part straightforward.\sidenote{\url{https://dimewiki.worldbank.org/wiki/SurveyCTO_Form_Settings}}
However, secure storage and transfer are your responsibility.
One method for doing this is to keep raw data in a secure cloud location such as \textbf{Amazon Web Services S3}.
Another is to use an enterprise-grade solution such as \textbf{Microsoft OneDrive}; but these can be expensive.
\marginnote{Dropbox is not a secure storage location by default, and neither is GitHub.}
\index{Dropbox}\index{GitHub}
Depending on your specific security requirements,
it may be acceptable to create an encrypted version of the dataset
using software like \textbf{VeraCrypt}\sidenote{\url{https://www.veracrypt.fr/}}, and store that in \textbf{Dropbox}.
\index{VeraCrypt}
To securely transfer information,
\marginnote{Email is \textit{never} a secure data transfer method, unless the file is password-protected or encrypted.}
you can use Dropbox with encrypted data only,
or you can use another service such as \textbf{WeTransfer}\sidenote{\url{https://wetransfer.com}},
or share a password-protected \texttt{.zip} file created by \textbf{7-Zip}\sidenote{\url{https://www.7-zip.org}},
\textbf{WinRAR}\sidenote{\url{https://rarlab.com/download.htm}},
or on \textbf{MacOS Terminal} by writing \texttt{zip -er [archive.zip] [folder]}.
\index{password-protection}

In general, though, you shouldn't need to handle PII data very often.
Once data is securely collected and stored, the first thing you will generally do is \textbf{de-identify} it.\sidenote{\url{https://dimewiki.worldbank.org/wiki/De-identification}}
\index{de-identification}
\marginnote{De-identified data should avoid, for example, you being sent back to every household to alert them that someone dropped all their personal information on a public bus and we don't know who has it.}
This simply means creating a copy of the data that contains no personally-identifiable information.
This data should be an exact copy of the raw data,
except it would be okay for it to be publicly released.\cite{matthews2011data}

\subsection{Security and password management}

If you are handling data properly, it is going to end up password-protected in one way or another.
\index{password-protection}
In a collaborative, long-term project, you need to be able to store and share these passwords.
Therefore you are going to have to get smart about password management and account security.
We recommend a service like LastPass.\sidenote{\url{https://www.lastpass.com}}
LastPass is free for personal features and cheap for sharing features.
It allows you to store an unlimited number of passwords, and you will have many.
\index{password management}
\marginnote{Your personal accounts also contain an awful lot of sensitive information.
We recommend that you use strong passwords and two-factor authentication for all of your online accounts.}
It also allows you to store small encryption keyfiles from services like SurveyCTO safely, attached to ``secure notes''.
\index{encryption}
You will have one (or more) for every website, for every project, for every encrypted dataset, and so on.
A service like LastPass allows you to store and share these passwords inside its own ecosystem,
since emailing passwords is just as vulnerable to theft or loss as writing them down.

LastPass also provides an app-based implementation for
\textbf{two-factor authentication (2FA)}\sidenote{\url{https://lastpass.com/auth/}}
\index{two-factor authentication}
both for its own service and for other services you use (Gmail, Dropbox, Facebook, Amazon, etc.),
and we recommend you enable this feature for both your business and personal accounts.
Two-factor authentication means you have to enter a second confirmation using your mobile device
to log into any of your password-protected services from new locations.
This is important not because anyone is up to anything nefarious,
but because random hacking is everywhere these days
and can arbitrarily damage, delete, or lock you out of your files in the worst case.
