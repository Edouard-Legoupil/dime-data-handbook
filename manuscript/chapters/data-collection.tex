%------------------------------------------------

\begin{fullwidth}
Most data collection is now done using digital data entry
using tools that are specially designed for surveys.
These tools, called ``computer-assisted personal interviewing'' (CAPI) softwares,
provide a wide range of features designed to make
implementing even highly complex surveys easy, scalable, and secure.
However, these are not fully automatic:
you still need to actively design and manage the survey.
Each software has specific practices that you need to follow
to enable features such as Stata-compatibility and data encryption.

While you can work in any software you like,
this guide will present tools and best practices
for working with SurveyCTO, a proprietary implementation of Open Data Kit (ODK).
Most of the processes below will also replicate in ODK
with minimal adjustment for the exact setup you have.
The important parts, of course, are primarily conceptual:
this chapter should provide a motivation for
planning data structure during survey design,
developing surveys that are easy to control for quality and security,
and having proper file storage ready for sensitive PII data.
\end{fullwidth}

%------------------------------------------------

\section{Primary data collection with SurveyCTO}

\subsection{Questionnaire design}

SurveyCTO surveys are primarily created in Excel or Google Sheets,
making them one of the few outputs for which no coding is required.
However, since they make extensive use of logical structure and
relate directly to the data that will be used later,
both the field team and the data people should
collaborate to make sure that the survey suits all needs.

\subsection{Secure data collection}

Any established data collection service will always encrypt
all data submitted from the field automatically while in transit
(ie, upload or download), so if you use servers hosted by SurveyCTO
this is nothing you need to worry about.
Your data will be encrypted from the time it leaves the device
(in tablet-assisted data collation) or your browser (in web data collection)
until it reaches the server.

\marginnote{Encryption at rest is the only way to ensure
that PII data remains private when it is stored
on someone else’s server on the open internet.
The World Bank’s and many of our donors’ security requirements
for data storage can only be fulfilled by this method.
We recommend keeping your data encrypted whenever PII data is collected --
therefore, we recommend it for all field data collection.}

Encryption in cloud storage, by contrast, is not enabled by default.
This is because the service will not encrypt user data unless you confirm
you know how to operate the encryption system and assume its risks.
Encryption at rest is different from password-protection:
encryption at rest makes the underlying data itself unreadable,
even if accessed, except to users who have a specific private key file.
Encryption at rest requires active participation from you, the user,
and you should be fully aware that if your private key is lost,
there is absolutely no way to recover your data.

To enable data encryption, you simply select the encryption option
when you create a new form on a SurveyCTO server.
At that time, the service will allow you to download -- once --
the keyfile pair needed to decrypt the data.
You must download and store this in a secure location.
Again, we recommend LastPass, a software whose free option
allows you to store passwords as well as small keyfiles like this.
You should make sure the ``secure note'' which you put the keyfile in
is descriptively named to match the survey to which it corresponds.
The Sync app will ask you for the location of this file
when you download and set up data for use,
and all you will need to do is copy that file to your desktop,
point Sync to it, and the rest is automatic.

Finally, you should ensure that all teams take basic precautions
to ensure the security of data, as most problems are due to human error.
Most importantly, all computers, tablets, and accounts used
\textit{must} have a logon password associated with them.
This policy should also be applied to physical data storage
such as flash drives and hard drives;
similarly, files sent to the field containing PII data
such as the sampling list should at least be password-protected.
This step significantly mitigates the risk in case there is
a security breach such as loss, theft, hacking, or a virus,
and adds very little hassle to utilization.


%------------------------------------------------

\section{Field management and quality assurance}

%------------------------------------------------

\section{Managing primary data}
