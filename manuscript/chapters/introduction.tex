
Welcome to Data for Development Impact.
This book is intended to serve as a resource guide
for people who collect or use primary data for development research.
In particular, the book is intended to guide the reader
through the process of research using primary survey data,
from survey design to fieldwork to data management to analysis.

This book will not teach you econometrics or epidemiology or agribusiness.
This book will not teach you how to design an impact evaluation.
This book will not teach you how to do data analysis, or how to code.
There are lots of really good resources out there for all of these things,
and they are much better than what we would be able to squeeze into this book.

What this book will teach you is how to think about data,
keeping in mind that you are not going to be the only person
collecting it, using it, or looking back on it.
We hope to provide you two key tools by the time you finish this book.
First, we want you to form a mental model of data collection as a ``social process'',
in which many people need to have the same idea about what is done, and when and where it is done,
so that they can collaborate effectively on large, long-term projects.
Secondly, we want you to gain a sense of best practices for coding these processes in Stata.
As research teams and timespans have grown dramatically over the last decade,
it has no longer become efficient for everyone to have their own personal style
dictating how they use different functions, how they store data, and how they write code.

The team responsible for this book is known as DIME Analytics. \cite{dimeanalytics}
DIME Analytics works within the Development Impact Evaluation unit (DIME) \cite{dime}
at the World Bank's Development Economics Research group (DEC). \cite{dec}
DIME Analytics supports all of the research projects operated under the DIME umbrella,
and each of its members also works on their own development research and data science projects.
This book was conceived as a complement to the DIME Wiki, \cite{dimewiki}
which is a free online collection of resources and best practices.
DIME Analytics has built up these ideas, best practices, and software tools
over the course of almost ten years of supporting hundreds of projects and staff in total.

In the time that we have been working in this field,
the proportion of projects that rely on primary empirical data has soared. \cite{angrist2017economic}
Today, the scope and scale of those projects are expanding rapidly,
meaning that more and more people are working on the same data over longer and longer timeframes.
The main lesson that we have learned working in the field over this period is that
the most important part of primary data work is collaboration.
You will be working with people who have very different skillsets and mindsets than you,
from and in a variety of cultures and contexts, and you will have to agree on workflows
that work for everyone and save time and hassle on every project.

This is not easy. But for some reason, the people who agreed to write this book enjoy doing it.
(In part this is because it has saved us ourselves a lot of time and effort.)
As we have worked with more and more DIME recruits -- economists, field managers, and research assistants --
we have realized that we barely have the time to give everyone the attention they deserve.
This book itself is therefore intended to be a labor-saving process for us:
we hope one day we can hand new people a copy of our accumulated knowledge,
doing a better job teaching them the ropes than any one of us could (and saving us all some time).

This book complements the detailed-but-unstructured DIME Wiki
with a guided tour of the major tasks that accompany primary data collection.
We will not give a lot of specific details in this text;
but we will point you to where they can be found.
Instead, each chapter will focus on one task, and give a primarily narrative account of:

* What you will be doing
* Where in the workflow this task falls
* When it should be done
* Who you will be working with
* Why this task is important
* How to implement the most basic form of the task

For the implementation portion, we will provide ``minimal'' code examples, like the following:

\marginnote[2\baselineskip]{This is a Stata code block.
Throughout this book, you will find examples like this,
which illustrate how to perform basic versions
of each task in Stata.}

\begin{Verbatim}[frame=lines,numbers=left,label=code.do]
// Load the auto dataset
sysuse auto.dta , clear

// Run a simple regression
reg price mpg

// Store the output
matrix results = r(table)'

// Load the results into memory
clear
  svmat results , n(col)
\end{Verbatim}

We have tried really hard to make sure that all the Stata code runs,
and that each block should be well-formatted and use built-in functions.
If it does not, we will make sure to install the necessary user-written packages,
but we will use these only where absolutely essential.
Providing some standardization to Stata code style is also a goal of this book,
since groups are collaborating on code in Stata more than ever before.

However, we will not explain Stata commands unless the behavior we are exploiting
is outside the usual expectation of its functionality;
we will comment the code generously (as you should),
but you should reference Stata help-files \texttt{h -command-}
whenever you do not understand the functionality that is being used.

%------------------------------------------------

\section{Figures}

\lipsum[1]

\begin{marginfigure}
\includegraphics[width=\linewidth]{helix}
\caption{This is a margin figure. The helix is defined by $x = \cos(2\pi z)$, $y = \sin(2\pi z)$, and $z = [0, 2.7]$. The figure was drawn using Asymptote (\url{http://asymptote.sf.net/}).}
\label{fig:marginfig}
\end{marginfigure}

\lipsum[2]

\begin{figure*}[h]
\includegraphics[width=\linewidth]{sine.pdf}
\caption{This graph shows $y = \sin x$ from about $x = [-10, 10]$.
\emph{Notice that this figure takes up the full page width.}}
\label{fig:fullfig}
\end{figure*}

\lipsum[3]

%------------------------------------------------

\section{Tables} \marginnote{This is a random margin note. Notice that there isn't a number preceding the note, and there is no number in the main text where this note was written. Use \texttt{sidenote} to use a number.}

\lipsum[4]

\begin{table} % Add the following just after the closing bracket on this line to specify a position for the table on the page: [h], [t], [b] or [p] - these mean: here, top, bottom and on a separate page, respectively
\centering % Centers the table on the page, comment out to left-justify
\begin{tabular}{l c c c c c} % The final bracket specifies the number of columns in the table along with left and right borders which are specified using vertical bars (|); each column can be left, right or center-justified using l, r or c. To specify a precise width, use p{width}, e.g. p{5cm}
\toprule % Top horizontal line
& \multicolumn{5}{c}{Growth Media} \\ % Amalgamating several columns into one cell is done using the \multicolumn command as seen on this line
\cmidrule(l){2-6} % Horizontal line spanning less than the full width of the table - you can add (r) or (l) just before the opening curly bracket to shorten the rule on the left or right side
Strain & 1 & 2 & 3 & 4 & 5\\ % Column names row
\midrule % In-table horizontal line
GDS1002 & 0.962 & 0.821 & 0.356 & 0.682 & 0.801\\ % Content row 1
NWN652 & 0.981 & 0.891 & 0.527 & 0.574 & 0.984\\ % Content row 2
PPD234 & 0.915 & 0.936 & 0.491 & 0.276 & 0.965\\ % Content row 3
JSB126 & 0.828 & 0.827 & 0.528 & 0.518 & 0.926\\ % Content row 4
JSB724 & 0.916 & 0.933 & 0.482 & 0.644 & 0.937\\ % Content row 5
\midrule % In-table horizontal line
\midrule % In-table horizontal line
Average Rate & 0.920 & 0.882 & 0.477 & 0.539 & 0.923\\ % Summary/total row
\bottomrule % Bottom horizontal line
\end{tabular}
\caption{Table caption text} % Table caption, can be commented out if no caption is required
\label{tab:template} % A label for referencing this table elsewhere, references are used in text as \ref{label}
\end{table}

%----------------------------------------------------------------------------------------

\mainmatter
