%------------------------------------------------

\begin{fullwidth}
Increasingly, research assistants are relied to manage some or all
of the publication process. This can include
managing the choice of software,
coordinating referencing and bibliography,
tracking changes across various authors and versions,
and preparing final reports or papers for release or submission.
Fortunately, modern software tools can make a lot of these processes easier,
but naturally there is some learning curve,
particularly for lead authors who have been publishing for a long time.
This chapter is intended to suggest some tools and processes
that can make writing and publishing in a team
a significantly easier process.
It should also give you the resources you need
to judge how best to adapt your team to the tools you agree upon,
since all teams vary in composition and technical experience.

Ideally, your team will be able to spend as little time as possible
fussing with the technical requirements of publication.
It is in nobody's interest for a skilled and busy RA
to spend days re-numbering references (and it can take days)
if a small amount of up-front effort could automate the task.
However, experienced academics will likely have a workflow
that they are already comfortable with,
and since they have worked with many others in the past,
that workflow is likely to be the least-common-denominator:
Microsoft Word with tracked changes.
This chapter will show you how you can avoid at least some
of the pain of Microsoft Word,
while still providing materials in the format
that co-authors need to feel comfortable and journals request.
\end{fullwidth}

%------------------------------------------------

\section{Collaborating on academic writing}

The gold standard for academic writing is \LaTeX.
\LaTeX\ allows automatically-organized sections like titles and bibliographies,
imports tables and figures in a dynamic fashion,
and can be version controlled using Git.
Unfortunately, \LaTeX\ can be a real pain to set up and use,
particularly for people who are unfamiliar with plaintext, code, or file management.
\LaTeX\ requires that all formatting be done in a special code language,
and it is not particularly informative when you do something wrong.
This can be off-putting very quickly for people
who simply want to get to writing, like lead authors.
Therefore, if we want to take advantage of the features of \LaTeX,
without getting stuck in the weeds of it,
we will need to adopt a few tools and tricks to make it effective.

The first is choice of software. The easiest way
for someone new to \LaTeX\ to be able to ``just write''
is typically the web-based Overleaf suite.\sidenote{\url{https://www.overleaf.com}}
This is because Overleaf offers a ``rich text'' editor
that behaves pretty similarly to familiar tools like Word.
\marginnote{TeXstudio and atom-latex are the most popular
desktop-based tools for writing \LaTeX;
they allow more advanced integration with Git,
among other advantages, but the entire team needs to be comfortable
with \LaTeX\ before adopting one of these tools.
\newline\noindent \url{https://www.texstudio.org}
\newline\noindent \url{https://atom.io/packages/atom-latex}}
With minimal workflow adjustments, you should be able
to show coauthors how to write and edit in Overleaf,
so long as you make sure you are always available to troubleshoot
the inevitable \LaTeX\ crashes and errors.
(The most common will be special characters, namely
\texttt{\&}, \texttt{\%}, and \texttt{\_},
which need to be ``escaped'' or ``interpreted literally''
by writing a backslash (\texttt{\textbackslash}) before them,
such as  \texttt{40\textbackslash\%}.)
Overleaf offers a convenient selection of templates,
so it is very easy to start up a project
and replicate a lot of the underlying setup code.

The most important tool available is the BibTeX bibliography manager.
This tool allows for the storage of unformatted references
in an accompanying \texttt{.bib} file, and insertion in text
using the \texttt{\textbackslash cite\{\}} command.
With this structure, \LaTeX\ will automatically pull
all the citations into text; it will allow you to specify
how they should be displayed in text
(ie, as superscripts, inline references, etc.)
and how the bibliography should be styled and in what order.

With these tools, you can ensure that co-authors are writing
in a format you can manage and control.
The purpose of this setup, just like with other synced folders,
is to avoid there ever being more than one master copy of the document.
This means that people can edit simultaneously without fear of conflicts,
and it is never necessary to ``resolve'' differences in the document.
Finally, \LaTeX\ has one more useful trick:
if you download the Citation Styles Library\sidenote{
\url{https://github.com/citation-style-language/styles}}
and use pandoc\sidenote{\url{http://pandoc.org/}},
you can translate the raw document into Word by running:
\texttt{[pandoc -s -o main.docx main.tex --bibliography sample.bib --csl=/.../styles/the-lancet.csl]}.
Therefore, even in the case where you are requested to provide
\texttt{.docx} versions of materials to others, or tracked-changes versions,
you can create them effortlessly,
and use Word's compare feature to generate a single integrated tracked version.

%------------------------------------------------

\section{Publishing data and code for replication}
